\documentclass[11pt,oneside]{report} % one-sided or two-sided hardcopy
%%%%%%%%%%%%%%%%%%%%%%%%%%%%%%%%%%%%%%%%%%%%%%%%%%%%%%%%%%%%%%%%%%%%%%
%% uncomment these two lines when running pdflatex (i.e. pdf output)
\usepackage[pdftex]{color}
\usepackage[pdftex,colorlinks,breaklinks,backref]{hyperref}
%%
%% uncomment this line when running normal latex (i.e. dvi output)
%\usepackage{hyperref}
%%
%% comment all three of the above lines for latex2html output
%%%%%%%%%%%%%%%%%%%%%%%%%%%%%%%%%%%%%%%%%%%%%%%%%%%%%%%%%%%%%%%%%%%%%%
%% all the rest of the feffdoc stuff:
\usepackage{feffdoc}
\usepackage{html} % put after hyperref to avoid conflict in pdf mode
\usepackage{graphicx}
\usepackage{float}
\usepackage{amsmath}
\newcommand{\norm}[1]{\left\Vert#1\right\Vert}
\newcommand{\abs}[1]{\left\vert#1\right\vert}
\newcommand{\bra}[1]{\left<#1\right\vert}
\newcommand{\ket}[1]{\left\vert#1\right>}
\newcommand{\braket}[2]{\left<#1\vert#2\right>}


%%%%%%%%%%%%%%%%%%%%%%%%%%%%%%%%%%%%%%%%%%%%%%%%%%%%%%%%%%%%%%%%%%%%%%
%% these lines handle html and pdf issues that cannot be addressed in
%% the feffdoc.sty file
\bodytext{bgcolor=white}

\begin{latexonly}
  \renewcommand{\htmladdnormallink}[2]{\href{#2}{#1}}
\end{latexonly}
%% moved outside of latexonly to fix card linking problems with \htmlref
%% it's not clear why it's an issue
\renewcommand{\htmlref}[2]{\hyperlink{#2}{#1}}


%%%%%%%%%%%%%%%%%%%%%%%%%%%%%%%%%%%%%%%%%%%%%%%%%%%%%%%%%%%%%%%%%%%%%%
%% This portion of the file is used to generate card documentation 
%% for JFEFF (the GUI) using latex2html. It is included here since 
%% neither \usepackage{feffdoc} nor \include can be used. 
%% 
%%%%%%%%%%%%%%%%%%%%%%%%%%%%%%%%%%%%%%%%%%%%%%%%%%%%%%%%%%%%%%%%%%%%%%
%% On a system with latex2html installed, 
%% use on the command line: 
%% > latex2html feffXX[this file].tex
%%
%% Be sure to comment out the above three lines used for dvi/pdf.
%%
%% Warnings should include: 
%% No implementation found for style 'feffdoc'
%% redefining command \htmlref
%% Substitution of arg to newlabelxx delayed.
%% Unknown commands: dotfill tightlist bullet
%%
%% if there is no feffXX.aux present, first run regular latex:
%% > latex feffXX.tex
%%%%%%%%%%%%%%%%%%%%%%%%%%%%%%%%%%%%%%%%%%%%%%%%%%%%%%%%%%%%%%%%%%%%%%
%% The document uses htmlonly and latexonly environments to preserve 
%% compatibility with both html and single-file output.
%%
%% These environments have been used to remove references to other 
%% sections that are not appropriate to card descriptions to be used 
%% in JFEFF. They also prevent outputting anything but the chapter 
%% on FEFF control cards, when using latex2html.
%%
%% This is also done in some cases because latex2html doesn't check for 
%% duplicates and the card html filenames should be of the form: 
%% CARDNAME.html
%% (set by $LONG_TITLES = 1 in .latex2html-init)
%%
%% Note that this file won't result in a complete set of html documentation
%% (but it is a start, since latex2html won't choke on it anymore).
%%%%%%%%%%%%%%%%%%%%%%%%%%%%%%%%%%%%%%%%%%%%%%%%%%%%%%%%%%%%%%%%%%%%%%
%% The file .latex2html-init sets the important parameters for 
%% formatting the latex2html output, so it should be present in the 
%% "home/user" directory. alternatively, the same parameters can be 
%% set on the whole latex2html installation, but this is not advisable.
%%%%%%%%%%%%%%%%%%%%%%%%%%%%%%%%%%%%%%%%%%%%%%%%%%%%%%%%%%%%%%%%%%%%%%
%% ::8/24/06 - Bryan Schultz::
%% 2009-2013 - Kevin Jorissen - many updates
%%%%%%%%%%%%%%%%%%%%%%%%%%%%%%%%%%%%%%%%%%%%%%%%%%%%%%%%%%%%%%%%%%%%%%
\begin{htmlonly} %%ignored by regular latex, for use of latex2html
  \HTMLset{TITLE}{FEFF10 Documentation}
  % mimics feffdoc.sty, but also contains important modifications:
  \newcommand{\program}[1]{\textsc{#1}}
  \newcommand{\feff}{\program{feff}}
  \newcommand{\vnum}{10}
  \newcommand{\feffcur}{\feff\vnum}
 
  %%% Card environment for describing a feff keyword - modified for GUI
    \newenvironment{Card}[4]%
      {\vspace{3ex}%
        \subsection{#1}
        %\hypertarget{card:#4}{}%
        \quad\textsl{#3}\newline
        \quad\texttt{#2}\newline%
        \label{card:#4}\\}
      {}  
  %%% miscellaneous definitions
                                 % typographic conventions
  \newcommand{\file}[1]{`\texttt{#1}'}
  \newcommand{\module}[1]{\textrm{\bf{#1}}} 
  \newcommand{\command}[1]{\texttt{\textsl{#1}}}
                                  % program names
  \newcommand{\atoms}{\program{atoms}}
  \newcommand{\feffit}{\program{feffit}}
                                  % other redefined commands
  \renewcommand{\Re}{\operatorname{Re}}
  \renewcommand{\Im}{\operatorname{Im}}
  \renewcommand{\htmlref}[2]{{#1}} % (linking currently disabled)
\end{htmlonly}
%%%%%%%%%%%%%%%%%%%%%%%%%%%%%%%%%%%%%%%%%%%%%%%%%%%%%%%%%%%%%%%%%%%%%%
%%%%%%%%%%%%%%%%%%%%%%%%%%%%%%%%%%%%%%%%%%%%%%%%%%%%%%%%%%%%%%%%%%%%%%









%%%%%%%%%%%%%%%%%%%%%%%%%%%%%%%%%%%%%%%%%%%%%%%%%%%%%%%%%%%%%%%%%%%%%% BEGIN
%%%%%%%%%%%%%%%%%%%%%%%%%%%%%%%%%%%%%%%%%%%%%%%%%%%%%%%%%%%%%%%%%%%%%% BEGIN
\begin{document}
\begin{latexonly}%ends at end of chapter

\pagenumbering{roman}           % Roman numerals in front matter
\MakeTitle                      % write out a title page


\newchapter{}

\begin{abstract}
  {\feff} is an {\it ab initio} multiple-scattering 
code for calculating excitation spectra and 
electronic structure.  It is based on a real space Green's
function approach including a screened core-hole, inelastic losses and
self-energy shifts, and Debye-Waller factors. Optionally
{\feff} includes local fields (TDLDA) and an approximate treatment
of multi-electron excitations. 
The spectra include extended x-ray absorption fine structure (EXAFS),
x-ray absorption near edge structure (XANES), x-ray 
natural and magnetic circular dichroism (XNCD and XMCD), spin polarized 
x-ray absorption spectra (SPXAS and SPEXAFS), non-resonant x-ray emission 
spectra (XES), Compton scattering, Non-Resonant Inelastic X-ray Scattering (NRIXS), and the x-ray scattering amplitude (including Thomson and 
anomalous parts). In addition the code can treat relativistic
electron energy loss spectroscopy (EELS).   {\feffcur} is written in
Fortran 90 and can be run from a Java based graphical interface.


  This work has been supported in part by grants from the U.S.\ 
  Department of Energy and by the University of Washington
  Office of Technology Transfer. {\feff} is copyright \copyright\
  1992--2013, The {\feff} Project, Department of Physics, University
  of Washington, Seattle, WA 98195-1560.

  \vspace*{\stretch{1}}

	\noindent This document is copyright \copyright\ 2002-2013 by J.J.\
Rehr, K.\ Jorissen, A.\ Ankudinov, and B.\ Ravel.
	  
  \noindent Permission is granted to reproduce and distribute this
  document in any form so long as the copyright notice and this
  statement of permission are preserved on all copies.
  
  \noindent This document can also be found on the \htmlref{feff project website}
 {http://www.feffproject.org}.
 
\end{abstract}

\newchapter{}
\tableofcontents
\newchapter{}

\setcounter{page}{1}
\pagenumbering{arabic}








%%%%%%%%%%%%%%%%%%%%%%%%%%%%%%%%%%%%%%%%%%%%%%%%%%%%%%%%%%%%%%%%%%%%%%%%%%%%%%%%% SYNOPSIS
%%%%%%%%%%%%%%%%%%%%%%%%%%%%%%%%%%%%%%%%%%%%%%%%%%%%%%%%%%%%%%%%%%%%%%%%%%%%%%%%% SYNOPSIS

\chapter{Synopsis}
\label{sec:Synopsis}

{\feffcur} calculates spectroscopic properties, including
extended x-ray-absorption fine structure (EXAFS), 
x-ray-absorption near-edge structure (XANES), x-ray
natural circular dichroism (XNCD), spin-dependent calculations 
of x-ray magnetic dichroism (XMCD) and spin polarized x-ray absorption 
(SPXAS and SPEXAFS), Compton scattering, nonresonant x-ray emission (XES), and Non-Resonant
Inelastic X-ray Spectroscopy (NRIXS). 
In addition the code calculates electronic structure including local
densities of states (LDOS), and the x-ray elastic scattering amplitude
$f=f_0+f'+if''$ including Thomson and anomalous parts, and relativistic
electron energy loss spectroscopy (EELS). 

{\feff} uses an \textit{ab initio} self-consistent real space multiple 
scattering (RSMS) approach, including 
polarization dependence, core\--hole effects, and local field 
corrections, based on self-consistent, spherical muffin-tin scattering
potentials.  Calculations are based on an all-electron, real space 
relativistic Green's function formalism with no symmetry requirements.
The code builds in inelastic losses in terms of a GW self-energy,
and includes vibrational effects in terms of Debye-Waller
factors.  For periodic structures reciprocal space calculations based on
periodic boundary conditions are also available.
{\feff} can use both full multiple scattering based on LU or Lanczos 
algorithms and a high-order path expansion based on the Rehr--Albers 
multiple scattering formalism.

The name {\feff} is derived from $f_{eff}$, the effective curved wave
scattering amplitude in the modern EXAFS equation.  This was the first
application of the {\feff} code, and is the basis for the 
multiple-scattering path-expansion in the code.

For a quick start or self-guided tutorial
we suggest that new users study the tutorial chapter \ref{sec:tutorial} and try a few of the examples in Section
\ref{sec:Calc-Strat-Exampl}. For details on use of the code, examples
and suggestions for calculation strategies, see Sections
\ref{sec:Calc-Strat-Exampl}, \ref{sec:Input-and-Output-Files}, and \ref{sec:Input-Control-Cards}. For details about the algorithms,
see the discussion for the appropriate module in Section
\ref{sec:Input-Control-Cards}.  For additional details, see the published
references in Appendix~\ref{sec:Append-C-Refer} and the {\feff} website.

{\feff} is written in ANSI Fortran 90. It typically requires 250 MB - 2GB of available memory (RAM). More memory may be needed on MS Windows systems or for very large calculations.  See Appendix~\ref{sec:Append-B-Inst} for installation instructions.  Many problems run well on a laptop.  For larger calculations of XANES, NRIXS, or COMPTON a small cluster may be more appropriate.  The JFEFF Graphical User Interface requires a recent version of Java.

Please contact the authors concerning any problems with the code. See
Appendix~\ref{sec:Append-H-Trouble} for trouble-shooting hints and
problem/bug reports or the FAQ on the \htmlref{{\feff} website}
  {http://www.feffproject.org}.




\section{Acknowledgements and history}
\label{Acknowledgements}

The Principal Investigator of the {\feff} project is:
\begin{description}
\item[John J.~Rehr, Principal Investigator]\hfill\\
Department of Physics, BOX 351560 University of Washington, Seattle, WA 98195\\
email: \htmladdnormallink{jjr@uw.edu}
{mailto:jjr@phys.washington.edu}\\
telephone: (206) 543-8593,\  FAX: (206) 685-0635
\end{description}


The full {\feffcur} code is copyright protected software and users must
obtain a license from the University of Washington Office of Technology
Transfer for its use. 
See Appendix~\ref{sec:Append-A-Copyr} for
complete copyright notice and other details. Documentation and
information on how to obtain the code can be found at the {\feff}
Project website:

\centerline{\htmladdnormallink{http://www.feffproject.org}
  {http://www.feffproject.org}}

\noindent or by e-mail to the {\feff} Project Coordinator at
\htmladdnormallink{feff@uw.edu}{mailto:feff@uw.edu}.

Please cite an appropriate {\feff} reference if the code
or its results are used in published work. The main reference for calculations using the current version {\feffcur}  is:

\emph{Parameter-free calculations of x-ray spectra with FEFF9},
J.J. Rehr, J.J. Kas, F.D. Vila, M.P. Prange, K. Jorissen,
Phys. Chem. Chem. Phys., 12, 5503-5513 (2010).


The main references for the theory behind {\feff} are:

\emph{Ab initio theory and calculations of X-ray spectra},
 J.J. Rehr, J.J. Kas, M.P. Prange, A.P. Sorini, Y. Takimoto, F.D. Vila, Comptes Rendus Physique 10 (6) 548-559 (2009).
 
and for EXAFS theory:

\emph{Theoretical Approaches to X-ray Absorption Fine Structure},
J. J. Rehr and R. C. Albers, Rev. Mod. Phys. {\bf72}, 621, (2000).
 
See Appendix~\ref{sec:Append-C-Refer}  or the feffproject website for a list of additional references.

 
 

\medskip

{\bf FEFF Project Developers}

 The core developers team for {\feffcur} consists of the following people :
\begin{itemize}
\label{developers} 
\item {\bf Kevin Jorissen} implemented relativistic EELS calculations, k-space algorithms, cloud computing, and many more;
\item {\bf Joshua Kas} developed the many-pole self-energy and improved treatments of inelastic losses, RIXS calculations, and many more;
\item {\bf Fernando D.\ Vila} implemented improved Debye-Waller factors;
\item {\bf Brian Mattern} implemented COMPTON calculations;
\item {\bf Towfiq Ahmed} is the developer of the Hubbard U module;
\item {\bf Micah Prange} developed the {\feff}/optical constant extension; 
\item {\bf Yoshinari Takimoto} implemented the RPA core hole;
\item {\bf Adam Sorini} investigated inelastic mean free paths;
\item {\bf Aleksi Soininen} implemented q-dependent NRIXS calculations;
\item {\bf Alexei Ankudinov} added TDLDA and automated XMCD calculations.
\end{itemize}

The JFEFF GUI was originally developed by {\bf David Bitseff}. 
Later, remote ssh, MPI, and other functionality was added by
{\bf Lukas Svec}. {\bf Kevin Jorissen} expanded the GUI and added automated Cloud Computing on the Amazon EC2 cloud with {\bf William Johnson}.

Alexei Ankudinov was the principal developer for
the {\feff}8 series which included automated self-consistent potential
s, calculations of the $\ell$DOS, the Fermi level, and charge transfer.
He also added calculations of elastic scattering
amplitude, x-ray natural circular dichroism and nonresonant x-ray
emission, and quadrupolar transitions.  Bruce Ravel developed the full
multiple scattering (FMS) algorithm in {\feff}8. 
A.\ Nesvizhskii added new routines for TDLDA and sum-rules
applications. Anna Poiarkova and Patrick Konrad
contributed algorithms for calculating multiple-scattering Debye--Waller
factors and anharmonic contributions. Jim Sims (NIST) extended the code for
MPI based parallel execution, in collaboration with C.\ Bouldin and
J. Rehr.  The high-order multiple-scattering routines, pathfinder and
correlated Debye-Waller routines were developed by Steven Zabinsky and
J. Rehr for {\feff}5.
The Hedin--Lundqvist self-energy and phase shift routines were developed
by Jose Mustre de Leon, Dan Lu, J. Rehr, and R.C.~Albers.

We acknowledge invaluable feedback from many of our regular users and beta users.  In particular we thank Michel Jaouen, Sergio Moreno, Rainer Wilcke, and many others for testing the current version {\feffcur}.  We also thank R.C.\
Albers, K.\ Baberschke, C.\ Bouldin, C.\ Brouder, G.\ Brown,
S.D.\ Conradson, F.\ Farges, G.\ Hug, M.\ Jaouen, J. Sims, H.\ Ebert and
E.\ Stern for useful comments and suggestions.

References for previous {\feff} versions and implementations mentioned above can be found in the appendix.



\newpage
\section{The FEFF Users Guide}
\label{sec:UG}

The current {\feff} Users Guide is organized differently from previous versions, partly because of
the increased number of input cards and program modules ; and partly because
we have introduced a Graphical User Interface or GUI referred to as JFEFF.

The next chapter (2) is a \htmlref{quick start guide}{sec:tutorial} to {\feff} and shows how to run a simple calculation
using the GUI.  We recommend that new users start here.  It is also a nice introduction to the new GUI.

Chapter (3) is devoted to \htmlref{calculation material properties}{sec:Calc-Strat-Exampl}.  Here, we discuss in detail how to calculate various spectra, accounting for spin variables, obtaining XMCD spectra, and so on.  Many example input files are described.  These examples are also included in your JFEFF installation, so you can run them yourself on your computer.  This chapter is required reading for serious readers.

Chapter (4) is a reference chapter describing all of \htmlref{{\feff}'s input options}{sec:Input-Control-Cards}, which are specified through CARDS in the input file.
In previous versions of the {\feff} users guide these cards were grouped by the module of {\feff} that they ``belong" to.  
Cards are now organized in the same way that they appear in the GUI.  First, we discuss cards that specify the
system under consideration (molecule, cluster, crystal ...).  Then we discuss cards that specify which spectroscopy or material property will be calculated.  All of these cards globally affect the
{\feff} program.  Lastly, we discuss more specialized cards that really do ``belong" to a specific module of {\feff}.  This classification combines an intuitive classification for new users with efficiency for advanced users (who may not run the entire {\feff} program flow).

Chapter (5) describes \htmlref{input and output files}{sec:Input-and-Output-Files}, illustrating the data flow of the program and helping users to find the information they need in {\feff}s many output files.

Finally, the last pages of the UG comprise \htmlref{installation instructions}{sec:Append-B-Inst}, license information, and other technical information.

If you notice any mistakes or omissions in this Users Guide, please contact us at \htmlref{feff@uw.edu}{mailto:feff@uw.edu}.  We appreciate any feedback that will help us improve this document.  The most recent version of the documentation is available on the \htmlref{{\feff} website} {http://www.feffgroup.org} in pdf and wiki format.




%\newpage
\vspace*{\stretch{1}}
\begin{table}[htbp]
  \caption{Typographic conventions in this document}
  \label{tab:typographic}
  \begin{center}
    \begin{tabular}[h]{ll}
      \hline\hline
      \quad font & \quad denotes \\
      \hline
      \program{small caps} & names of programs\\
      \texttt{typewriter font} &  contents of files\\
      \file{quoted typewriter font} & file names\\
      ROMAN CAPITALS & names of cards in the \file{feff.inp} file\\
      \texttt{\textsl{slanted typewriter font}} &
      commands executed at a command line \\
      \hline\hline
    \end{tabular}
  \end{center}
\end{table}
\vspace*{\stretch{1}}
\end{latexonly} % begins at start of this chapter











%%%%%%%%%%%%%%%%%%%%%%%%%%%%%%%%%%%%%%%%%%%%%%%%%%%%%%%%%%%%%%%%%%%%%%%%%%%%%%%%% TUTORIAL
%%%%%%%%%%%%%%%%%%%%%%%%%%%%%%%%%%%%%%%%%%%%%%%%%%%%%%%%%%%%%%%%%%%%%%%%%%%%%%%%% TUTORIAL
\chapter{Tutorial}
\label{sec:tutorial}


This chapter demonstrates a simple FEFF calculation using the Graphical User Interface (GUI).  We recommend you start here.
There are more examples throughout this manual and in the examples directory of your FEFF distribution.  If you have already installed {\feff}, you may skip the next section.

JFEFF is updated regularly, and the screenshots in this manual may be  different from the actual JFEFF GUI on your system.

\section{Installing {\feff}.}
You first need to install{\feff}.  It comes bundled with its GUI JFEFF as well as this Users' Guide and examples.  The program is available from \htmlref{the {\feff} project website}{http://www.feffgroup.org}, where you can buy a license if you haven't already done so.  Once you receive your license code, you have access to the download section, where you can download the current version of {\feff}.

\begin{figure}[H]
	\centering
		\includegraphics[height=2.548in]{install1.png}
	\label{fig:install1}
\end{figure}

We offer a very straightforward step-by-step installer that should have you up and running in moments.  {\feff} should install on your computer without any difficulties.  On most computers it will ask for an Administrator password.  If you do experience problems, we would like to hear about your experience and we will help you resolve the problem.  If you receive a complaint about missing Java or the Java JRE, please download the latest Sun Java release to your computer and try again.  There are no further requirements.

For more advanced instructions that only apply to advanced users, see \htmlref{the appendices}{sec:Append-B-Inst}.

\section{Running your first {\feff} calculation}

To begin, please open the JFEFF-GUI by double-clicking the program icon on your desktop, Start Menu, or LaunchPad.  If this is the first time you run this version of {\feff}, the program will ask you for permission to copy user files to your home directory.  Please click OK to allow this.  JFEFF then installs a set of examples.  You can find them in \url{~/jfeff_examples}, where \url{~} means your home directory.  When JFEFF opens, it loads by default the EXAFS example for Cu (located in \url{~/jfeff_examples/EXAFS/Cu}).  However, we now calculate the X-ray Absorption Spectrum of GeCl4.


Select \file{Open Example} from the \texttt{File} menu.  Browse to \file{EXAFS}, then to \file{GeCl4}, and select the file \file{feff.inp}.

\begin{figure}[H]
	\centering
		\includegraphics[height=3.0in]{open1.png}
	\label{fig:open1}
\end{figure}

All corresponding files for this calculation can be found in  \url{~/jfeff_examples/EXAFS/GeCl\\_4/feff.inp}.  ("\url{~}" stands for the home directory, e.g. \url{C:\\Documents and Users\\John Rehr\\})  Each {\feff} calculation consists of an input file and a set of output files in one directory.  After loading the GeCl4 input file JFEFF looks like this:

\begin{figure}[H]
	\centering
		\includegraphics[height=4.0in]{open2.png}
	\label{fig:open2}
\end{figure}

To view the material you are calculating, click the \texttt{Jmol} button.
\begin{figure}[H]
	\centering
		\includegraphics[height=2.0in]{view1.png}
	\label{fig:view1}
\end{figure}


Take a look around.  There are many input options, corresponding to `CARDS' discussed in the next chapters.  Only a few options are explicitly initialized; many others will have default values for this simple test calculation.  The options are grouped :  In the top panel, you specify the molecule or crystal you're studying.  In the middle panel, you specify which spectrum you want to calculate (e.g., an EXAFS spectrum over a certain energy range).  The bottom panel contains a large number of options that are all the "technical settings" that tell the FEFF code how to calculate the spectrum.  It's fine if most of these don't make much sense right now.  When you read the next chapter of this Users Guide, you will quickly find out which options are most important.  Knowlege of about two dozen cards is sufficient for many {\feff} calculation.  For now, have a look at the atomic coordinates, and try to see where the type of spectroscopy and the relevant path length cutoff are set.  You can look up information on any card in chapter 4.  

There is an easier way to get help.  Click the \texttt{Help} button in the bottom right corner.  A new window pops up.  Now hover the mouse over any CARD in the interface.  A description appears in the pop-up box.  If you want to consult the \file{pdf} manual, simply click \file{Open User Manual} in the \texttt{Tools} menu.

\begin{figure}[H]
	\centering
		\includegraphics[height=3.0in]{help1.png}
	\label{fig:help1}
\end{figure}

Finally, the righthand panel is where you start the calculation.  When you have satisfied your curiosity, click the red \file{Save \& Run} button to run the example.  

Now the \file{Run} window opens and runtime output appears on the screen.  Wait until {\feff} has finished.  For this tiny calculation it will take only a few moments.  Calculations can take minutes or even hours in some cases.  Here we see that the spectrum was calculated using 4 "scattering paths", of which only 1 is unique.  This makes sense since we have a small molecule consisting of Ge symmetrically surrounded by 4 Cl atoms.  If we increased the RPATH parameter (as we probably should - it is currently set to 3.0 Angstrom) then more paths would be taken into account.
.
\begin{figure}[H]
	\centering
		\includegraphics[height=3.0in]{run1.png}
	\label{fig:run1}
\end{figure}


Now we plot the EXAFS (X-ray absorption) spectrum that we just calculated.  Click on the {\feff} logo in the right corner.
\begin{figure}[H]
	\centering
		\includegraphics[height=3.0in]{plot1.png}
	\label{fig:plot1}
\end{figure}


This plots the content of the file \file{xmu.dat}, which contains the EXAFS spectrum.  It is automatically loaded when the plot frame is opened. You can select \file{open} from the \file{file} menu in the plot frame to add another data set.  E.g. in the same directory you will find a file \file{referencexmu.dat}.  This is not a measured reference (i.e. experimental data) but output from the same calculation executed by the {\feff} authors.  It's there so you can check that you are doing the calculation correctly.  You can also display the \file{xmu.dat} output in your favorite plotting program.  As you gain more experience you can also investigate the other output files. 


It's worth playing around with the options in the plot window: you can add, hide, refresh, or delete datasets.  (The results aren't deleted from disk, merely removed from the graph.)  You can also select the data column to plot.  The plotter automatically chooses the most useful one - column 4 for EXAFS.  In the figure below we have loaded the file \file{xmu.dat} three times, selecting first the default column 4 for the total EXAFS spectrum $\mu(E)$, then column 6 for the EXAFS fine structure $\chi$, and finally column 5 for the atomic background $\mu_0$. 

\begin{figure}[H]
	\centering
		\includegraphics[height=3.0in]{plot2.png}
	\label{fig:plot2}
\end{figure}

If your results differ from the provided reference, there may be a problem with your installation that you need to correct before proceeding.


Congratulations!  You have succesfully finished your first {\feff} calculation.




\section{More examples}
Your {\feff} installation includes a folder containing examples of calculations for nearly all different types of calculations {\feff} can do.  It's worth spending some time looking at the files or opening the files in JFEFF and exploring which options were used.  Most of the examples are also discussed in the next chapter of this Users Guide.

Note that many examples have reduced basis sets that allow you to run the test quickly, but do not produce a "converged" spectrum that can be compared to experiment or used in a publication.  At least the cutoff radii of the SCF, FMS, and RPATH cards need to be converged to achieve proper results.

Each example has a self-contained directory containing the input file \file{feff.inp}, a reference spectrum file \file{xmu.dat}, and a compressed file \file{REFERENCE.zip} containing the entire calculation, including intermediary files.

Once you start setting up and running your own calculations, we recommend that you continue to launch each calculation in a separate directory to avoid problems and confusion.  (Once files get mixed up, it can be very hard to figure out what happened, and the only solution is often to recalculate everything from the beginning.)





%%%%%%%%%%%%%%%%%%%%%%%%%%%%%%%%%%%%%%%%%%%%%%%%%%%%%%%%%%%%%%%%%%%%%%%%%%%%%%%%% CALCULATION
%%%%%%%%%%%%%%%%%%%%%%%%%%%%%%%%%%%%%%%%%%%%%%%%%%%%%%%%%%%%%%%%%%%%%%%%%%%%%%%%% CALCULATION

\chapter{Calculating material properties with {\feff}}
\label{sec:Calc-Strat-Exampl}

%%%%%%%%%%%%%%%%%%%%%%%%%%%%%%%%%%%%%%%%%%%%%%%%%%%%%%%%%%%%%%%%%%%%%%%%%%% GENERAL %%%%%%%%%%%%%
\section{General Comments}

A {\feff} calculation starts with two basic tasks: calculation of the potentials, followed by calculation of the scattering phase shifts.  Using these phase shifts, scattering paths are found and their scattering amplitudes calculated.  There are two ways to do this: either using an explicit enumeration and summation, most appropriate for extended absorption spectra; or using the implicit summation of full multiple scattering, appropriate for near-edge absorption spectra.  Finally, these scattering amplitudes are combined and processed to a particular type of spectrum - currently EXAFS, XANES, ELNES, EXELFS, XES, NRIXS, DANES, COMPTON, or XMCD.

Each of these basic steps corresponds to a handful of {\feff} program modules controlled by a set of \file{feff.inp} input options.  In later chapters we discuss these aspects in a technical way by listing the details of program modules, input cards, and files.  The current chapter takes a more user-oriented approach.  We discuss each type of spectroscopy individually.  We indicate which options (cards) and program modules are relevant for that type of spectrum, and give recommendations on strategies, as well as examples.

We encourage users to spend some time exploring the \file{examples} folder that came with their {\feff} distribution.  Most users will find this in \url{\~/jfeff_examples}.  (If you installed only the {\feff} code, the examples are in \url{\~/feff90/examples}.)  This folder contains several testcases, including all major types of spectroscopy.  Each test case comes with an input file \file{feff.inp}, a reference output file \file{referencexmu.dat} or \file{reference\_eels.dat}, and an archive \file{REFERENCE.zip} containing the entire calculation with intermediate output files.  Every example that is quoted at length in this chapter of the Users Guide is available in the \url{jfeff_examples} folder.

We endeavour to keep {\feff} accessible to the novice user: you need only one input file \file{feff.inp} and run a single command, \file{feff} (or the \textbf{Save \& Run} button in the GUI).  Nevertheless, we encourage you to gain insight in the underlying structure of {\feff}.  For example, many parameters can be changed without requiring the entire calculation to be redone.  If the model (i.e., the atom positions) doesn't change significantly, it is usually not necessary to recalculate the potentials.  E.g., if one wants to add more paths to the path expansion, the calculation can be restarted with module \module{path}, saving much time.  If one wants to change the beam orientation in an EELS calculation, only the \module{eels} module needs to run again.  However, significant changes to the model require rerunning the entire calculation.  Generally, \module{pot}, \module{fms}, and \module{genfmt} take much time, while all other program modules are very fast.

The study of a material usually requires running several calculations, because there are parameters to converge, for example the cutoff radius of the full multiple scattering algorithm.  There are no default values for these parameters that work for all materials.  Debye-Waller factors may require much attention in the EXAFS regime but the XANES regime may be less sensitive.  It takes some experimentation to find out what treatment is adequate for a given material.

The information in this chapter aims to illuminate such matters; some hands-on experience will do the rest.


%%%%%%%%%%%%%%%%%%%%%%%%%%%%%%%%%%%%%%%%%%%%%%%%%%%%%%%%%%%%%%%%%%%%%%%%%%% RUNNING FEFF %%%%%%%%%%%%%
\section{Running FEFF}

\subsection{Preparing the input}
The {\feff} program consists of a set of program modules driven by a single input file \file{feff.inp} .  The user can either supply this file herself; or she can set all input options through the JFEFF GUI, which runs {\feff} without the need for manual text editing.  (A {\file{feff.inp}} is created behind the scenes and can be edited later.)

To prepare {\feff} input, you must answer four questions:
\begin{enumerate}

\item What is the material I am simulating?
\item What spectrum am I interested in?
\item How do I want {\feff} to do the simulation?
\item What parts of {\feff} do I want to run?
\end{enumerate}

The answer to question 1 requires knowledge of the structure.  In most cases, this means obtaining a list of x,y,z Carthesian atom coordinates and atomic numbers.  {\feff} is happy with an explicit list of atoms, e.g. for molecules, aperiodic systems, or as an approximation to a periodic material.  {\feff} can also read \file{.cif} files, a widely accepted format for representing crystal data.  Such files can be found for many materials in online databases and can be fed straight into the JFEFF GUI, or can be linked to in the \file{feff.inp} file.  Finally, there are some third-party tools that can set up a \file{feff.inp} file for you based on atomic positions or crystallographic information.  An example is \htmladdnormallink
{webatoms}{http://cars9.uchicago.edu/~ravel/software/aboutatoms.html}, developed by B. Ravel; or the popular Athena program.    More {\feff}-compatible crystallographic tools are available from \htmladdnormallink{ICMPE}{http://www.icmpe.cnrs.fr/spip.php?article578&lang=fr}.

Although {\feff} input can certainly be constructed from scratch, the most common approach is to start from an existing input file, and modify it to what you want to do.  Example input files included in your {\feff} installation and discussed in this chapter are an excellent starting point.  You can take such a file and replace the atomic coordinates with your own (typically, using the ATOMS and POTENTIAL cards, or equivalently the CIF card).  Then you can continue to edit the file to answer questions 2-4, or you can open the file in the JFEFF GUI, where it is easy to change options and to find help on what they mean.

As you look at the examples in this chapter, it may be helpful to refer to Chapter \ref{sec:Input-Control-Cards}, which contains an exhaustive reference list of all input options.


\subsection{Conventions for the input file}

If you use the JFEFF GUI to set up your calculation, you don't need to worry about this.

The input file \file{feff.inp} is a loosely formatted, line-oriented text file. Each type of input read by the program is on a line 
starting with a CARD:  a keyword possibly followed by some numbers or words.  The sequence of keyword cards is arbitrary. If any card or optional data is omitted,
default values are used.  Most calculations do require the POTENTIALS card, and either the ATOMS card or the OVERLAP card.  Together, the POTENTIALS and ATOMS/OVERLAP cards specify the physical structure of the
molecule or solid one is interested in.  {\feff} uses Carthesian coordinates and Angstrom units.  Hence it is easy to copy atom positions from, e.g., an \file{xyz} file or other popular file formats.  It is also possible to import CIF files (\file{*.cif}).  There are many optional cards to specify the material property to calculate (e.g., X-ray spectroscopy) and to control how {\feff} calculates it. 
 
Users working from the GUI can get more information about any CARD by pressing the \file{Help} button and simply hovering the mouse over that CARD.

All distances are in Angstrom ({\AA}) and energies are in electron-volts (eV). 


\subsection{Starting the calculation}

Once the input is ready, the calculation is started by clicking the \texttt{Save \& Run} button in the GUI, or by giving a single command
\texttt{feff} from a command line terminal.  This executes a script bundling about fifteen modules that together comprise the {\feff} code.  

Generally, not all modules are run for a given calculation.  To some extent, each step can be
controlled individually through cards or an intermediate input file (\file{pot.inp}, \file{ldos.inp}, ...).  Note that it's useless to run any module unless the preceding modules have been run already.  We recommend that novice users simply run the \texttt{feff} command until they develop a more detailed understanding of how the {\feff} code works.  

In the GUI one can easily select which modules one wants to run by checking or unchecking their checkboxes in the righthand panel of the main window.  On the command line one can either call each of the modules by name, or use the \htmlref{CONTROL}{card:con} card in \file{feff.inp} to select modules.  


\begin{figure}[H]
	\centering
		\includegraphics[height=3.0in]{control1.png}
		\caption{The user is calculating an EXAFS spectrum.  After finishing a first calculation, she increases the path length parameter RPATH to 9.0.  To see the effect on the spectrum, she only needs to re-run the \module{path list}, \module{path expansion}, and \module{cross section}.  She does not need to recalculate the potentials and phase shifts - therefore, the checkboxes of these modules are unchecked.  This way, she saves much time.}   
	\label{fig:control1}
\end{figure}

The underlying program flow of the {\feff} program is as follows:

\begin{enumerate}

\item The \file{feff.inp} file is read and checked for consistency.
Various intermediate input files are written (in the module 
\module{rdinp}, read input).

\item Ab initio Debye-Waller factors are calculated if the dynamical matrix is provided. (\module{dmdw}, dynamical matrix to debye-waller.)

\item Atomic overlap (Mattheiss) potentials are calculated.
(module \module{atomic}, atomic potentials.)

\item  The scattering potentials are calculated self-consistently using
an automated SCF loop. Absolute energies are estimated.
(module \module{pot}, potentials.)

\item RPA screening is used to calculate the core-hole potential. (module \module{screen}.)

\item  The angular momentum projected density of 
states is calculated. (module \module{ldos}.) 

\item Some intermediate output is written to file. (module \module{opconsat}.)

\item  The scattering phase shifts, dipole matrix
elements, and x-ray cross-section are calculated. 
(module \module{xsph}, cross-section and phases.)

\item  Full multiple scattering XANES/ELNES calculations are
done to obtain the full Green's function for a specified cluster size.
(module \module{fms}, full multiple scattering.)

\item The relevant elements of the FMS Green's function are traced and
multiplied by matrix elements.
(module \module{mkgtr}, trace Green's function.)

\item  The leading multiple scattering paths for the cluster are enumerated.
(module \module{path}, path enumeration.)

\item  The effective scattering amplitudes $f_{\rm eff}$ and other XAFS
parameters are calculated for each scattering path.
(module \module{genfmt}, general-path F-matrix calculation.)

\item  The XAFS parameters from one or more paths are combined to
calculate a total XAFS or XANES spectrum.
(module \module{ff2x}, scattering amplitude to $\chi$.)

\item Regular XAS output files (e.g., \file{xmu.dat})
are convolved with a many body spectral function to include many body effects.
(module \module{sfconv}.)

\item The Compton spectrum is calculated. (module \module{compton}, output in \file{compton.dat})

\item The electron energy loss spectrum is calculated. (module  \module{eels}. See Section~\ref{sec:EELS}.)

\end{enumerate}


\begin{figure}[H]
	\centering
		\includegraphics[height=3.0in]{flow1.jpg}
		\caption{Functional flow chart of the FEFF program}   
	\label{fig:flow1}
\end{figure}

\begin{figure}[H]
	\centering
		\includegraphics[height=3.0in]{flow3.jpg}
		\caption{Flow chart of the FEFF program showing all submodules.  The \module{ldos} module is not shown - it can be executed any time after the calculation of the potentials.}   
	\label{fig:flow3}
\end{figure}

\begin{figure}[H]
	\centering
		\includegraphics[height=3.0in]{flow4.jpg}
		\caption{Flow chart of the FEFF program showing all submodules as well as some of the most important input and output files.  Many other files are produced by the FEFF program depending on spectrum type and selected options.  File descriptions are in Chapter 5.}   
	\label{fig:flow4}
\end{figure}


\subsection{Studying the results}

First the quality of the calculation must be examined.  We list a tentative set of questions one can ask:
\begin{itemize} \tightlist
\item Making sure that the Self-Consistent Field (SCF) calculation of the potentials has converged (more below) - see \file{log1.dat}
\item Check that the Density of States and Fermi level are reasonable (\file{ldos01.dat}, \file{log1.dat})
\item Check that the calculations are converged with respect to SCF cutoff radius
\item Check that the calculations are converged with respect to FMS cutoff radius
\item Check that the calculations are converged with respect to RPATH cutoff radius
\end{itemize}


In a next step, one plots the calculated spectrum (e.g. the XANES spectrum in file \file{xmu.dat}) together with a measured spectrum from an experiment.  If the agreement is not satisfactory, one can check many factors in the calculation:

\begin{itemize} \tightlist
\item Is the model of the structure correct?
\item Does the calculation need more broadening applied?
\item Does the calculation require a shift of the Fermi level to shift more states into the unoccupied band?  ({\feff} often puts the Fermi energy too high by a small amount.)
\item Does a different choice of core hole treatment improve the calculation?
\item Is it important to add Debye-Waller factors to the calculation to account for thermal effects?
\item Will the ManyPole Self-Energy improve the calculation?  (Typically yes, if peaks slightly beyond the edge need to move to slightly higher energies and require more broadening.)
\item If the system contains f-states, did they converge properly in the SCF calculation?
\item If the system has strongly correlated electrons, will a Hubbard (GW+U) calculation improve things?  (Note: this functionality will become available in the next release of FEFF.)
\item Could the spherical potential approximation be problematic for this material?
\item ...
\end{itemize}


Most of these techniques are discussed in greater detail throughout this chapter.  The paper by Moreno and Jorissen refers to an older version of {\feff} but still provides a good case study of all the steps to check when doing a {\feff} study.



%%%%%%%%%%%%%%%%%%%%%%%%%%%%%%%%%%%%%%%%%%%%%%%%%%%%%%%%%%%%%%%%%%%%%%%%%%% RUNNING FEFF %%%%%%%%%%%%%
\section{Some notes on the JFEFF GUI}
The JFEFF GUI is a user-friendly way to use the {\feff} code.  It removes the need to work on the command line, which is difficult for many new users, especially on MS Windows computers.  There may still be times when you want to use a text editor (any will do) to edit the \file{feff.inp} master input file by hand.  You may also want to have a plotting program to create images for reports or papers.  \texttt{Gnuplot} is a plotting program that exists for all platforms in one way or another; it's old-school but very flexible.

We work hard to make sure the Java GUI is up to date with the Fortran {\feffcur} code.  However, there can be inevitable lags and the latest developments may not yet be available through the GUI.

In the current version (10.0.0) we are aware of the following limitations :
\begin{itemize}  \tightlist
\item  JFEFF cannot open files containing the LATTICE or COORDINATES cards.  JFEFF can only treat periodic systems using the CIF option.
\item  JFEFF cannot do NRIXS calculations for more than one q-vector.
\item  JFEFF does not currently treat the CONFIG card correctly.
\item  JFEFF is slightly limited in the EGRID card: it can use only 5 different grids (10 for the command-line version of {\feff}) and cannot use the 'user\_grid' gridtype.
\item  JFEFF does not execute the \module{dym2feffinp} program for DMDW calculations.  The user needs to take care of this step first if applicable.
\item  JFEFF can get confused if you modify the \file{feff.inp} file in the working directory outside of JFEFF (e.g. in a text editor, or by copying or moving files).  It is safer to quit or save and quit JFEFF first, then copy your files, then re-open JFEFF and load the new file - rather than copying files and hoping JFEFF somehow realizes changes were made.
\item  JFEFF currently uses only one occurrence of the SCREEN card (that would be the last one occurring in \file{feff.inp} ) whereas the {\feff} code can handle many.
\item  JFEFF does not display the content of a \file{cif} file.  You must look at the file yourself to figure out what position you'd like to put the core hole at, and then enter the corresponding number in the TARGET field.  JFEFF does display the structure in the Plotter window.
\item  JFEFF cannot combine CIF and POTENTIALS cards, whereas this is possible (and sometimes useful) in {\feff}.
\item If both RPATH and FMS cards are present in \file{feff.inp}, JFEFF's results may depend on the order they appear in.  It is generally not recommended to use both cards in any one calculation.
\end{itemize}

We will not remedy the ability to read the LATTICE card, as using the CIF option is the preferred strategy for the future.  If any of the other limitations are hindering you, please contact us to see if we can help you out in the short run.  We are dedicated to making JFEFF as useful and pleasant as possible.


%%%%%%%%%%%%%%%%%%%%%%%%%%%%%%%%%%%%%%%%%%%%%%%%%%%%%%%%%%%%%%%%%%%%%%%%%%% POTENTIAL %%%%%%%%%%%%%
\section{Calculating the potential and phase shifts.}
\label{sec:calcpots}
Potentials and phase shifts are calculated by executing the first modules of
{\feff} : \module{atomic}, \module{pot}, \module{screen}, \module{xsph}.  This
sequence corresponds to the first two fields in the CONTROL card, i.e.,
\begin{verbatim}
CONTROL 1 1 0 0 0 0 
\end{verbatim}
calculates potentials and phase shifts (and nothing more).  Similarly, it corresponds to the first two checkboxes in the Run panel of the JFEFF GUI.

Initially the free atom potentials of each atomic type are calculated
using a relativistic Dirac--Fock atom code, treating the atoms as if
they were isolated in space. Scattering potentials are
calculated by overlapping the free atom densities in the muffin tin
approximation (Mattheiss prescription), and then including the
Hedin--Lundqvist/Quinn self energy for excited states. 
Non-overlapping muffin-tin radii are determined automatically from
the calculated Norman radii. Automatic overlapping of muffin tin
spheres (see the AFOLP card) is done by default, since it typically
leads to better results than non-overlapping muffin-tin spheres.
{\feffcur} can also calculate self-consistent potentials by successively
calculating the electron density of states, electron density and Fermi
level at each stage within a small cluster and then iterating, using
the Mattheiss prescription for the initial iteration.  This behavior is activated by the SCF card in \file{feff.inp}.  
It is strongly recommended for calculation of near-edge properties.  In this case, the radius
of that small cluster is a parameter that must be converged for good results.  Extended spectra, such as EXAFS or EXELFS,
can typically be calculated without self-consistent potentials.

The SCF radius typically falls between 20 - 100 atoms, with 50 atoms being a reasonable first attempt; however, it must always be verified carefully if an accurate calculation is desired.

XAFS spectra are referenced to the threshold Fermi level. This
quantity is best determined with the self-consistent field procedure 
(typically to within a fraction of an eV), 
or (less accurate but faster) can be estimated from the electron
gas result at the mean interstitial density in the Mattheiss prescription. 
An absolute energy scale is obtained by an atomic
calculation of the total energy of the system with and without the
core-hole. Atomic configurations and core-hole lifetimes are built
in, and mean free paths are determined from the imaginary part of the
average interstitial potential, including self-energy and lifetime
contributions.

The potential calculations need as input only the atomic number of the
atoms, and, for the absorbing atom, the type of the core hole being
considered. To do the overlapping of the unique potentials, the
neighboring atoms must be identified, either by position (from a list
of the Cartesian coordinates of each atom) or by explicit overlapping
instructions using the OVERLAP card.

To save time the code calculates the overlapped atom potential for each
unique potential only once, using as a sample geometry the atom with
a given unique potential index that is closest to the absorbing atom. 
Thus it is essential that the neighborhood of each sample atom be appropriate.
One should give consideration to adding potentials for sufficiently different atoms
of the same atomic number.  {\feffcur} only calculates the spherical part of the potential
("muffin tin potential"), so if atoms have environments identical up to a unitary transformation
(e.g. a space group symmetry operation in a crystal), they can certainly be considered equivalent
in {\feff}.  In strongly anisotropic environments, the muffin tin approximation may lead
to inaccuracies in the potentials and density of states (DOS). 

The potentials are written to file if the PRINT card is set high enough:
\begin{verbatim}
PRINT 2 0 0 0 0 0 
\end{verbatim}
results in files \file{pot00.dat}, \file{pot01.dat} and so on.  These contain single and overlapped potentials
and densities and can be plotted, e.g. in gnuplot.

The progression of a self-consistent calculation can be checked in the \file{.scfconvergence-feff}.

Note that {\feff} has historically accumulated options for setting the
edge and core hole.  We highly recommend that users use only the EDGE card for
specifying the edge, and the COREHOLE card for selecting the core hole treatment.
The HOLE and NOHOLE cards are deprecated and supported for backward compatibility only.

How best to treat the core-hole in XAS is an interesting theoretical question. The default choice in FEFF has always been a fully screened core-hole (meaning that the hole is self-consistently screened during the SCF loop), which is consistent with the "final state rule" ("COREHOLE fsr"). But this approximation sometimes breaks down for the L-shell metals. Often a better approximation is to use RPA screening.  This screening approximation is similar to that in the Bethe-Salpeter equation (BSE) and is available in FEFF9 ("COREHOLE RPA"). However, missing is the exchange term in the particle-hole interaction, which tends to cancel the core-hole. Interestingly the cancellation is often strong for the L-shell materials, which is why "COREHOLE none" sometimes works better. Recently we've developed a BSE code to better study these phenomena. See also ``Final-state rule vs the Bethe-Salpeter equation  for deep-core x-ray absorption spectra," J.J. Rehr, J. A. Soininen, and E. L. Shirley, Physica Scripta {\bf T115}, 207 (2005); as well as a forthcoming paper, "BSE calculations of transition metal L-shell spectra, J. Vinson and J.J. Rehr".


FIX add paragraph about screening.


\module{xsph} writes its main output to \file{xsect.dat} : this file contains the matrix elements and the atomic background as a function of energy.
If you set the PRINT card to 1 or higher, the phases will be written to a file \file{phase.dat}, though this file is not
usable for plotting.  Setting the PRINT card even higher makes {\feff} calculate hole counts and write them to \file{log2.dat}.  Careful - this is quite slow.
\begin{verbatim}
PRINT 0 3 0 0 0 0  *hole counts - slow calculation
\end{verbatim}

Relativistic dipole matrix elements (alpha form) are calculated using
atomic core and normalized continuum wave functions. Polarization
dependence is optionally incorporated in the dipole-operator.
Scattering phase shifts are determined by matching at the muffin-tin
radius. 

{\feff} is designed to calculate absorption from completely filled
shells. You can try to simulate absorption from valence electrons
with {\feff}, but you may get unreliable results. If you
encounter difficulties and need valence shell absorption, please
contact the authors.


%%%%%%%%%%%%%%%%%%%%%%%%%%%%%%%%%%%%%%%%%%%%%%%%%%%%%%%%%%%%%%%%%%%%%%%%%%% DOS %%%%%%%%%%%%%

\section{Calculating the Density of States}
Once the potentials and phase shifts are known, one can calculate the Density of States.  This requires the LDOS card.  If you are not
running all of \texttt{feff}, you can simply run the \module{ldos} module.  It will write files \file{ldos00.dat}, \file{ldos01.dat} and so on, giving
the $l-$resolved DOS for each potential type.  There is currently no way to obtain $m-$resolved DOS.

The \module{ldos} module uses a fixed energy mesh of 81 points which sometimes frustrates users wanting the DOS over a larger energy range.
The simplest solution is to simply run \module{ldos} several times with a different energy range and combine the output.
\begin{verbatim}
LDOS -20.  20.  0.05  * first run, -20->20 eV
*LDOS 20.  60.  0.05  * second run, 20->60eV
\end{verbatim}


%%%%%%%%%%%%%%%%%%%%%%%%%%%%%%%%%%%%%%%%%%%%%%%%%%%%%%%%%%%%%%%%%%%%%%%%%%% MULTIPLE SCATTERING %%%%%%%%%%%%%
\section{Calculating the Multiple Scattering Green's function.}


Once we have the potentials and phase shifts, we can calculate the multiple scattering (MS) Green's function.  This corresponds to the sequence of modules
\module{fms}, \module{mkgtr}, \module{path} and \module{genfmt}.  Equivalently, fields 3-5 in the CONTROL card, or checkboxes 3-5 in the Run panel of the JFEFF GUI:
\begin{verbatim}
CONTROL 0 0 1 1 1 0 * calculate G and nothing else
\end{verbatim}


There are two ways to calculate the Green's function.  The first method is suitable for near-edge properties and is executed by modules \module{fms} and \module{mkgtr}.
These modules carry out a full multiple scattering XANES/ELNES calculation.  
In real space calculations this is done for a cluster centered on the absorbing atom. In k-space it is done
for a matrix containing only the atoms of the unit cell.  Thus all multiple-scattering paths within this system are summed to infinite
order. This is useful for XANES and ELNES calculations, but usually cannot be
used for EXAFS analysis. FMS loses accuracy beyond $k =
(l_{\mathrm{max}}+1)/r_{\mathrm{mt}}$, which is typically about 4
\AA$^{-1}$ since the muffin-tin radius $r_{\mathrm{mt}}$ is
typically about 1 \AA.  In real-space calculations, the FMS cluster radius must be converged to ensure adequate accuracy.
FMS is typically the most time-consuming part of the calculation and slows down significantly with cluster or unit cell size.

The FMS Green's function is written to file \file{gg.dat}.  Its trace - which yields the spectrum - can be found in \file{gtr.dat}.

For energies high above the Fermi level or edge threshold, the Path Expansion (PE) is more appropriate.  It is executed by modules \module{path} and \module{genfmt}.  Here, instead of summing implicitly over an infinite number of paths, we sum explicitly over a select range of scattering paths.  The code uses a constructive algorithm with several path importance
filters to explore all significant multiple-scattering paths in order
of increasing path length. The paths are determined from the list of
atomic coordinates in \file{feff.inp}. An efficient degeneracy
checker is used to identify equivalent paths (based on similar
geometry, path reversal symmetry, and space inversion symmetry). To
avoid roundoff errors, the degeneracy checker is conservative and
occasionally treats two degenerate paths as not degenerate.
These errors occur in the third or fourth decimal place (less than
0.001 Ang) but are fail-safe; that is, no paths will be lost. All 
paths which are completely inside the FMS cluster are automatically 
excluded from the paths list, if specified by the \htmlref{FMS}{card:fms} 
card.

The criteria used in filtering are based on increasingly accurate
estimates of each path's amplitude. The earliest filters, the
pathfinder heap and keep filters, are applied as the paths are being
searched for. A plane wave filter, based on the plane wave approximation
(plus a curved wave correction for multiple-scattering paths) and
accurate to about 30\%, is applied after the paths have been enumerated
and sorted. Finally, an accurate curved wave filter is applied to
all remaining paths.

In the event of a k-space calculation, the \module{rdinp} module generates a
large real-space list of atom coordinates from the unit cell information in \file{feff.inp}.
This list will be used for the \module{path} and \module{genfmt} modules, which always work
in real space, as this is the optimal space for treating the extended spectrum.

The list of all paths can be found in file \file{paths.dat}.  You can manually edit this list.

For each path the code calculates the effective scattering amplitude
 and the total scattering phase
shift along with other XAFS parameters using the scattering matrix
algorithm of Rehr and Albers. This requires only the scattering phase shifts (module \module{xsph}) and
the paths (module \module{path}) as input.


It is possible to combine the two strategies.  E.g. the following input
\begin{verbatim}
FMS 4.0
RPATH 10.0
XANES
\end{verbatim}
tells {\feff} to calculate a XANES spectrum by doing Full Multiple Scattering up to a radius of 4. \AA around the aborber (summing all paths to infinite order within this radius), and adding Path Expansion contributions from paths extending beyond this radius but no longer than 10. \AA .  Although this is possible, we do not recommend it.  We recommend the combination FMS + XANES for near-edge spectra, and EXAFS + RPATH for extended spectra.



%%%%%%%%%%%%%%%%%%%%%%%%%%%%%%%%%%%%%%%%%%%%%%%%%%%%%%%%%%%%%%%%%%%%%%%%%%% SPECTRUM %%%%%%%%%%%%%
\section{Calculating the spectrum}
Finally, we can calculate the spectrum from the Green's function using the sequence \module{ff2x}, \module{sfconv}, \module{compton} and \module{eels}, corresponding to the last field of the CONTROL card:
\begin{verbatim}
CONTROL 0 0 0 0 0 1 * calculate the spectrum from G
\end{verbatim}

The modules \module{ff2x}, \module{sfconv}, \module{compton} and \module{eels} construct the XAS spectrum $\chi(k)$ or $\mu$ using the
XAFS parameters
\begin{latexonly}
  described in Section~\ref{sec:Vari-EXAFS-form} 
\end{latexonly}
from one or more paths, and adding the FMS contributions.
Single and multiple scattering Debye--Waller
factors are calculated using, for example, the correlated Debye model.
The spectrum can be found in file \file{xmu.dat} for most types of spectroscopy.  Numerous
options for filtering, Debye--Waller factors, and other corrections are
available.  The details depend on the type of spectroscopy.  For more details and examples, we refer to the section below that is appropriate for your type of calculation.

FIX Add a paragraph or two about sfconv either here or in a section of itself

%%%%%%%%%%%%%%%%%%%%%%%%%%%%%%%%%%%%%%%%%%%%%%%%%%%%%%%%%%%%%%%%%%%%%%%%%%% EXAFS %%%%%%%%%%%%%

\section{EXAFS Calculation}
\label{sec:EXAFS-calculation}
Self-consistent or overlapped atom potentials are necessary
for the calculation of the scattering phase shifts. Self-consistent
calculations (using the SCF card) take more time. 
Although the effect of self-consistency on EXAFS is small,
such calculations give an accurate determination of $E_0$,
thus eliminating an important parameter in EXAFS
distance determinations.

The calculation of potentials and phase shifts can be relatively slow, so it is usually best to run
it only once and use the results while studying the paths and XAFS.

To enumerate the necessary paths, the pathfinder module \module{path} needs the
atomic positions of any atoms from which scattering is expected. If
the structure is completely unknown, only single-scattering paths can be
created explicitly. Because the number of possible paths increases
exponentially with total path length, one should start with a short total path
length, examine the few paths (representing scattering from the nearest
neighbors), and gradually increase the total path length, possibly
studying the path importance coefficients and using the filters to
limit the number of paths. This process is not automated, and if done
carelessly can yield so many paths that no analysis will be possible.

Finally, use \module{genfmt} to calculate the XAFS parameters, and \module{ff2x} to
assemble the results into a chi curve. Here, the slow part is \module{genfmt}
and \module{ff2x} is very fast. Therefore, to explore parameters such as
Debye--Waller factors, mean free path and energy zero shifts, various
combinations of paths and coordination numbers, run only module \module{ff2x}
using the results saved from \module{genfmt}.

If your model changes significantly, the phase shifts (which are based
in part on the structure of the material) should be recalculated.
Any time the phase shifts change, the XAFS parameters will also have to be
re-calculated. If the path filters have been used, the path list will
also have to be recomputed.



\subsection{SF$_6$ Molecule}
\label{sec:Molecule}
SF$_6$ Molecule. This is the simplest example of running {\feff} to obtain
results for EXAFS. Only two input cards, POTENTIALS and ATOMS, are strictly necessary. However it is good practice to list the cards EXAFS and RPATH.  The output $\chi$ can be found in the file \file{chi.dat}.

\begin{verbatim}
  TITLE Molecular SF6
  
  POTENTIALS
  * absorbing atom must be unique pot 0
  *    ipot    z   tag
         0    16   S        
         1     9   F
  
  EXAFS 20
  RPATH 10
  EDGE K
  
  ATOMS
  *  x      y      z     ipot
     0      0      0       0          S absorber
     1.56   0      0       1          6 F backscatters
     0      1.56   0       1
     0      0      1.56    1
    -1.56   0      0       1
     0     -1.56   0       1
     0      0     -1.56    1
\end{verbatim}




\subsection{Solids}
\label{sec:Solid}

\subsubsection{Cu metal}
\label{sec:Cu-metal}

Cu, fcc metal, 4 shells. The list of atomic coordinates
(\htmlref{ATOMS}{card:ato} card) for crystals can be produced by the program
{\atoms}. Thus instead of giving a long atoms list, we present a short
\file{atoms.inp} file.  For connection with EXAFS fitting programs see
Section~\ref{sec:Input-and-Output-Files} and the \htmlref{PRINT}{card:pri}
card on page \pageref{card:pri}.

\begin{verbatim}
  TITLE Cu crystal, 4 shells
  * Cu is fcc, lattice parameter a=3.61 (Kittel)
  
  *Cu at 190K, Debye temp 315K (Ashcroft & Mermin)
  DEBYE  190  315 0
  EDGE K
  
  POTENTIALS
    0  29  Cu0
    1  29  Cu
  
  ATOMS
  atoms list generated using atoms.inp file below
  --------------------------------------------
  title Cu  metal  fcc a=3.6032
  fcc                   ! shorthand for F M 3 M
  rmax= 11.13   a=3.6032
  out=feff.inp          ! index=true
  geom = true
  atom
  !   At.type    x    y    z
  Cu       0.0  0.0  0.0
  --------------------------------------------
\end{verbatim}



\subsubsection{YBCO High-Tc superconductor}
\label{sec:YBCO-High-Tc}


\begin{verbatim}
  TITLE  YBCO: Y Ba2 Cu3 O7      Cu2 core hole
  EDGE K
  CONTROL  1  1  1  1  1  1
  PRINT    0  0  0  0  0  0
  
  RPATH   4.5
  
  POTENTIALS
  *    ipot  z  tag
        0   29  Cu2
        1    8  O
        2   39  Y
        3   29  Cu1
        4   56  Ba
  
  ATOMS
  atoms list generated by the following atoms.inp file
  -----------------------------------
  title YBCO: Y Ba2 Cu3 O7  (1-2-3 structure)
  space P M M M
  rmax=5.2         a=3.823  b=3.886  c=11.681
  core = Cu1
  atom
  ! At.type  x     y     z       tag
    Y       0.5   0.5   0.5
    Ba      0.5   0.5   0.184
    Cu      0     0     0        Cu1
    Cu      0     0     0.356    Cu2
    O       0     0.5   0        O1
    O       0     0     0.158    O2
    O       0     0.5   0.379    O3
    O       0.5   0     0.377    O4
  --------------------------------------
\end{verbatim}





\subsection{Estimate of $S_0^2$}
\label{sec:S02-estimate}
All above examples yield calculations for the K edge (default). To do
calculations for other edges, use the \htmlref{EDGE}{card:edg} or 
\htmlref{HOLE}{card:hol} cards. These cards
will also yield an estimate of $S_0^2$ from atomic calculations if you
set $S_0^2<0.1$ by the one of two possible ways shown below.
\begin{verbatim}
  EDGE   L3    0.0
  S02     0.0
\end{verbatim}

The result for $S_0^2$ is given in \file{chi.dat} or \file{xmu.dat}.
$S_0^2$ is a square of determinant of overlap integrals for core
orbitals calculated with and without core hole. The core-valence
separation can be changed by editing the subroutine \texttt{getorb}, but
it is currently set by default to the most chemically reasonable one.


\subsection{Configuration Averaging Over Absorbers}
\label{sec:Aver-over-absorb}
In amorphous materials or materials with distortions from regular
crystals, the absorbing atoms (with the same number in the periodic table)
may have different surroundings. Thus one may want to average the
calculation over different types of sites for the same atom or even
over all atoms in the \file{feff.inp} file. This can be accomplished using
\htmlref{CFAVERAGE}{card:cfa} card of Section~\ref{sec:Structural-Information-Cards}.
This type of calculation is currently curtailed by 
the limited functionality of the CFAVERAGE card, which should be used 
with caution.  Please contact the authors if problems occur.

FIX we need an example here


\subsection{Adding Self-consistency}
\label{sec:Adding-self-cons}
Self-consistency is expected to be more important for XANES
calculations, but even for EXAFS one may want to have a more reliable
determination of Fermi level or to account for charge transfers in
order to do fits with a single energy shift $E_0$. Our experience shows
that reliable EXAFS phase shifts are best achieved using the 
\htmlref{SCF}{card:scf} card.

\begin{verbatim}
  *calculate EXAFS with SCF potentials and paths to R=6 angstroms
  EXAFS
  SCF   3.8
  RPATH 6.0
\end{verbatim}

The above example works for solids or large molecules, but for molecules with 
less than 30 atoms, calculations can be done faster if you set $\mathtt{lfms1}=1$:
\begin{verbatim}
  SCF   10.0  1
\end{verbatim}

For details see the \htmlref{SCF}{card:scf} and \htmlref{FMS}{card:fms} cards in
Sections~\ref{sec:Scatt-potent-modul} and \ref{sec:Full-mult-scatt}.

%%%%%%%%%%%%%%%%%%%%%%%%%%%%%%%%%%%%%%%%%%%%%%%%%%%%%%%%%%%%%%%%%%%%%%%%%%% XANES %%%%%%%%%%%%%

\section{XANES Calculations}
\label{sec:XANES-calculations}

\subsection{Need for SCF and Additional Difficulties for XANES}
\label{sec:Addit-diff}
XANES calculations usually take more time and require more experience than EXAFS calculations.
They require self-consistent potentials using the \htmlref{SCF}{card:scf} card. The use of the SCF card 
also gives a more reliable estimate of the Fermi level.  (The CORRECTIONS card
can still be used, since the error in Fermi level position is only a few eV).
{\feffcur} thus automatically accounts for charge transfer. The ION card should 
be used only to specify the total charge of a cluster. Overlapping of muffin tins (AFOLP) leads to better results for XANES and is done by default.

The high order MS path expansion can lead to unreliable XANES calculations when the MS series converges
poorly (as is often the case near the Fermi level). Thus using Full Multiple Scattering (FMS) instead of the path expansion is essential for calculations of $\ell$DOS and
electronic densities and is usually an improvement for XANES. We suggest to use FMS exclusively and uses path expansion for testing its convergence.  The FMS calculations
typically take more time and memory than the other 5 modules. The
results can be somewhat better with larger clusters, but typically
one achieves convergence with about 100-300 atoms.  Calculation
time scales as a third power of the number of atoms in a cluster and
quickly becomes expensive.  However, the description of the FMS card lists options for iterative matrix solvers that improve
the scaling somewhat.
Here are sample input files for XANES calculation.


\subsection{GeCl$_4$ Molecule}
\label{sec:Molecule-1}

This is historically the first molecule for which EXAFS was calculated using short range order
theory. (Hartree, Kronig and Peterson, 1934)

\begin{verbatim}
  TITLE   GeCl_4  r=2.09 /AA
  
  COREHOLE none
  EDGE K   1.0
  RSIGMA
  
  CONTROL   1  1  1  1  1  1
  
  SCF    3.0  1
  FMS    3.0  1
  RPATH  1.0
  XANES  8.0 0.05
  
  AFOLP  1.30
  
  POTENTIALS
  *   ipot   z  label
        0   32   Ge  3 3
        1   17   Cl  3 3
  ATOMS
  *     x          y          z       ipot atom        
     0.0000     0.0000     0.0000     0   Ge
     1.2100     1.2100     1.2100     1   Cl
     1.2100    -1.2100    -1.2100     1   Cl
    -1.2100     1.2100    -1.2100     1   Cl
    -1.2100    -1.2100     1.2100     1   Cl
  END
\end{verbatim}


\subsection{Solid: XANES and $\ell$DOS}
\label{sec:Solid-1}
BN crystal has a zinc sulfide structure.  The
multiple scattering path expansion does not converge near the Fermi level.
Using the full multiple scattering approach leads to good agreement
with experiment.

\begin{verbatim}
  TITLE   BN cubic zinc sulfide structure
  CONTROL   1 1 1 1 1 1
  PRINT     5 0 0 0 0 0 
 
  SCF  3.1
  EDGE K   1.0     *  s0^2=1.0
  EXCHANGE  0  0  1.0
  LDOS -20  10  0.5
  FMS  5.1
  RPATH   1.0
  XANES 4.0
 
  INTERSTITIAL 0  1.54
 
  POTENTIALS
 *   ipot   z  label lmax1  lmax2
        0    5   B     2    2  0.1
        1    7   N     2    2  1
        2    5   B     2    2  1
 
  ATOMS
  list generated by ATOMS program
  -------------------------------------
  title BN (zincblende structure)
  Space  zns
  a=3.615 rmax=8.0   core=B
  atom
  ! At.type  x        y       z      tag
     B      0.0      0.0     0.0
     N      0.25     0.25    0.25
  ----------------------------------------
\end{verbatim}

%%%%%%%%%%%%%%%%%%%%%%%%%%%%%%%%%%%%%%%%%%%%%%%%%%%%%%%%%%%%%%%%%%%%%%%%%%% ABSOLUTE %%%%%%%%%%%%%

\subsection{Cross-section in absolute units}
\label{sec:Absol-cross-sect}
The absolute cross section for XAS spectroscopy can be obtained from the output in
\file{xmu.dat}. Look for this line:

\begin{verbatim}
  xsedge+100, used to normalize mu           2.5908E-04
\end{verbatim}

Since our distances are in \AA, we report the XAS cross section also in \AA$^2$.
If you multiply the 4-th or 5-th column by this normalization value
you will obtain the cross section in \AA$^2$. Literature often reports the absolute
cross section in barns (1 \AA$^2 = 100 \mathrm{Mbarn}$).

Note that this normalization can be switched off using the ABSOLUTE card.  This is set
by default for NRIXS, ELNES and EXELFS calculation.  The normalization factor then shows as "`1.0"'. 

As of {\feff}9.1, EELS spectra in \file{eels.dat} are always given in units of $a_0^2 / eV$.  Multiply by $28.00 10^-18$  to get units of $cm^-2 / eV$.  Or multiply by $28$ to get units of  $Mbarn / eV$.


%%%%%%%%%%%%%%%%%%%%%%%%%%%%%%%%%%%%%%%%%%%%%%%%%%%%%%%%%%%%%%%%%%%%%%%%%%% XMCD/SPXAFS %%%%%%%%%%%%%

\section{Spin-dependent Calculations}
\label{sec:Spin-depend-calc}

\subsection{General Description}
\label{sec:General-description}
This section contains information on extracting the \htmlref{XMCD}{card:xnc} 
signal and on the SPXAS technique, including example input 
files.

Spin-dependent calculations are automated. All 
spin-dependent calculations require that the SPIN card be present in 
\file{feff.inp} (see the \htmlref{SPIN}{card:spi} card in 
Section~\ref{sec:Scatt-potent-modul}). 
The method depends on the value of the 
parameter \texttt{nspx} in the header file \file{feff90/src/COMMON/m\_dimsmod.f90}.
Please contact the authors if you need help modifying the source code.

In order for the final result to be contained in \file{xmu.dat}, 
{\feff} must be compiled with $\mathtt{nspx}=2$ to combine both the 
spin-up and spin-down calculations. This will also add the contribution from 
spin-flip processes (which we find typically very small), but may require up 
to 4 times the memory and 8 times the execution time for the XANES region. 

With $\mathtt{nspx}=1$, one can run the code twice, once for spin-up 
and once for spin-down.  Spin-flip terms are not calculated, but by making the proper combination of spin-up and spin-down spectrum, one obtains \htmlref{XMCD}{card:xnc} or SPXAS (Section~\ref{sec:SPXAS}).  A simple program \file{spin.f}, printed in the  \htmlref{appendix}{sec:Append-F-Spinf} and also on the {\feff} 
(\htmladdnormallink{website}{http://feff.phys.washington.edu/feff/}), can do this.  The same paths should be used for spin-up and spin-down calculations, 
otherwise the difference between 2 calculations may be due to 
different paths used. Typically the paths list in \file{paths.dat} 
should be generated by running the usual EXAFS calculations and 
comparing with experiment (to make sure that all important paths 
are included). Then, when running with SPIN, turn off the pathfinder 
module using the \htmlref{CONTROL}{card:con} card.


\subsection{XMCD for the Gd L1 edge}
\label{sec:XMCD-ex}
Example input file for calculation of XMCD for the Gd L1 edge. 
Note the presence of both the \htmlref{SPIN}{card:spi} and 
\htmlref{XMCD}{card:xnc} cards.

\begin{verbatim}
  TITLE   Gd  l1  hcp 
  
  XMCD
  EDGE L1   1.0    *  s0^2=1.0
  SPIN    1
  EXCHANGE  2
  RGRID .01
 
  CONTROL   1      1     1     1    1    1
  RPATH    7.29
  CRITERIA   0.0  0.0  curved   plane
  DEBYE   150    176   temp     debye-temp
  XANES
 
  POTENTIALS
  *   ipot   z  label
        0   64   Gd
        1   64   Gd
 
  ATOMS
  the list of atoms is created by ATOMS program
  ---------------------------------------------
  title    Gd , hcp
  ! Wycoff, vol.1 p.331
  space  hcp
  rmax = 9.0
  a = 3.6354
  c = 5.7817
  atom
    Gd   0.33333   0.66667   0.25   center
  -----------------------------------
\end{verbatim}



\subsection{SPXAS}
\label{sec:SPXAS}
For antiferromagnets, the XMCD should be zero, and a measure of the 
spin-up and spin-down signals can be accomplished using SPXAS. SPXAS 
is a technique where you measure the spin-up and spin-down signal by 
measuring the intensity of two spin-split K$_{\beta}$ lines. This 
corresponds to measuring spin-order relative to the spin on the absorber 
(not relative to the external magnetic field as in XMCD). As an example, 
let us look at the Mn K edge of antiferromagnetic MnF$_2$. Our calculations 
agree well with experiment in the EXAFS region.

Here is an example input file for MnF$_2$:

\begin{verbatim}
  TITLE   MnF2 (rutile) cassiterite (Wykoff)
  
  EDGE K   1.0    *  s0^2=1.0
  SPIN   2
 
  CONTROL   1      1     1     1     1     1
  RPATH  10.0
  XANES
  PCRITERIA       0.8     40.0
  CRITERIA       0.0     0.0  
  DEBYE        300       350
  NLEG         4

  POTENTIALS
  *   ipot   z  label
        0   25   Mnup
        1    9   F
        2   25   Mnup
        3   25   Mndown
 
  ATOMS
    0.0000     0.0000     0.0000    0   Mnup             0.0000
    1.4864     1.4864     0.0000    1   F                2.1021
   -1.4864    -1.4864     0.0000    1   F                2.1021
    0.9503    -0.9503     1.6550    1   F                2.1319
    0.9503    -0.9503    -1.6550    1   F                2.1319
   -0.9503     0.9503     1.6550    1   F                2.1319
   -0.9503     0.9503    -1.6550    1   F                2.1319
    0.0000     0.0000    -3.3099    3   Mndown           3.3099
    0.0000     0.0000     3.3099    3   Mndown           3.3099
   -3.3870     1.4864     0.0000    1   F                3.6988
    3.3870    -1.4864     0.0000    1   F                3.6988
    1.4864    -3.3870     0.0000    1   F                3.6988
   -1.4864     3.3870     0.0000    1   F                3.6988
    2.4367     2.4367    -1.6550    2   Mnup             3.8228
   -2.4367    -2.4367    -1.6550    2   Mnup             3.8228
    2.4367    -2.4367    -1.6550    3   Mndown           3.8228
   -2.4367    -2.4367     1.6550    3   Mndown           3.8228
   -2.4367     2.4367    -1.6550    3   Mndown           3.8228
    2.4367     2.4367     1.6550    3   Mndown           3.8228
    2.4367    -2.4367     1.6550    2   Mnup             3.8228
   -2.4367     2.4367     1.6550    2   Mnup             3.8228
  ...
  END
\end{verbatim}


%%%%%%%%%%%%%%%%%%%%%%%%%%%%%%%%%%%%%%%%%%%%%%%%%%%%%%%%%%%%%%%%%%%%%%%%%%% DANES %%%%%%%%%%%%%

\section{Elastic Scattering Amplitudes}
\label{sec:DANES}
All necessary components to obtain the elastic scattering amplitude
can be calculated. The Thomson scattering
amplitudes are written in the file \file{fpf0.dat}. The elastic amplitudes
near a specific edge are calculated with the \htmlref{DANES}{card:dan} 
card, while those far from the edge are calculated with the 
\htmlref{FPRIME}{card:fpr} card, which neglects solid state effects on $f'$.
$f''$ can be obtained with the \htmlref{XANES}{card:xan} card. 
The formula connecting $f''$ and the absorption cross section $\sigma $ 
is (in atomic units) $f'' = \omega c \sigma /4/\pi $. For calculations 
at energies well above the absorption edge, we found that ground state 
potentials yield better results and that quadrupolar transitions 
have to be included.

An example can be found in the \url{jfeff_examples} directory of your {\feff} installation.


%%%%%%%%%%%%%%%%%%%%%%%%%%%%%%%%%%%%%%%%%%%%%%%%%%%%%%%%%%%%%%%%%%%%%%%%%%% XES %%%%%%%%%%%%%

\section{X-ray Emission Spectra XES}
\label{sec:XES}
Nonresonant x-ray emission spectra (fluorescence spectra) 
are treated in the same way as the x-ray absorption process
for states below the Fermi level. To perform these calculations one simply
replaces the \htmlref{XANES}{card:xan} card with \htmlref{XES}{card:xes}. 
Preliminary comparisons with experiment
for the phosphorous K$_{\beta}$ line show good agreement with experiment for various
compounds. Please report any problems with this card to the authors.

An example can be found in the \url{jfeff_examples} directory of your {\feff} installation.


%%%%%%%%%%%%%%%%%%%%%%%%%%%%%%%%%%%%%%%%%%%%%%%%%%%%%%%%%%%%%%%%%%%%%%%%%%% EELS %%%%%%%%%%%%%

\section{Calculation of EELS}
\label{sec:EELS}
Electron-energy loss spectroscopy (EELS) measures the spectrum of 
energy losses of a beam of high-energy electrons passing through a 
sample in an electron microscope. We include relativistic effects in the cross 
section.  {\feff} is optimized for EELS calculations and includes various instrumental parameters

To calculate EELS, add either the \htmlref{ELNES}{card:eln} card or the \htmlref{EXELFS}{card:exe} card to the \file{feff.inp} input file.  For the near-edge region, full multiple scattering is appropriate:
\begin{verbatim}
FMS 7.0
ELNES ...
\end{verbatim}
While the extended region requires the path expansion:
\begin{verbatim}
RPATH 8.0
EXELFS ...
\end{verbatim}

The EELS spectrum is written to the \file{eels.dat} file in absolute units.  See the file header for more details.  Old versions of {\feff} used arbitrary units and may have a wildly different order of magnitude.

  The EELS engine calculates the $\Sigma$ tensor, containing 'basis spectra' for 
all 9 polarization components.  EELS calculations cannot be combined with the \htmlref{POLARIZATION}{card:pol} or \htmlref{ELLIPTICITY}{card:ell} cards.  The
corresponding variables (the polarization vector resp. the beam direction) are set internally based on the options of the \htmlref{ELNES}{card:eln} card or the \htmlref{EXELFS}{card:exe} card.



\subsection{Changing EELS parameters}
\label{sec:EELS-changing}
If you change something in the calculation of the material properties - such as atom 
positions, or the FMS radius - you need to rerun all or most of {\feff}.
If you only change the experimental setup  (i.e., beam direction, beam energy, detector aperture or position, or beam convergence angle ; or toggle relativistic cross-section on/off ) but leave the sample unchanged, then you need only do the following :

\begin{itemize} \tightlist
  \item Edit \file{feff.inp}
  \item Run module \module{rdinp} to update \file{eels.inp} (alternatively, you can edit 
    \file{eels.inp} directly--but beware, this file is somewhat format-sensitive)
  \item Run module \module{eels} to calculate a new EELS spectrum.
\end{itemize}

Alternatively, in the JFEFF GUI,
\begin{itemize} \tightlist
\item Click the "Other Options" in the spectrum panel and make the desired changes; 
\item Unselect all checkboxes except \file{cross section} in the Run panel of the main window;
\item Click \file{Save \& Run}.
\end{itemize}

 The \module{eels} module is very fast and can be looped over to fit some of the experimental parameters mentioned.

\begin{figure}[H]
	\centering
		\includegraphics[height=3.0in]{eels1.png}
		\caption{Pressing the \file{Other Options} button in the Spectrum panel pops up a dialog for the EELS-specific parameters.  In this case, the user is editing the collection angle to better reflect her experimental setup.  Notice that in the Run panel only the \file{cross section} calculation is selected.  It is not necessary to re-run the rest of the calculation when only the microscope parameters are changed.  In this case the new calculation will take less than a second.}
	\label{fig:eels1}
\end{figure}


\subsection{Example input.}
See the \htmlref{example file} {sec:k-space} in the k-space chapter.


%%%%%%%%%%%%%%%%%%%%%%%%%%%%%%%%%%%%%%%%%%%%%%%%%%%%%%%%%%%%%%%%%%%%%%%%%%% NRIXS %%%%%%%%%%%%%

\section{NRIXS}
\label{sec:nrixs}

Non-Resonant Inelastic X-ray Scattering (NRIXS) is determined by the momentum transfer dependent
dynamic structure factor $S(\vec{q},E)$ (DFF).  {\feff} NRIXS calculations output the DFF for a given $\vec{q}$ over a range of $E$.
From this quantity, the NRIXS signal is obtained by simply adding a multiplicative factor containing beam energies and polarization.  (See references.)
NRIXS calculations are controlled by the following cards :  NRIXS, LJMAX, and LDEC.  In addition, one must use either the XANES 
and FMS cards for the near-edge region ; or the EXAFS and RPATH cards for the extended region.
Note that for NRIXS, if $xkmax < 0$ in the XANES or EXAFS card, then an energy grid of approximately constant energy
step is used (instead of the usual constant k-step).

Although the code has been tested for wide range of momentum transfers the code is most
stable for the medium region of momentum transfer values. This typically means $0.1 < q <
14 a.u.$ for light elements or shallow edges, i.e. binding energy less than 400 eV. Higher
momentum transfers are possible for more tightly bound electrons (see examples below). The
small momentum transfers can cause numerical instabilities at near edge region and large
momentum transfers also in the extendend energy range. There is no default value for the
momentum transfer.

NRIXS produces two main output files.  \file{xmu.dat} contains the total spectrum in the usual 6-column format.  \file{xmul.dat} contains:
\begin{itemize}
\item col 1:  the excitation energy (in eV).
\item col 2:  the value of $k$ (in Angstrom) at this energy.
\item col 3:  the atomic background value $S_0(\vec{q},E)$
\item col 4 - 4+ld:  the next $ld+1$ columns give the contribution to the atomic background $S_l^0(\vec{q},E)$ for the final
state electron angular momentum values l = 0, . . . , ld :
\begin{equation} \label{nrixseq1}
S_l^0(\vec{q},E) = (2l + 1) |M_l(\vec{q},E)|^2 \rho_{l}^0 (E).
\end{equation}
\item col 4+ld+1 - end:  the next $(ld+1)  (ld+1)$ columns give the decomposition of fine structure i.e.
\begin{equation} \label{nrixseq2}
\chi_{\vec{q}}^{ll'}(k) = \frac {1} {S_0(\vec{q},E)} \sum_{mm'} {M_L(\vec{-q},E)\rho_{LL'}^{sc}(E)M_{L'}(\vec{q},E)}
\end{equation}
for $l \leq ld$ and $l' \leq ld$.
\end{itemize}

To plot the contribution of only s-type $(l = 0)$ final states one would do in gnuplot $(ld = 2)$ :
\begin{verbatim}
> plot xmul.dat u 1:($4+$3*$7)
\end{verbatim}
and for only p-type:
\begin{verbatim}
> plot xmul.dat u 1:($5+$3*$11).
\end{verbatim}
To get the s-p non-diagonal contribution to $S_0(\vec{q},E) \chi_{\vec{q}}(k)$:
\begin{verbatim}
  > plot xmul.dat u 1:($3*($10+$8)). 
\end{verbatim}
%$

\subsection{Some example input}
Here is the standard FEFF example for the GeCl4 molecule.
\begin{verbatim}
TITLE GeCl_4 r=2.09 /AA
COREHOLE none
EDGE K  1.0
RSIGMA
CONTROL 1 1 1 1 1 1
SCF 3.0 1
FMS 3.0 1
RPATH 1.0
XANES 8.0 0.05
AFOLP 1.30
POTENTIALS
* ipot z label
0 32 Ge 3 3
1 17 Cl 3 3
ATOMS
* x    y    z    ipot    atom
 0.0000 0.0000 0.0000 0 Ge
 1.2100 1.2100 1.2100 1 Cl 
 1.2100 -1.2100 -1.2100 1 Cl
-1.2100 1.2100 -1.2100 1 Cl
-1.2100 -1.2100 1.2100 1 Cl

* Choose one of these two options :
* 1/ calculation with 24 a.u. momentum transfer along the z-axis
NRIXS 1 0.0 0.0 24.0
LJMAX 10
* 2/ Since this is a molecule (gas phase or liquid) we really should average 
*  over the direction
NRIXS -1 24.0
LJMAX 10
LDEC 2

END
\end{verbatim}

\subsection{Limitations and practicalities.}
The NRIXS code requires substantial stack space.  We recommend changing the available stacksize to unlimited using shell commands  (in tcsh)
\begin{verbatim} > limit stacksize unlimited \end{verbatim}
or (in bash)
\begin{verbatim} > ulimit -s unlimited.\end{verbatim}

The following input cards cannot currently be used for NRIXS:
  CFAVERAGE, SPIN, ELLIPTICITY, MULTIPOLE, POLARIZATION, TDLDA, RPHASES, XES, XMCD.

The NRIXS code cannot run in k-space.

The NRIXS code can simultaneously calculate a list of q-vectors if run from the command line.  However, this functionality is not yet implemented in the JFEFF GUI,
which remains limited to a single q-vector at this time.

%%%%%%%%%%%%%%%%%%%%%%%%%%%%%%%%%%%%%%%%%%%%%%%%%%%%%%%%%%%%%%%%%%%%%%%%%%% PMBSE/TDDFT %%%%%%%%%%%%%

\section{Local Field Effects and Core-hole Effects (PMBSE, TDDFT)}
\label{sec:Loc-field-core-hole}

These are features of {\feff} that are still under development. 
They are not reliable in this release of {\feffcur}.

Many-body effects such as local fields and the core-hole interaction 
can be significant in x-ray absorption spectra, even several hundred 
eV above an absorption edge. The treatment of these effects requires 
theories beyond the independent-particle approximation, e.g., the Bethe-Salpeter
equation (BSE) or the time-dependent density-functional theory (TDDFT).

The projection-operator method Bethe-Salpeter equation (PMBSE) is used 
for core-hole absorption spectra calculations. The BSE is usually limited 
to low energies, while the TDDFT often ignores the nonlocality of the 
core-hole interaction. Time dependent density functional theory (TDDFT) 
is a general framework for studying non-stationary electronic processes. 
TDDFT is used for local-field absorption spectra calculations. 

The approach being developed for {\feff} is a combined approach for 
calculations of the x-ray spectra that include both of these effects, 
together with inelastic losses and self-energy shifts over a wide 
energy range. 

Note that the \htmlref{TDLDA}{card:tdl} card accounts for some of the same 
effects, and is functional in this release of {\feffcur}.

See \emph{Combined Bethe-Salpeter equations and time-dependent density-functional 
theory approach for x-ray absorption calculations}; A.L.\ Ankudinov, Y.\ Takimoto, 
and J.J.\ Rehr, Phys.\ Rev.\ B 71, 165110 (2005)



%%%%%%%%%%%%%%%%%%%%%%%%%%%%%%%%%%%%%%%%%%%%%%%%%%%%%%%%%%%%%%%%%%%%%%%%%%% CIF-FILE %%%%%%%%%%%%%
\section{Using a .cif file}

The \file{*.cif} file format is a standard way of specifying structural data.  Such files are available in online databases for a very large number of materials.  {\feff} can read such files directly, removing the need for the user to worry about, e.g., details of crystallography and space group notations to generate a list of cluster coordinates for \file{feff.inp}.  

The \file{*.cif} file import feature is currently only implemented for calculations of crystals.  We intend to enable it for molecules also; please contact us for collaboration if you have an interest in this.  \file{cif} files can be used for real-space or k-space calculations, for any type of spectrum.

{\feff} uses the CIFTBX library to read \file{.cif} files.  This should allow you to use any valid \file{.cif} file.  "Warning - library not found" warnings at runtime can usually be ignored.  The \htmladdnormallink{Open Crystallography Database}{http://www.crystallography.net/search.html} is an excellent resource for \file{.cif} files that can be accessed from anywhere.  The \htmladdnormallink{Inorganic Crystal Structure Database}{http://icsd.fiz-karlsruhe.de/} is another good resource for those with an academic subscription.  We recommend saving the \file{.cif} file to the working directory containing also \file{feff.inp} and renaming it with a useful filename.

Use the CIF CARD as in the \htmlref{Cr2GeC example}{K-space-example-Cr2GeC} below.  You cannot use an ATOMS card in the same \file{feff.inp} file.  It is no longer necessary to use a POTENTIALS card.  {\feff} automatically assigns a potential type to each crystallographically inequivalent atom in the unit cell.  These potentials are given the default angular cutoffs ($lmax$) based on atomic number.  Inspecting \file{pot.inp} or \file{log1.dat} shows the list of assigned potentials.  Note that the current scheme could be problematic for large unit cells, where there might be many crystallographically inequivalent atoms of the same atomic number.  Calculating a large number of potentials will make the calculation slow and potentially unstable (the SCF algorithm may have a hard time reaching convergence).  Furthermore, in our experience respecting strict crystallographic equivalence does not always improve the accuracy of a calculation.  For such large systems, it may be preferable to make all atoms of the same atomic number equivalent, or to determine equivalence based on first-shell coordination only.  The EQUIVALENCE card achieves that end.

If both a CIF card and a POTENTIALS card are present in \file{feff.inp}, the program does the following.  First, it determines if the list of potentials in the POTENTIALS card matches exactly that generated from the CIF file.  If it does not match, the POTENTIALS card is ignored and a warning is printed to the screen.  If the list of potentials does match, the program takes the options for the potentials (i.e.,  lmax1  lmax2  xnatph  spinph ) and uses these values instead of the defaults generated from the CIF file.  This way, the user can control the angular momentum cutoff and set the spin-related variables.

Finally, a word of warning on using the TARGET card to indicate the corehole atom.  Combined with the CIF card, this counts an atom in a list of the crystallographically inequivalent atoms in the unit cell (e.g., 2 C atoms for graphite) in the order given in the CIF file.  However, combined with LATTICE/ATOMS cards, it counts an atom in a list of all atoms in the unit cells (e.g., 4 C atoms for graphite, 2 of which are equivalent by symmetry to the other 2) in the order given in the ATOMS card.  Hence, the value of the TARGET card must be reevaluated if you switch from one representation to the other.  Note that the JFEFF GUI currently does not display the content of the CIF file; you must look at the file yourself to find out in what order the atoms are listed and identify which one you want to place the core hole on.  If it is the third atom listed in the CIF file, use "TARGET 3" in \file{feff.inp} or in the JFEFF GUI.

\subsection{Formal requirements for cif files}
Although CIF is a mature standard, there is a wild proliferation of options and input fields for cif files.  We've come across some files that seem to contain endless "junk" entries, or that enter data in sufficiently nonstandard ways to confuse {\feff}.

With most cif files there is no problem, although we sometimes delete fields we don't need just for the sake of simplicity.  When there is a problem, it is almost always due to one of two things:
* The H-M space group is specified in a nonstandard way, often with extra ":1" or " 1" symbols: the solution is to delete these extra characters;
* The atomic symbols are not specified: in this case, one simply adds them.

The Cr2GeC example shows a good cif file.  The following fields are required by the {\feff} parser:
\begin{verbatim}_cell_length_a, _cell_length_b, _cell_length_c
_cell_angle_alpha, _cell_angle_beta, _cell_angle_gamma
_atom_site_type_symbol, _atom_site_label, _atom_site_fract_x, _atom_site_fract_y, _atom_site_fract_z
_symmetry_space_group_name_H-M or _symmetry_Int_Tables_number
_symmetry_equiv_pos_as_xyz \end{verbatim}

Most cif files contain much, much more information than this.


%%%%%%%%%%%%%%%%%%%%%%%%%%%%%%%%%%%%%%%%%%%%%%%%%%%%%%%%%%%%%%%%%%%%%%%%%%% K-SPACE %%%%%%%%%%%%%

\section{K-space FEFF}
\label{sec:kspace}

Although {\feff} is traditionally a real-space code, {\feffcur} is capable of calculating infinite periodic systems (crystals) in
reciprocal space.  The reciprocal space engine is based on impurity KKR.  This functionality is implemented through a handful of new input CARDs.  The main difference from the real-space code is that the system is specified in terms of a unit cell with lattice vectors and a basis of atoms in the unit cell, instead of a real space cluster.  We  describe how to do this below.

First, the RECIPROCAL card switches {\feff} to work in k-space.  

Next the crystal structure must be specified.  This can be done in two ways.  The recommended way (new as of {\feff}-9.5.1) is to use the CIF card to specify the name of a \file{*.cif} file containing the crystal structure.  The second way, which was the default in prior {\feff} versions but is less user-friendly and more error-prone, is to use the following CARDs in \file{feff.inp} : the LATTICE card gives the basis vectors spanning the unit cell ; and the ATOMS card lists all atoms in the unit cell. 

In both cases, the TARGET card places the absorber on one of the sites (either the list of sites in the ATOMS card, or the list of sites in the \file{.cif} file) ; and the KMESH card sets the size of the mesh used to sample Brillouin Zone integrals.  There are also a few optional cards, notably COORDINATES (sets the units for the ATOMS card if one uses the LATTICE/ATOMS approach), and SGROUP (only for the LATTICE/ATOMS approach).  STRFAC is an advanced card that adds internal broadening for convergence of lattice sums and is not usually needed.

In the JFEFF GUI only the CIF method is supported.  Select "Import CIF file" from the "Atoms" pulldown menu.  A "Select CIF file" appears, and the RECIPROCAL, TARGET, KMESH, and STRFAC cards become clickable.  If you click the "Jmol" button, you will see the unit cell (only the generating atom positions are shown).

If a core hole is needed, we recommend using COREHOLE RPA.  Note that NRIXS cannot currently be run in k-space.


\subsection{Core hole calculations without the supercell}
It is typical to calculate spectra of crystals using band-structure codes, which calculate crystals very efficiently in k-space using periodic boundary conditions (PBC).  But the introduction of a core hole is in essence identical to an impurity calculation and breaks the periodicity of the ground state of the crystal.  The typical solution is to construct a supercell large enough to separate the core hole atom from its twin in the neighboring unit (super)cell so that unphysical interactions are avoided.  This approach is inefficient and can be computationally expensive.  It is complicated by the absence of a universally appropriate size for the supercell, requiring a convergence study.

The {\feff} code operates similarly to impurity KKR calculations and is able to combine the best of both worlds.  The Green's function of the ground state crystal is first calculated in k-space and then transformed back to r-space.  The core hole impurity is added to the r-space Green's function through simple matrix algebra.  This is inexpensive and avoids the need for a supercell altogether.  Note that this requires using the "COREHOLE RPA" setting.  This is the recommended way of calculating a core hole spectrum of a crystal in {\feff}.  (The default "COREHOLE FSR" setting puts the core hole explicitly in the unit cell and would correspond to the bandstructure code calculation.)  No further user effort is required.



\subsection{Using the file \file{reciprocal.inp}}
This intermediate file written by \module{rdinp} contains all the parameters for k-space calculations.  The first parameter is the main switch : if it's set to 0, the calculation will be done in real space and all the parameters following it are ignored.  If it's set to 1, the calculation is done in reciprocal space, and all the parameters following it are read and used.  This allows the advanced user some leeway ; one could eg. calculate potentials in reciprocal space and FMS in real space by toggling this variable (perhaps for NRIXS).


\subsection{Converging the k-mesh}
The k-mesh is constructed using the tetrahedron of Bloechl et al., Phys. Rev. B, 1990.  It is written to file early on.
The number of k-points in the k-mesh is an unphysical parameter that simply needs to be converged.  Although it is impossible to give a general guideline, starting with 1000 k-points is a good idea for small unit cells.  Generally, the number of points needed scales inversely with the volume of the unit cell.  Some systems require more points than others.  One always needs to check.  The more broadened the property of interest (e.g. ELNES as opposed to DOS), the fewer points are necessary.  Also, the near edge structure requires more points, whereas more extended structure (e.g., 50-70 eV above threshold) is often converged with just a few k-points.

Generally speaking, the calculations of the potentials requires less accuracy than the calculation of FMS.  Just like one usually uses a smaller SCF-radius than the FMS-radius for real-space calculations, it makes sense to use a smaller k-mesh for SCF than for FMS.  Therefore, it can be a good strategy to, e.g., set the k-mesh to 200 points, run the potentials calculation, then raise the number of k-points to, e.g., 1000, and then run the FMS calculation.  This can save much calculation time, but requires more skill from the user.

Some time-saving schemes are implemented through the KMESH card and are described there.

\subsection{EXAFS and EXELFS}
For extended loss structure (i.e., upwards from 50-100 eV), the real space Path Expansion method is so efficient and robust that we do not see the point in trying to reformulate it in reciprocal space.  Therefore, it is always done in real space.
If the RECIPROCAL card is active, the rdinp module generates a real-space cluster based on the crystal structure and the value of the RPATH value.  This cluster is written to the atoms.dat file.  The spectrum is then calculated by the \module{path} and \module{genfmt} modules, which always work in real space.
It is good practice to look at this atoms.dat file when one is still learning to use k-space {\feff} - it's a good check to make sure no mistakes have been entered.


\subsection{Using crystal symmetry}
Symmetry could be used in two ways :
\begin{itemize} \item	reduce the k-mesh
\item reduce the Green's function $L,L'$ matrix \end{itemize}

The first of these is currently implemented.  However!  The extent to which symmetry can be used depends on what one is calculating.  E.g., to calculate the diagonal parts of $G_{LL'}$, one can reduce the k-mesh and just sum all the contributions from inequivalent k-vectors.  But for the offdiagonal components (which are needed in order to add the core hole), symmetry is more subtle and one needs to add all the equivalent ones, which can be somewhat shortcut by reconstructing them from the inequivalent ones through unitary transformations dictated by symmetry elements of the crystal.  Whereas the first strategy would yield a speedup ~ x 48 for diamond, the gain for the latter strategy is much more modest (~ x 5?) and depends on the size of the matrices, i.e. the number of atoms in the unit cell.

So, while a number of symmetry strategies are implemented in the code, it is currently recommended to test these cautiously on a smaller k-mesh before relying on them.  Testing against real-space results is another safety check.  We have not tested all possible lattice types and errors may occur.  Contact the authors in case of doubt.  Note that many symmetry options are disabled in the code for safety reasons and would require recompilation.

The second idea - $LL'$ symmetry within the $G$ matrix - has not been implemented in any way.


\subsection{Speed}
Generally, \module{fms} matrix inversion takes most of the computation time.
In real space, for every energy point {\feff} does one matrix inversion of order $nclus*(lmax+1)**2*nsp$.
In reciprocal space, for every energy point {\feff} does $nkp$ matrix inversions of order $nu*(lmax+1)**2*nsp$.
 Here, $nu$ is the number of atoms in the unit cell, $nkp$ the number of k-vectors in the mesh, $nclus$ the number of atoms in the real space cluster, $lmax$ the angular momentum cutoff, $nsp$ the number of spins (1 or 2).
So, the relative speed is something like $(nu/nclus)^a * nkp$ , where $a$ is the scaling of matrix inversion.
In general, unit cells with more atoms ($nu$) are larger and therefore require less k-vectors ($nkp$).  For small systems (1-10 atoms in the unit cell), RECIPROCAL is faster.  For larger systems, it depends.


\subsection{Example input files}
\label{K-space-example-Cr2GeC}
The Ge L3 edge of Cr2GeC using a Cr2GeC.cif file.  This is the only kind of {\feff} calculation that doesn't require a POTENTIALS card and an ATOMS or OVERLAP card.

\begin{verbatim}
TITLE Cr2GeC    (a=2.94 c=12.11)
*  Ge L3 edge energy = 1217.0 eV
EDGE L3
S02 1.0
COREHOLE None
XANES 20.0 0.07 0.0

CONTROL 1 1 1 1 1 1
PRINT 5 1 1 1 1 1

SCF 4.0
FMS 6.0
LDOS -30 15 0.01

* Options for a k-space calculation :
RECIPROCAL
* Use 200 k-points:
KMESH 200
* Spectrum of 3rd atom type in the cif file (Ge) :
TARGET 3
* This advanced option is not usually necessary:
STRFAC 1.0 0.0 0.0

* Read crystal structure from cif file:
CIF Cr2GeC.cif
END
\end{verbatim}

The \file{Cr2GeC.cif} file :
\begin{verbatim}
data_Cr2GeC
_cell_length_a                   2.9400(0)
_cell_length_b                   2.9400(0)
_cell_length_c                  12.1100(0)
_cell_angle_alpha               90.0000(0)
_cell_angle_beta                90.0000(0)
_cell_angle_gamma              120.0000(0)

_symmetry_space_group_name_H-M     'P 63/m m c'
_symmetry_Int_Tables_number         194
_symmetry_cell_setting             hexagonal
loop_
_symmetry_equiv_pos_as_xyz
'+x,+y,+z'
'-y,+x-y,+z'
'-x+y,-x,+z'
'-x,-y,1/2+z'
'+y,-x+y,1/2+z'
'+x-y,+x,1/2+z'
'-y,-x,+z'
'-x+y,+y,+z'
'+x,+x-y,+z'
'+y,+x,1/2+z'
'+x-y,-y,1/2+z'
'-x,-x+y,1/2+z'
'-x,-y,-z'
'+y,-x+y,-z'
'+x-y,+x,-z'
'+x,+y,1/2-z'
'-y,+x-y,1/2-z'
'-x+y,-x,1/2-z'
'+y,+x,-z'
'+x-y,-y,-z'
'-x,-x+y,-z'
'-y,-x,1/2-z'
'-x+y,+y,1/2-z'
'+x,+x-y,1/2-z'

loop_
_atom_site_type_symbol
_atom_site_label
_atom_site_fract_x
_atom_site_fract_y
_atom_site_fract_z
  C        C      0.0000    0.0000    0.0000
  Cr      Cr      0.6667    0.3333    0.0833
  Ge      Ge      0.6667    0.3333    0.7500

\end{verbatim}

(Most \file{cif} files contain many additional fields.  We have removed all the information that is ignored by {\feff}.)

The graphite C K edge using LATTICE and ATOMS cards.

\begin{verbatim}
 TITLE graphite
 *  C K edge energy = 284.20 eV
 EDGE      K   0.0
 COREHOLE RPA
 CONTROL   1  1  1  1  1  1
 
 ELNES 5.0 0.05 0.05
 300   # beam energy in keV
 0 0 1  # beam direction in the crystal frame
 1.10 0.3 # collection semiangle, convergence semiangle (in mrad)
 100 1  # q-integration mesh : radial size, angular size
 0.0 0.0 # position of the detector (x,y angle in mrad)

 MAGIC 40   # create plot that shows magic angle.  Evaluate at 40 eV above threshold.

 FMS       6.0
 SCF       4.5

 POTENTIALS
 *    ipot   Z  element            l_scmt  l_fms   stoichiometry
        0    6   C                  3       2       0.01
        1    6   C                  3       2       2
        2    6   C                  3       2       2

 RECIPROCAL
 KMESH 1000
 TARGET 1
 LATTICE P 2.456
      0.86603     -0.50000      0.00000
      0.00000      1.00000      0.00000
      0.00000      0.00000      2.72638
 
 ATOMS
 *   x          y          z          ipot  tag           
      0.00000      0.00000      0.68160  1  C1
      0.00000      0.00000      2.04479  1  C1
      0.57735      0.00000      0.68160  2  C2
      0.28868      0.50000      2.04479  2  C2
 END
\end{verbatim}



%%%%%%%%%%%%%%%%%%%%%%%%%%%%%%%%%%%%%%%%%%%%%%%%%%%%%%%%%%%%%%%%%%%%%%%%%%% CHARGE TRANSFER AND CHARGE COUNTS %%%%%%%%%%%%

\section{Charge Transfer and Charge Counts}
\label{sec:ChargeTransfer}

The charge tranfer in each \file{ldosNN.dat} file and in \file{log1.dat} is related to the amount of charge transferred into the Norman radius during the SCF routine. Before the SCF routine, the Norman radius is the radius of the neutral sphere, so the charge transfer can be related to the electronegativity or the oxidation state, although the actual numbers are usually much smaller than the formal oxidation state. The electron counts give the number of valence electrons in each of the angular momentum channels after the SCF has completed. The valence electrons are assumed to be those binding binding energy above and energy "ecv", which is usually at -40 eV, but can sometimes change if the atoms in the problem have binding energies close to -40eV. The difference between these numbers and the number of valence electrons in these channels in the atomic system should be equal to the charge transfer into each channel. If these are summed over angular momentum channels, you should get the total charge transfer. The only case where this is different is in the absorbing atom, which starts with and extra electron in the valence. As an example, we can look at NaF which gives

\begin{verbatim}
 Electronic configuration
  iph    il      N_el
    0     0    0.521
    0     1    6.535
    0     2    0.358
    0     3    0.000
    1     0    1.958
    1     1    5.285
    1     2    0.014
    1     3    0.000
    2     0    0.221
    2     1    6.292
    2     2    0.236
    2     3    0.000
\end{verbatim}
Here iph = 0 is the absorbing Na atom, iph = 1 is the F atom, and iph = 2 is
the other Na atom. These atoms start with the configuration
\begin{verbatim}
 Electronic configuration
  iph    il      N_el
    0     0    2.0  - extra screening electron
    0     1    6.0
    0     2    0.0
    0     3    0.0
    1     0    2.0
    1     1    5.0
    1     2    0.0
    1     3    0.0
    2     0    1.0
    2     1    6.0
    2     2    0.0
    2     3    0.0
\end{verbatim}
Looking at the difference between these two, you can see that in the solid state, some electrons are transferred from the sodium s to the sodium p and d states, and some to the F p states. There is also a small tranfer from the F s states, and into the F d states, but minor. Note that the final configuration allows more than 6 electrons in the Na p states. This is because the p-states are not limited to the 2p states, but include the 3p as well.


%%%%%%%%%%%%%%%%%%%%%%%%%%%%%%%%%%%%%%%%%%%%%%%%%%%%%%%%%%%%%%%%%%%%%%%%%%% DW FACTORS %%%%%%%%%%%%%

\section{Ab initio Debye-Waller factors}
\label{sec:DWfactors}
XAFS analysis can provide structural information, including average near-neighbor distances $R$, their mean square fluctuations $\sigma_{R}^{2} $, and coordination numbers $N_R$.  The quantities $\sigma_{R}^{2} $ that appear in the XAFS Debye Waller(DW) factor are crucial to the success of the modern theory of XAFS and its applications.  The DW factor accounts for thermal and structural disorder and generally governs the "melting" of the XAFS oscillations with respect to increasing temperature and their decay with respect to increasing photoelectron energy.  In practice, the DW factors of the many multiple-scattering terms in the XAFS signal can significantly complicate the analysis.  To overcome these difficulties, {\feff} offers several ways to calculate the Debye-Waller factors and account for the effects of thermal disorder in the {\it ab initio} XAFS calculations.  These are described below.

Fig. \ref{fig:DWschema} illustrates the way DW factors enter the XAFS problem.

\begin{figure}[H]
	\centering
		\includegraphics[height=6.5in]{annafig1.pdf}
	\label{fig:DWschema}
\end{figure}



If one uses the multiple scattering path expansion (PE), the DW factors are added
to each path individually in module \module{ff2x}.  If one uses Full Multiple Scattering (FMS), the effect of finite temperature is approximated by multiplying each free propagator by $e^{-\sigma^{2}k^{2}}$.  This is only exact for single scattering paths, but since the effect of thermal disorder is reduced in the near-edge region anyway, it's probably adequate.

There are three ways to modify the Debye--Waller factor. The \htmlref{DEBYE}{card:deb2} card
calculates a Debye--Waller factor for each path in PE or FMS. The \htmlref{SIG2}{card:sig} and SIG3 and SIGGK card add
various constant or near-constant Debye--Waller factors to all paths in PE. Finally, you can edit \file{list.dat}
to add a Debye--Waller factor to a particular path in PE. In PE, these three
Debye--Waller factors are summed, so if the DEBYE and SIG2/SIG3/SIGGK cards are present,
and if you have added a Debye--Waller factor to a particular path in \file{list.dat}, the Debye--Waller factor 
used will be the sum of all three.  

After changing the DW factors, {\feff} must be rerun starting with module \module{ff2x} for PE and starting with module \module{fms} for FMS.

The \htmlref{DEBYE}{car:deb2} card offers a choice between 5 different models for the DW factors:
\begin{itemize}
				\item  0	Correlated-Debye method  (default) (CD)
				\item  1	Equations of Motion method (EM)
				\item  2	Recursion method (RM)
				\item  3	Classical Correlated-Debye method (CCD)
				\item  4	Read from "sig2.dat" file
				\item  5	Dynamical-Matrix method (DM)
				\item <0	Do not calculate DW factors
\end{itemize}

\begin{figure}[H]
	\centering
		\includegraphics[height=2.6in]{annafig2.pdf}
	\label{fig:DWAnnaCompares}
\end{figure}


Only method CD and CCD can run without additional input.  These Correlated Debye models are isotropic and can be very inaccurate for anisotropic materials.
Methods 1, 2 and 5 require that
the force constants or the dynamical matrix be provided.  We will now describe these requirements from a computational point of view.  For more information on the physics
behind these calculations, we refer to \htmlref{Anna Poiarkova's thesis}{http://leonardo.phys.washington.edu/feff/papers/dissertations/thesis_poiarkova.ps} and our \htmlref{paper on the DM method}{http://leonardo.phys.washington.edu/feff/papers/fdv-ab-initio-dw-factors.pdf}.  

\subsection{EM and RM methods}
The Equation of Motion (EM) and Recursion Method (RM) require additional input to be given in the file \file{spring.inp}.  This file cannot currently be produced from the JFEFF GUI.
The file defines the force fields for {\feff}.  First, {\feff} searches for all similar bonds and angles in the material, creates complete lists of all bond stretches and angle bends, and then, based on the force field and geometry of the structure, calculates the cartesian force field matrix and scales it with the atomic masses, thus obtaining the dynamical matrix.  Finally, the DW factors are calculated from the dynamical matrix.

The file \file{feff.inp} has the same CARD-based structure as the master input file \file{feff.inp}.  For example, the \file{spring.inp} file for zinc tetraimidazole looks like this.
\begin{verbatim}
* 13-atom model of zinc tetraimidazole
*                 res          wmax         dosfit         acut
VDOS             0.02             1            1.2            3
PRINT 5
STRETCHES  *   i         j        k_ij       dR_ij[%]
               0         2        110.        2.
               1         2        626.        5.
ANGLES  * i     j    k     ktheta      dtheta[%]
          2     0    5         37.           10.
          1     2    3       2590.           10.
\end{verbatim}
The corresponding \file{feff.inp} file contains the DEBYE card:
\begin{verbatim}
       *    T   T_Debye     "2"=Equation of Motion Method
DEBYE    300.        0.    2    
\end{verbatim}

The CARDs for the \file{spring.inp} file are:

\begin{Card}{VDOS}{res wmax dosfit [acut]}{Standard}{spring-vdos}
  This card is needed only for EM runs and is optional - if it is omitted, default values are used.  The card is ignored for RM runs.  The keywords in the VDOS card define the integration parameters used in the
  VDOS calculation.  Here \texttt{res} is the VDOS spectral resolution width (default \texttt{res} = 0.05, i.e. 5\% of the bandwidth).  The smaller this number, the more fine structure is present in the spectrum and the longer the computation time.  Finer resolution is usually helpful for mode analysis in small molecules.  The next keyword, \texttt{res}, is a multiplication factor used to increase the maximum frequency to which the VDOS is calculated.  \texttt{dosfit} is a real positive number governing how much of the low frequency part of the VDOS is to be fitted to Debye-like behavior, $A . \omega^2$.  If it is equal to 0 then no fitting will be applied.  The higher the number, the more of the VDOS will be fitted.  The default value is \texttt{dosfit} = 1 (about 10\% of the total width).  This parameter is useful for elimination of low frequency "noise" and zero-frequency modes.  Finally, \texttt{acut} is the time integration cutoff parameter.  It rarely needs to be changed (usually in cases of very small open molecular structures).  The higher this number, the longer the computation time.  The \texttt{acut} keyword is optional; the default value is \texttt{acut} = 3.
\begin{verbatim}
*                 res          wmax         dosfit         acut
VDOS             0.02             1            1.2            3
\end{verbatim}
\end{Card}


\begin{Card}{PRINT}{[iprdos]}{Standard}{spring-print}
  If using the EM method, the PRINT card makes {\feff} write files \file{prdenNNNN.dat} containing projected VDOS for selected scattering paths.  Here \texttt{iprdos} specifies that such files will be written for the first \texttt{iprdos} paths in the paths list.  If using the RM method, the PRINT card makes {\feff} write a file \file{s2\_rm1.dat} containing first tier results.  The keyword \texttt{iprdos} is ignored in this case.
\begin{verbatim}
PRINT 5   * print files prden0001.dat - prden0005.dat
\end{verbatim}
\end{Card}



\begin{Card}{STRETCHES}{[i   j   k\_ij   dR\_ij]}{Standard}{spring-stretches}
  Required for EM and RM runs.  It is followed by the list of bond stretching force constants.  Here $i$ and $j$ are atomic indices (as in the file \file{geom.dat}; the absorber has index 0), and \texttt{k\_ij} is a single central force constant characterizing the interaction between atoms $i$ and $j$ in units of $10^2 mdyn/{\AA}^2$ or $N/m$.  One should include as many distinct bonds in the list as possible and then the code will search for the similar ones and assign them the same force constant.  The last parameter in each row, \texttt{dR\_ij}, is the tolerance in the bond length when searching for similar bonds and is measured in percentage points.  For example, if \texttt{dR\_ij} = 5 then all bonds between pairs of atoms with the same potentials as $i$ and $j$ and with bond length within 5\% of $R_ij$ will be assigned the same stretching force constant \texttt{k\_ij}.
\begin{verbatim}
STRETCHES  *   i         j        k_ij       dR_ij[%]
               0         2        110.        2.
               1         2        626.        5.
\end{verbatim}
\end{Card}


\begin{Card}{ANGLES}{[i   j   k    ktheta   dtheta]}{Standard}{spring-angles}
  Similar to STRETCHES, but optional in most cases.  This card allows one to include $\theta_{ijk}$ angle bending force constants $k_{\theta}^{ijk}$ in the calculation.  The force constants are in units of $10^2 mdyn \AA / rad^2$.  Here \texttt{dtheta} is tolerance in the angle value when searching for similar angles.  Sometimes it is useful to include this card in order to avoid zero-frequency modes. 
\begin{verbatim}
ANGLES  * i     j     k     ktheta      dtheta[%]
          2     0    5         37.           10.
          1     2    3       2590.           10.
\end{verbatim}
\end{Card}


\subsubsection{Output files}
All $\sigma_j^2$ values in the output files are given in units of ${\AA}^2$, all frequencies are in units of THz, and all reduced masses are in atomic units.
The output files from a EM calculation include:
\begin{itemize}
\item \file{s2\_em.dat}  Contains $\sigma_j^2$ for each scattering path in \file{paths.dat}. 
\item \file{prdenNNN.dat} which contains projected VDOS for selected scattering paths (as indicated by the PRINT card).
\end{itemize}

The output files from a RM calculation include:
\begin{itemize}
\item \file{s2\_rm2.dat}  Contains $\sigma_j^2$ for each scattering path in \file{paths.dat} calculated using the second tier approximation.
\item \file{s2\_rm1.dat}  Contains $\sigma_j^2$ for each scattering path in \file{paths.dat} calculated using the first tier approximation.
\end{itemize}

Further details can be found on pp. 98 of  \htmlref{Anna Poiarkova's thesis}{http://leonardo.phys.washington.edu/feff/papers/dissertations/thesis_poiarkova.ps}

\subsubsection{Examples}

These examples can also be found in the \file{examples} folder of the feff90 distribution.

First we present a \file{spring.inp} example for a 177-atom cluster of a Cu crystal.  Here only a single central force constant between the first nearest neighbors is taken into account.  Similar input files can be constructed for other fcc structures, e.g. Pt, Al, etc.
\begin{verbatim}
* Cu crystal, single central force constant
VDOS   0.03    0.5   1
PRINT  3
STRETCHES
0    1     27.9     2.
\end{verbatim}


As a second example we show the \file{spring.inp} file for a 147-atom cluster of c-Ge crystal.  The force constants used here were fitted to phonon dispersion curves.  Similar output files can be constructed for other diamond-type lattices, e.g. Si, C, etc.
\begin{verbatim}
* c-Ge crystal
VDOS    0.02    0.7    0.    3.
PRINT   6
STRETCHES
0    1    103.58    2.
0    5      5.81    2.
0   20     -1.08    2.
0   30     -0.30    2.
ANGLES
1    0    2    31.45    2.
\end{verbatim}

The output files for these examples can be found in the \file{examples} folder of the feff90 distribution.  They are also discussed in pp. 100-104 of  \htmlref{Anna Poiarkova's thesis}{http://leonardo.phys.washington.edu/feff/papers/dissertations/thesis_poiarkova.ps}.




\subsection{DM method}

\subsubsection{Preparation}

First, the user needs to calculate the dynamical matrix.  {\feff} cannot do this, but many other codes can, e.g. ABINIT or Gaussian.  Using such a code, the user needs to produced a file that contains the dynamical matrix, and which we will refer to as the \file{dymfile}.

Before explaining how to include dynamical matrix DW factors in EXAFS and XANES
{\feff} calculations, it is important that the user becomes familiar with some
constraints and limitations that are present in the current implementation. These
limitations stem from the fact that both the dynamical matrix file and the FEFF input
file contain structural information. These structures must match for the DW
values to be correct. Since FEFF internally sorts the atoms according to
distance to the absorber, special care must be taken to avoid mismatching the
structural information coming from the FEFF input with that from the \file{dymfile}.

Therefore, it is not recommended to start from an existing \file{feff.inp} file, add a \file{dymfile}, and run the calculation.
Instead, we recommend to start from the dymfile, and generate correctly matched \file{feff.inp} and \file{feff.dym} files from the dymfile.  Then we can edit the newly generated \file{feff.inp} file to add CARDS, before running the {\feff} calculation.

\textbf{dym2feffinp} is a utility that helps in the generation of FEFF input files with
structures that match those in the dynamical matrix files. The usage of
dym2feffinp is as follows:

\begin{verbatim} dym2feffinp [Options] dymfile
\end{verbatim}

where dymfile is the name of the file containing the dynamical matrix. This
command creates two files, \file{feff.dym} and \file{feff.inp}, which contain correctly
matched structures. (Note of warning: If dymfile is named "feff.dym", then
dym2feffinp will APPEND the updated dym information to it. Please don't call the dymfile 'feff.dym'. Future version will check that dymfile
doesn't use the "feff.dym" name.)

The default behavior of dym2feffinp can be modified with the following options:
\begin{itemize}
\tightlist
 \item  --c  iAbs   Use atom iAbs as absorber
  \item --f  fname  Write feff input to file fname
 \item  --d  dname  Write adjusted dym file to file dname
\end{itemize}
The --c options allows the user to choose different absorbers. The usual
approach of editing a FEFF input file and changing the potential type of a
certain atom to 0 will result in mismatched FEFF and dym structures. This should
be avoided. It is recommended that different input files be generated using the
--c option.

The --f and --d change the default output filenames from "feff.inp" and
"feff.dym" to fname and dname, respectively.

\subsubsection{Feff.inp syntax}
To calculate Debye-Waller factors from a dynamical matrix (or matrix of force constants or Hessian matrix) using the Lanczos recursive algorithm, the following syntax is required in \file{feff.inp} :
\begin{verbatim}
DEBYE Temp Debye_Temp  [DW_Opt [dymFile DMDW_Order DMDW_Type DMDW_Route]]
\end{verbatim}
where:
\begin{itemize}
\tightlist
	\item Temp		Temperature at which the DW factors are calculate
	\item Debye\_Temp	Debye Temperature of the material
	\item DW\_Opt		The model used to calculate the DW factors - here, $DW\_opt=5$
	\item dymFile		Name of the dynamical matrix information file. The
			default value is \file{feff.dym}
	\item DMDW\_Order	Lanczos recursion order to be used in the calculation.
			The default value is 2. Well converged results are
			usually obtained for $DMDW\_Order=6-10$. For small size
			systems, these values might be too large. As a rule
			of thumb, DMDWOrder should be less than
			$3*(Number of atoms)-6$. Some paths, within systems with
			high symmetry, might require a lower DMDW\_Order. The
			user should always check convergence.
	\item DMDW\_Type	Type of DW calculation.
			The possible values are:
				0	Parallel $s^2$  (default)
	\item DMDW\_Route	Which paths to use in the dmdw module. These paths do
			not affect the path selection in the XAS calculations,
			they are used for the generation of an input file for
			the independent dmdw module.
			The possible values are:
			\begin{itemize}
			\tightlist
			\item	 0	Skip \module{dmdw} module  (default)
			\item	 1	All SS paths from absorber
			\item	 2	Same as 1 + all DS paths from absorber
			\item	 3	Same as 2 + all TS paths from absorber
			\item	11	All SS paths
			\item	12	Same as 1 + all DS paths
			\item	13	Same as 2 + all TS paths
				\end{itemize}
\end{itemize}

\subsubsection{Example 1: Creating a FEFF input file from a dym file using dym2feff.inp}
 
 Here we demonstrate how to convert a dym file, in this case created from a
Gaussian fchk file, into matched pairs of \file{feff.inp} and \file{feff.dym} files, for
different absorbing centers.

The dym file for a CO2 molecule \file{CO2.dym}, converted from the formatted
checkpoint file, looks like:
\begin{verbatim}
###############################################################################
   1
   3
   8
   6
   8
    15.99491460
    12.00000000
    15.99491460
   0.00000000   0.00000000   2.20979482
   0.00000000   0.00000000   0.00000000
   0.00000000   0.00000000  -2.20979482
   1   1
  3.501599e-02 -7.311989e-13 -8.941376e-12
 -7.311989e-13  3.501599e-02  3.206256e-11
 -8.941376e-12  3.206256e-11  1.042343e+00
   1   2
 -7.001817e-02 -5.485621e-12  2.501278e-11
 -3.211930e-12 -7.001817e-02 -2.341738e-11
 -3.124223e-11 -3.615378e-11 -9.594793e-01
   1   3
  3.500217e-02 -8.681691e-13 -1.846564e-11
  8.681691e-13  3.500217e-02 -1.245331e-11
 -1.846599e-11  1.245327e-11 -8.286417e-02
   2   1
 -7.001817e-02 -3.211930e-12 -3.124223e-11
 -5.485621e-12 -7.001817e-02 -3.615378e-11
  2.501278e-11 -2.341738e-11 -9.594793e-01
   2   2
  1.400363e-01  1.883207e-12  2.058909e-11
  1.883207e-12  1.400363e-01  4.005329e-11
  2.058909e-11  4.005329e-11  1.918959e+00
   2   3
 -7.001817e-02  2.586772e-12  5.500370e-12
  4.227571e-12 -7.001817e-02  3.602353e-12
  3.495310e-11  1.591451e-11 -9.594793e-01
   3   1
  3.500217e-02  8.681691e-13 -1.846599e-11
 -8.681691e-13  3.500217e-02  1.245327e-11
 -1.846564e-11 -1.245331e-11 -8.286417e-02
   3   2
 -7.001817e-02  4.227571e-12  3.495310e-11
  2.586772e-12 -7.001817e-02  1.591451e-11
  5.500370e-12  3.602353e-12 -9.594793e-01
   3   3
  3.501599e-02  7.311989e-13 -8.940090e-12
  7.311989e-13  3.501599e-02 -3.206153e-11
 -8.940090e-12 -3.206153e-11  1.042343e+00
###############################################################################
\end{verbatim}
The dym files are not required to be sorted in any particular order, they retain
the atom order of the program that generated the dynamical matrix. In this case,
the order is O, C and O, with the C atom at the origen. Since FEFF internally
sorts the atoms according to their distance to the absorber, if this dym file is
used as is to compute the EXAFS/XANES for the C atom, the results would be
incorrect. We can generate the appropriate FEFF input file and associated dym
file with the following command:

\begin{verbatim} dym2feffinp --c 2 --f CO2-C-feff.inp --d CO2-C-feff.dym CO2.dym 
\end{verbatim}

This creates new files \file{CO2-C-feff.inp} and \file{CO2-C-feff.dym}.  The CO2-C-feff.dym file:
\begin{verbatim}
###############################################################################
    1
    3
    6
    8
    8
   12.000000
   15.994915
   15.994915
    0.00000000    0.00000000    0.00000000
    0.00000000    0.00000000    2.20979482
    0.00000000    0.00000000   -2.20979482
    1    1
  1.400363E-01  1.883207E-12  2.058909E-11
  1.883207E-12  1.400363E-01  4.005329E-11
  2.058909E-11  4.005329E-11  1.918959E+00
    1    2
 -7.001817E-02 -3.211930E-12 -3.124223E-11
 -5.485621E-12 -7.001817E-02 -3.615378E-11
  2.501278E-11 -2.341738E-11 -9.594793E-01
    1    3
 -7.001817E-02  2.586772E-12  5.500370E-12
  4.227571E-12 -7.001817E-02  3.602353E-12
  3.495310E-11  1.591451E-11 -9.594793E-01
    2    1
 -7.001817E-02 -5.485621E-12  2.501278E-11
 -3.211930E-12 -7.001817E-02 -2.341738E-11
 -3.124223E-11 -3.615378E-11 -9.594793E-01
    2    2
  3.501599E-02 -7.311989E-13 -8.941376E-12
 -7.311989E-13  3.501599E-02  3.206256E-11
 -8.941376E-12  3.206256E-11  1.042343E+00
    2    3
  3.500217E-02 -8.681691E-13 -1.846564E-11
  8.681691E-13  3.500217E-02 -1.245331E-11
 -1.846599E-11  1.245327E-11 -8.286417E-02
    3    1
 -7.001817E-02  4.227571E-12  3.495310E-11
  2.586772E-12 -7.001817E-02  1.591451E-11
  5.500370E-12  3.602353E-12 -9.594793E-01
    3    2
  3.500217E-02  8.681691E-13 -1.846599E-11
 -8.681691E-13  3.500217E-02  1.245327E-11
 -1.846564E-11 -1.245331E-11 -8.286417E-02
    3    3
  3.501599E-02  7.311989E-13 -8.940090E-12
  7.311989E-13  3.501599E-02 -3.206153E-11
 -8.940090E-12 -3.206153E-11  1.042343E+00
###############################################################################
\end{verbatim}

This \file{feff.dym} file is now centered on the C atom and the atoms are sorted correctly.
The associated CO2-C-feff.inp FEFF input file has the following structure
section:
\begin{verbatim}
###############################################################################
POTENTIALS
    0    6    C
    1    8    O

ATOMS
    0.00000    0.00000    0.00000    0    C  0.00000    0
    0.00000    0.00000    1.16937    1    O  1.16937    1
    0.00000    0.00000   -1.16937    1    O  1.16937    2
END
###############################################################################
\end{verbatim}
which is correctly centered and has the same structure as the dym file.


 
 \subsubsection{Example 2: XANES and EXAFS calculation}
 
  A typical FEFF input file that uses ab initio DW factors in a XANES
  calculation looks as follows:
  \begin{verbatim}
* This feff9 input file was generated by dym2feffinp
EDGE      K   1.0
CONTROL          1       1       1       1       1       1
SCF          4.000
XANES        4.000
FMS          6.000

DEBYE    500.0  1073.0  5 feff.dym  6  0  1

POTENTIALS
    0    8    O
    1    1    H

ATOMS
     0.00000     0.00000    0.00000    0    O  0.00000    0
     0.96141    -0.12674    0.00000    1    H  0.96972    1
    -0.12674     0.96141    0.00000    1    H  0.96972    2
END
\end{verbatim}

  This input calculates the XANES O K edge spectrum of a single water
  molecule. If an EXAFS calculation is required, the same DEBYE card parameters
  apply, but the XANES card should be substituted by the EXAFS card. It uses {\it ab
  initio} DW factors at 500K and a dynamical matrix stored in the file
  \file{feff.dym}. The number of Lanczos recursion iterations is set to 6, the type
  of DW calculation is to calculate parallel $\sigma^2$, and it asks that all single
  scattering paths from the absorber be calculated independently in the \module{dmdw}
  module.  When {\feff} reads this \file{feff.inp} file, it will create an intermediate input file \file{dmdw.inp}:
\begin{verbatim}
   6
   1    500.000    500.000
   0
feff.dym
   1
   2   1   0           3.20
\end{verbatim}  
  The first line gives the number of Lanczos iterations. The second
  defines the temperature, in this case a grid with a single point. The third
  defines the type of DW calculation and the fourth the name of the dynamical
  matrix file. The fifth line declares that the input contains a single path
  descriptor, which is included in the next line. This descriptor defines all
  single scattering paths from the absorber that are less than 3.2 Bohr long.

 The dynamical matrix file \file{feff.dym} for this calculation looks like this:
\begin{verbatim}
    1
    3
    8
    1
    1
   15.994915
    1.007825
    1.007825
    0.00000000    0.00000000    0.00000000
    1.81679640   -0.23950080    0.00000240
   -0.23950080    1.81679640    0.00000240
    1    1
  5.398996E-01 -1.171079E-01  5.031484E-07
 -1.171079E-01  5.399060E-01  9.690730E-07
  5.031484E-07  9.690730E-07 -1.841479E-03
    1    2 # next 3 lines omitted here for brevity
    1    3 # and again ...
    2    1
    2    2
    2    3
    3    1
    3    2
    3    3
  5.501607E-02 -7.712230E-02 -1.185055E-07
 -7.712230E-02  5.034998E-01  8.548785E-07
 -1.185055E-07  8.548785E-07 -8.715942E-04
\end{verbatim}

The user must obtain this file using another program, for example, Gaussian or ABINIT.  Given these files \file{feff.inp} and \file{feff.dym}, {\feff} calculates a XANES spectrum including the Debye-Waller factors derived from the given dynamical matrix.

Additional technical information about the calculation of ab initio Debye-Waller factors is provided in Appendix \ref{sec:Append-G-DMDW}.




%%%%%%%%%%%%%%%%%%%%%%%%%%%%%%%%%%%%%%%%%%%%%%%%%%%%%%%%%%%%%%%%%%%%%%%%%%% MPI %%%%%%%%%%%%%

\section{Parallel Calculations}
\label{Sec:Parallel}
Although {\feff} is an efficient code and many spectra can be calculated on a laptop computer, some calculations require more calculation time, e.g. those using large FMS clusters.  To meet the memory and cpu needs of such calculations, {\feff} can be run in parallel.  Running parallel calculations is not trivial.  It requires:
\begin{itemize}
  \item  Suitable hardware, e.g. a multicore computer, or a computing cluster
  \item  An MPI environment.  There are different MPI implementations, and configuring MPI on your system can be non-trivial.  If you will run parallel calculations across several computers (e.g., the nodes of a cluster) then these computers must also share disk access and have passwordless access to each other.  Your system administrator can help set up MPI or can tell you the specifications for your computer.  Unfortunately, it's very hard for us to provide support as we don't know your setup.
  \item {\feff} binaries for your configuration.  Although some versions of {\feff} come with precompiled parallel binaries, it is unlikely that they will work on your system.  You'll probably have to compile {\feff}-mpi yourself.  That requires a suitable FORTRAN90 compiler.
  \end{itemize}
  
{\feff}-MPI scales well even on slow networks because the parallel threads communicate very little.  We simply distribute a grid of energy points over the MPI processor grid and calculate the Green's function for each energy point.  The limitation of this approach is that we cannot use more cores than there are energy points in the grid.  Typically, {\feff} maxes out at $N=64$ or $N=128$.  Speedup will be slightly lower than linear because of sequential tasks and the efficiency of distributing the energy points.

Several parallel threads may write identical output to the screen or log file.  This may cause the standard output to be less readable.

\subsection{An easier alternative}
If you do not have access to a parallel computer with MPI configuration, or if the software configuration seems daunting to you, you may be interested in our \htmlref{Cloud Computing platform}{Sec:Cloud}.  This platform is incorporated in JFEFF for easy access and does automated parallel calculations.  It incurs modest per-hour charges.

\subsection{MPI calculations on the command line}
  
    Although the precise syntax varies with software and hardware environment, a command like
\begin{verbatim}
prompt> mpirun -NP 40 --hosts n1,n2,n3,n4,n5 --hostfile /home/me/stuff/my.hostfile 
/home/me/feff90/bin/MPI/fms
\end{verbatim}
will execute \module{fms} in parallel using 40 threads.  Your {\feff} installation includes a simple \file{feffmpi} script that contains the above instruction for all consecutive {\feff} modules.  Or you can copy it here:

\begin{verbatim}
#!/bin/bash
# Adjust this line: it must point to a directory containing the MPI FEFF modules
FeffPath=/home/jorissen/feff90/bin/MPI
# Adjust this line to meet your system configuration: "
MPICommand="mpirun -n 12 --host n20,n20,n20,n20,n17,n17,n17,n17,n18,n18,n18,n18 "
# In this example, we are using 12 parallel threads on 3 cluster nodes 
# (n17, n18, and n20) with 4 (or more) cores each.
# Other common mpirun options are: --hostfile ; --nolocal ; etc.
# The calculation will be (a little less than) 12 times faster than a non-parallel 
# calculation on the same computer.

# There should be no need to edit the following lines:
$MPICommand $FeffPath/rdinp
$MPICommand $FeffPath/dmdw
$MPICommand $FeffPath/atomic
$MPICommand $FeffPath/pot
$MPICommand $FeffPath/screen
$MPICommand $FeffPath/opconsat
$MPICommand $FeffPath/xsph
$MPICommand $FeffPath/fms
$MPICommand $FeffPath/mkgtr
$MPICommand $FeffPath/path
$MPICommand $FeffPath/genfmt
$MPICommand $FeffPath/ff2x
$MPICommand $FeffPath/sfconv
$MPICommand $FeffPath/compton
$MPICommand $FeffPath/eels
$MPICommand $FeffPath/ldos
\end{verbatim}

On some computers you can omit the hosts and hostfile options (e.g. when running on a multicore desktop).  On other computers that are shared among many users you may have to interact with a queuing system.  This may involve writing a small script that determines node availability on the fly and sets the hosts list.  A simpler example is shown here:

\begin{verbatim}
#!/bin/csh

# This is an example for submitting a parallel FEFF job to a PBS queue.
# Details may depend on your configuration.

# Typically, you save this file as "feff.pbs.sh" in the pwd containing feff.inp .
# Then submit as "qsub feff.pbs.sh".
# You can monitor the job using "qstat".
# But find out the specifics of your configuration before submitting.
# E.g. oftentimes you have to specify maximum runtime.

# Configuring the PBS run:
# The lines starting with '#PBS' are NOT comments - don't delete them!
# Here, we are using 4 nodes with 16 cores each, or 64 cores total.
# The job name is "feff9-Cu" (look for this in the queue monitor).
# Stdout/Stderr will go to the files specified below.
# (You may want to purge these files if you run FEFF several times.)

#PBS -l nodes=4:ppn=16
#PBS -N feff9-Cu
#PBS -o feff.out
#PBS -e feff.err
#PBS -q batch
#PBS -V

# Adjust this number to = nodes x ppn, e.g. 4 x 16 here:
set NP=64
# Adjust next line to point to the location of the FEFF MPI executables: 
set FEFF_HOME=/home/me/feff90/bin/MPI

# No need to edit below:
cd $PBS_O_WORKDIR
mpirun -np $NP --hostfile $PBS_NODEFILE $FEFF_HOME/rdinp 
mpirun -np $NP --hostfile $PBS_NODEFILE $FEFF_HOME/atomic 
mpirun -np $NP --hostfile $PBS_NODEFILE $FEFF_HOME/dmdw 
mpirun -np $NP --hostfile $PBS_NODEFILE $FEFF_HOME/pot  
mpirun -np $NP --hostfile $PBS_NODEFILE $FEFF_HOME/ldos  
mpirun -np $NP --hostfile $PBS_NODEFILE $FEFF_HOME/screen 
mpirun -np $NP --hostfile $PBS_NODEFILE $FEFF_HOME/opconsat 
mpirun -np $NP --hostfile $PBS_NODEFILE $FEFF_HOME/xsph 
mpirun -np $NP --hostfile $PBS_NODEFILE $FEFF_HOME/fms   
mpirun -np $NP --hostfile $PBS_NODEFILE $FEFF_HOME/mkgtr 
mpirun -np $NP --hostfile $PBS_NODEFILE $FEFF_HOME/path  
mpirun -np $NP --hostfile $PBS_NODEFILE $FEFF_HOME/genfmt 
mpirun -np $NP --hostfile $PBS_NODEFILE $FEFF_HOME/ff2x  
mpirun -np $NP --hostfile $PBS_NODEFILE $FEFF_HOME/sfconv 
mpirun -np $NP --hostfile $PBS_NODEFILE $FEFF_HOME/compton 
mpirun -np $NP --hostfile $PBS_NODEFILE $FEFF_HOME/eels 
\end{verbatim}


More information can be found in Sec. \ref{sec:Append-B-Inst}.  If you are unable to figure out the particularities of your local setup, we ask that you consult your system administrator or system documentation for help first.

\subsection{MPI calculations in JFEFF}

The JFEFF GUI also starts parallel calculations.  It is still up to you to fulfill the software requirements outlined above.  JFEFF may install MPI-FEFF binaries to your computer, but it is very likely that you will have to compile your own instead.

Use the \file{Settings} window to specify the relevant parameters, in particular number of threads (NP), host list, and location of the FEFF-MPI binaries.

\begin{figure}[H]
	\centering
		\includegraphics[height=4.0in]{settingsMPIlocal.png}
	\label{fig:settingsMPIlocal}
\end{figure}

Start the calculation by clicking \file{Save \& Run}.

On a few occasions we have received errors such as  $sh: mpirun not found $.  These were caused by \module{mpirun} not being found by the shell through which JFEFF executes the program.  This shell does not read user configuration files such as \file{.bashrc}.  \module{mpirun} must be available through some global path (e.g. \file{/usr/bin}), or the explicit path can be given in the \file{Settings}.


\subsection{Remote MPI calculations in JFEFF}

JFEFF can dispatch calculations to a remote machine.  E.g. you can open JFEFF on your laptop and use it to start a {\feff} calculation on your desktop work computer, or on a cluster in a different country.  The remote machine must be available via the SSH protocol.

You must first set up the remote configuration.  In \file{Settings - SSH} proceed as in the figure below.

\begin{figure}[H]
	\centering
		\includegraphics[height=4.0in]{settingsMPIssh.png}
	\label{fig:settingsMPIssh}
\end{figure}

Once all fields are completed, press \file{add}.  You can enter several different configurations for JFEFF to remember.  Select the appropriate one before starting a calculation.  When all the information is entered, click \file{OK} to save your settings.  In the main JFEFF window drag the Run slider to \file{ssh} and click \file{Save \& Run} to start the calculation.  JFEFF will ask you to enter a password to access the remote machine (unless you have configured passwordless access through a SSH keyfile).  JFEFF does not save this password beyond the current JFEFF session.

Note that the same setup can be used to start serial (non-parallel) calculations on the remote machine by selecting the \file{Single} radio button.

Note that you must have {\feff} installed on the remote machine!  

%%%%%%%%%%%%%%%%%%%%%%%%%%%%%%%%%%%%%%%%%%%%%%%%%%%%%%%%%%%%%%%%%%%%%%%%%%% CLOUD COMPUTING %%%%%%%%%%%%%

\section{Cloud Computing}
\label{Sec:Cloud}

JFEFF can run {\feff} calculations in the EC2 cloud.  JFEFF will then send a request to EC2 to prepare a high-performance cloud cluster containing a specified number of nodes.  The {\feff} calculation is run on the cloud cluster and the results are copied back to the user's local machine when finished.  This way of performing calculations, which we call Scientific Cloud Computing, is an alternative to traditional parallel computing.  It allows one to run demanding calculations that would traditionally require expensive and hard-to-maintain local clusters.  

Note that it takes a short time (typically 3-4 minutes) to prepare the cloud cluster.  Therefore, it makes no sense to use Scientific Cloud Computing for very short calculations.

Cloud computing is not a free service.  EC2 charges its users per hour and per node (currently about USD \$0.08 per core-hour).  The user needs to set up a customer account with Amazon Web Services (AWS).  This is very easy but requires a credit card.  EC2 usage charges are then billed to this account.  We have found that these charges are typically reasonable and cheaper than buying one's own hardware unless one has a constant demand for computational power.  

In the JFEFF Settings window you can specify the credentials identifying your AWS account.  Once this is done, deploying a calculation to the EC2 cloud is as easy as setting the slider on the main panel to "cloud" and pressing "Save \& Run".  It takes a few minutes to set up the cloud cluster, plus calculation time depending on the calculation you request and the number of nodes you make available.  Typically, we recommend running the {\feff} code on 1-64 threads.  Note that in the current implementation the cloud cluster will be terminated when the calculation ends, to avoid being charged for a cluster you accidentally forget to terminate.  However, this means that if you start many short calculations on the cloud that each take only a few minutes, each will be billed for an hour of EC2 time.  

All the {\feff} output files will automatically be copied back to your own computer when the calculation finishes.  There is no need for you to log in to the cloud cluster, or to use the command line terminal on your own computer.

To give you the opportunity to try out Scientific Cloud Computing before you set up your own AWS account, JFEFF installs a "Demo Mode".  This means that you can run a cloud calculation "on the house".  You don't need to do anything to activate the Demo Mode.  If you simply put the slider of the "Run" panel to "Cloud", it will automatically launch using demo mode if you haven't yet configured your own account.  Please don't abuse this privilege - all these calculations are billed to the {\feff} project and we are not able to fund your research project.  Also, such calculations are limited to using a maximum of 8 threads. 

If you require any help with this new development, please contact us.  We are interested in user feedback.


%%%%%%%%%%%%%%%%%%%%%%%%%%%%%%%%%%%%%%%%%%%%%%%%%%%%%%%%%%%%%%%%%%%%%%%%%%% MPSE %%%%%%%%%%%%%

\section{Using the Many-Pole Self-Energy}
\label{Sec:MPSE}
FIX give definition of SE.

To use the many-pole model self-energy within FEFF, you must obtain an
estimate of the loss function $L(\omega) = {\rm
  Im}[\epsilon^{-1}(\omega)]$. The loss function can either be
calculated, or measured experimentally. A rough estimate can be
calculated very quickly with the feff code using the
\htmlref{OPCONS}{card:opc} 
card, along with (optionally) the \htmlref{NUMDENS}{card:num} card.
For more accurate calculations of the loss function, one can use codes
based on the Beth-Salpeter equation or time-dependent density
functional theory. Several codes can perform these calculations:
\begin{itemize} \tightlist
  \item{AI2NBSE}
  \item{YAMBO}
  \item{WIEN2K}
  \item{QUANTUM ESPRESSO}
  \item ...
\end{itemize}

\subsection{Examples}
An input file that uses OPCONS to calculate the loss function
and MPSE/SFCONV to calculate the resulting many-pole self-energy and
many-body spectral function for a Cu K-edge XANES calculation
follows:

\begin{verbatim}
TITLE Cu crystal
* COREHOLE treatment RPA or FSR or NONE
COREHOLE RPA

*        pot  xsph fms path genfmt ff2x
CONTROL  1   1   1    1    1      1

* Use the atomic database to form the loss function
OPCONS

* Use many-pole self-energy with density dependence
MPSE 2

* Spectral function convolution
SFCONV

* This specifies HL exchange for both fine structure
* and background, but in this case, the many-pole model
* will be used, since the MPSE card is present.
EXCHANGE        0       0       0       0

* Full multiple scattering with a cluster of 8.0 angstroms.
FMS 9.5

* Shift the fermi-level back 1.5 eV and add 0.3 eV of experimental broadening.
CORRECTIONS 1.5 0.3

* Self consistent potentials with cluster size of 5.0 angstroms.
SCF     7.0

* Calculate XANES (changes energy grid used)
XANES 5.

* Calculate Debye-Waller factors for a temp of 10 K and Debye temp of
* 315 K using correlated Debye model.
DEBYE  10  315    0

* Cu is fcc, lattice parameter a=3.61 (Kittel)

POTENTIALS
  0  29 Cu  3 3 0.01
  1  29 Cu  3 3 1.0

ATOMS
   0.0000     0.0000     0.0000    0   Cu_00            0.0000
   0.0000    -1.8050     1.8050    1   Cu_01            2.5527
   1.8050     0.0000    -1.8050    1   Cu_01            2.5527
  -1.8050     0.0000    -1.8050    1   Cu_01            2.5527
   1.8050    -1.8050     0.0000    1   Cu_01            2.5527
  -1.8050     1.8050     0.0000    1   Cu_01            2.5527
     .
     .
     .
\end{verbatim}

This input file specifies that we want to calculate the Cu K-edge
XANES of bulk Cu using the many-pole model dielectric function to
calculate self-energy effects (MPSE) as well as multi-electron
excitation effects (SFCONV). The file \file{loss.dat} is required input for the many-pole
dielectric function and in this case is obtained by specifying the
OPCONS card. Note that the OPCONS card gives only a rough
estimate of the loss function.

For a more accurate calculation, comment out the "OPCONS" card and provide your own \file{loss.dat} file, e.g. from an experimental measurement or 
an ab initio calculation.  An example of a \file{loss.dat} file can be found in \url{$\sim$/jfeff_examples/MPSE/Cu/}.  You can compare this to the
\file{loss.dat} file that will be generated from the OPCONS card in the example above or at \url{$\sim$/jfeff_examples/MPSE/Cu\_OPCONS/}.  You can also compare the resulting spectra.  (Note that the 2 examples provided have many other differences.  You should compare 2 calculations differing only in the use of the OPCONS card to study its effects.)


\section{Compton scattering}
\label{Sec:Compton}
As of version 9.5.1, {\feff} can calculate Compton scattering.  This is activated by the \htmlref{COMPTON}{card:com} card.  Optionally one can use the \htmlref{CGRID}{card:cgr} card to set the grid, and \htmlref{RHOZZP} {card:rho} card to calculate a z,z' slice of $\rho$.  An example is given below.  Note that calculations of Compton scattering tend to be fairly slow but can be parallellized.  They consume rather a lot of disk space by {\feff} standards (of the order of ~ 100MB) and results in the creation of hundreds of individual files, which can be cumbersome.
 
  First, $J(z,z')$ is calculated.  This is the Fourier transform of $J(p_q)$.   Since this is computationally expensive, it is saved to \file{jzzp.dat}.  The final Compton profile is saved in \file{compton.dat}.

\begin{verbatim}
TITLE Cu crystal fcc, lattice parameter a=3.61 (Kittel)
CONTROL 1 1 1 1 1 1  
SCF     4.0
COREHOLE None
LDOS -30 0 0.1
FMS 4.0
* Calculate the Compton profile:
COMPTON
* Calculate a slice rho(z,z')
RHOZZP
* The grid for calculating rho(r,r')
CGRID 10 32 32 32 120 

POTENTIALS
0 29 Cu 3 3 0.01 
1 29 Cu 3 3 1.0 

ATOMS
  0.00000    0.00000    0.00000    0 Cu_00              0.0000
  0.00000   -1.80500    1.80500    1 Cu_01              2.5527
  1.80500    0.00000   -1.80500    1 Cu_01              2.5527
 -1.80500    0.00000   -1.80500    1 Cu_01              2.5527
  1.80500   -1.80500    0.00000    1 Cu_01              2.5527
* [Truncated]
END
\end{verbatim}


\section{The Mixed Dynamic Form Factor}
\label{sec:MDFF}

Currently available to Developers only.


%%%%%%%%%%%%%%%%%%%%%%%%%%%%%%%%%%%%%%%%%%%%%%%%%%%%%%%%%%%%%%%%%%%%%%%%%%% CARDS %%%%%%%%%%%%%
%%%%%%%%%%%%%%%%%%%%%%%%%%%%%%%%%%%%%%%%%%%%%%%%%%%%%%%%%%%%%%%%%%%%%%%%%%% CARDS %%%%%%%%%%%%%

\chapter{{\feffcur} Control Cards}
\label{sec:Input-Control-Cards}

\section{The master input file \file{feff.inp}}

The {\feff} program consists of a set of program modules driven by  a single input file \file{feff.inp} .  The user can either supply this file
 herself; or she can set all input options through the JFEFF GUI, which then
 creates a {\file{feff.inp}} and launches {\feff} without the need for 
 manual text editing.

This section describes the input . It may be helpful to look at the
sample input files in Section \ref{sec:Calc-Strat-Exampl} while
reading this section.  The current chapter is meant as a reference,
while Section \ref{sec:Calc-Strat-Exampl} makes for friendlier, tutorial-style reading.

The input file \file{feff.inp} is a loosely formatted, line-oriented
text file. Each type of input read by the program is on a line which
starts with a CARD, which is a keyword possibly followed by alpha-numeric data.  The sequence of keyword cards is arbitrary. If any card or optional data is omitted,
default values are used.  An exception is that the user must always specify the crystal structure, which requires one or more of the POTENTIALS, ATOMS, CIF, LATTICE, and OVERLAP cards.

Alpha-numeric values are listed in free format,
separated by blanks. Tab characters are not allowed and may cause confusing error
messages. Any characters appearing after the card and its required or
optional data on a given line are ignored by {\feff} and can be used
as end-of-line comments. Empty lines are ignored. Any
line beginning with an asterisk (\texttt{*}) is regarded as a comment
and is also ignored. 

Users working from the GUI do not need to worry about formatting, and can get more information about any CARD by simply hovering the mouse over it.

All distances are in angstroms ({\AA}) and energies are in electron-volts (eV). 

An auxiliary program ({\atoms}), developed
by Bruce Ravel, can generate the \file{feff.inp} file for crystals from crystallographic 
input parameters. A GUI to {\atoms} is
available at \htmladdnormallink
{http://cars9.uchicago.edu/$\sim$ravel/software/aboutatoms.html}
{http://cars9.uchicago.edu/~ravel/software/aboutatoms.html}

Additionally, {\feffcur} itself can treat crystals specified by their unit cell.



\section{Complete List of FEFF9 Control Cards}
\label{sec:Complete-list-FEFF9}


We divide the \file{feff.inp} options into three categories: 
\begin{itemize}
\item structural information describing the molecule or solid
\item spectrum information specifying what type of spectroscopy to calculate
\item options that determine how {\feff} calculates the required spectrum for the given structure
\end{itemize}

The three main panels of the GUI correspond to these three categories.  Additionally, we can
classify CARDS as \textsl{standard} options frequently and easily used, \textsl{useful}
options that are often used, and \textsl{advanced} options that are
seldom necessary, but may be helpful in some cases.

\begin{description}

  %%
  %% STRUCTURAL INFORMATION
\item[\large\textbf{Structural information}]\dotfill\
  \begin{description}
  \item[\textbf{Purpose:}] Specify the structure
  \item[\textbf{Standard Cards:}] 
    \htmlref{ATOMS}{card:ato}, 
    \htmlref{POTENTIALS}{card:pot}, 
    \htmlref{RECIPROCAL}{card:rec},
    \htmlref{REAL}{card:rea},
    \htmlref{CIF}{card:cif},
    \htmlref{LATTICE}{card:lat},
    \htmlref{TARGET}{card:tar}, 
    and \htmlref{TITLE}{card:tit}
  \item[\textbf{Useful Cards:}] 
    \htmlref{COORDINATES}{card:coo},
    \htmlref{RMULTIPLIER}{card:rmu} and
    \htmlref{SGROUP}{card:sgr}
  \item[\textbf{Advanced Cards:}] 
    \htmlref{CFAVERAGE}{card:cfa},
    \htmlref{OVERLAP}{card:ove} and
    \htmlref{EQUIVALENCE}{card:equ}
  \end{description}

  %%
  %% SPECTRUM INFORMATION
\item[\large\textbf{Spectrum information}]\dotfill\
  \begin{description}
  \item[\textbf{Purpose:}] Specify the spectrum or material property to be calculated
  \item[\textbf{Standard Cards:}] 
    \htmlref{EXAFS}{card:exa},
    \htmlref{ELNES}{card:eln},
    \htmlref{EXELFS}{card:exe},
    \htmlref{LDOS}{card:ldo},
    and \htmlref{XANES}{card:xan}
  \item[\textbf{Useful Cards:}] 
    \htmlref{ELLIPTICITY}{card:ell},
    \htmlref{MULTIPOLE}{card:mul},
    and \htmlref{POLARIZATION}{card:pol} 
  \item[\textbf{Advanced Cards:}] 
    \htmlref{COMPTON}{card:com},
    \htmlref{DANES}{card:dan}, 
    \htmlref{FPRIME}{card:fpr},
    \htmlref{MDFF}{card:mdf}, 
    \htmlref{NRIXS}{card:nri},
    \htmlref{XES}{card:xes},
    and \htmlref{XNCD (or XMCD)}{card:xnc}
  \end{description}

  %%
  %% FEFF PROGRAM CONTROL
  \item[{\large FEFF program control}]  \hfill
  \begin{description}
  \item[\textbf{Purpose:}] Control the way {\feff} works

  %%
  %% GENERAL
\item[{\large general options}]\dotfill\  
  \begin{description}
  \item[\textbf{Purpose of cards:}] general settings
  \item[\textbf{Standard cards:}] 
    \htmlref{CONTROL}{card:con},
    \htmlref{END}{card:end}, 
    \htmlref{KMESH}{card:kme}, 
    and \htmlref{PRINT}{card:pri}
    \item[\textbf{Advanced Cards:}] 
    \htmlref{DIMS}{card:dim} and
    \htmlref{EGRID}{card:egr}
\end{description}
  %%
  %% POT
\item[\large\module{atomic}, \module{pot}, \module{screen}, \module{opconsat}]\dotfill\  
  \begin{description}
  \item[\textbf{Purpose of Module:}] Calculate self-consistent
    scattering potentials and Fermi energy
  \item[\textbf{Standard cards:}] 
    \htmlref{AFOLP}{card:afo}, 
    \htmlref{COREHOLE}{card:chl},
    \htmlref{EDGE}{card:edg},
    \htmlref{SCF}{card:scf}, 
    and \htmlref{S02}{card:s02}
  \item[\textbf{Useful Cards:}] 
    \htmlref{CHBROAD}{card:chb},
    \htmlref{CONFIG}{card:cfg},
    \htmlref{EXCHANGE}{card:exc},
    \htmlref{FOLP}{card:fol},
    \htmlref{HOLE}{card:hol},    
    \htmlref{NOHOLE}{card:noh}, 
    \htmlref{RGRID}{card:rgr},
    and \htmlref{UNFREEZEF}{card:unf}
  \item[\textbf{Advanced Cards:}] 
    \htmlref{CHSHIFT}{card:csh},
    \htmlref{CHWIDTH}{card:chw},
    \htmlref{CORVAL}{card:crv},
    \htmlref{EGAP}{card:ega},
    \htmlref{EPS0}{card:eps},
    \htmlref{EXTPOT}{card:ext}, 
    \htmlref{INTERSTITIAL}{card:int}, 
    \htmlref{ION}{card:ion},
    \htmlref{JUMPRM}{card:jum},
    \htmlref{NUMDENS}{card:num},
    \htmlref{OPCONS}{card:opc},
    \htmlref{PREP}{card:pre},
    \htmlref{RESTART}{card:res},
    \htmlref{SCREEN}{card:scr},
    \htmlref{SETE}{card:set},
    and \htmlref{SPIN}{card:spi} 
  \end{description}
  %%
  %% XSPH
\item[{\large\module{xsph}}]\dotfill\  
  \begin{description}
  \item[\textbf{Purpose of Module:}] Calculate cross-section and phase shifts
  \item[\textbf{Advanced Cards:}] 
    \htmlref{LJMAX}{card:ljm},
    \htmlref{LDEC}{card:lde},
    \htmlref{MPSE}{card:mps},
    \htmlref{PLASMON}{card:mps},
    \htmlref{PMBSE}{card:pmb},
    \htmlref{RPHASES}{card:rph},
    \htmlref{RSIGMA}{card:rsi},
    and \htmlref{TDLDA}{card:tdl}
  \end{description}
  %%
  %% FMS
\item[{\large\module{fms and mkgtr}}]\dotfill\  
  \begin{description}
  \item[\textbf{Purpose of Module:}] Calculate full multiple
    scattering for XANES and ELNES
  \item[\textbf{Standard cards:}] \htmlref{FMS}{card:fms}
  \item[\textbf{Useful Cards:}] \htmlref{DEBYE}{card:deb1}
  \item[\textbf{Advanced Cards:}] \htmlref{BANDSTRUCTURE}{card:ban}
    and \htmlref{STRFACTORS}{card:stf}
  \end{description}
  %%
  %% PATH
\item[{\large\module{path}}]\dotfill\  
  \begin{description}
  \item[\textbf{Purpose of Module:}] Path enumeration
  \item[\textbf{Standard cards:}] \htmlref{RPATH}{card:rpa}
  \item[\textbf{Useful Cards:}] \htmlref{NLEG}{card:nle}
  \item[\textbf{Advanced Cards:}] 
  \htmlref{PCRITERIA}{card:pcr},
  \htmlref{SS}{card:ss}, 
  and \htmlref{SYMMETRY}{card:sym}
  \end{description}
  %%
  %% GENFMT
\item[{\large\module{genfmt}}]\dotfill\  
  \begin{description}
  \item[\textbf{Purpose of Module:}] Calculate scattering amplitudes
    and other XAFS parameters
  \item[\textbf{Useful Cards:}] \htmlref{CRITERIA}{card:cri}
  \item[\textbf{Advanced Cards:}] \htmlref{IORDER}{card:ior} and
    \htmlref{NSTAR}{card:nst}
  \end{description}
  %%
  %% FF2X
\item[{\large\module{ff2x}}]\dotfill\ 
  \begin{description}
  \item[\textbf{Purpose of Module:}] Calculate X-ray spectra. 
  \item[\textbf{Standard cards:}] \htmlref{DEBYE}{card:deb2}
  \item[\textbf{Useful Cards:}] \htmlref{ABSOLUTE}{card:abs},
  \htmlref{CORRECTIONS}{card:cor},
    \htmlref{SIG2}{card:sig}, \htmlref{SIG3}{card:si3} and
    \htmlref{SIGGK}{card:siggk}
  \item[\textbf{Advanced Cards:}] \htmlref{MBCONV}{card:mbc}
  \end{description}
  %%
  %% SFCONV
\item[{\large\module{sfconv}}]\dotfill\ 
  \begin{description}
  \item[\textbf{Purpose of Module:}] Convolve output with spectral function. 
  \item[\textbf{Useful Cards:}] \htmlref{SFCONV}{card:scc}
  \item[\textbf{Advanced Cards:}]     \htmlref{RCONV}{card:rco},
    \htmlref{SELF}{card:sel} and
    \htmlref{SFSE}{card:sfs}
  \end{description}
  %% COMPTON
\item[{\large\module{compton}}]\dotfill\ 
  \begin{description}
  \item[\textbf{Purpose of Module:}] Calculate Compton scattering. 
  \item[\textbf{Useful Cards:}] \htmlref{CGRID}{card:cgr} and
    \htmlref{RHOZZP}{card:rho}
  \end{description}  
  %%
  %% EELS
\item[{\large\module{eels}}]\dotfill\ 
  \begin{description}
  \item[\textbf{Purpose of Module:}] Calculate EELS spectra. 
  \item[\textbf{Useful Cards:}] \htmlref{MAGIC}{card:mag}
  \end{description}

  \end{description}

\end{description}


These CARDS are listed below in the same order as in the table above.
Each CARD description is of this form:

\begin{Card}{CARD}{required arguments [optional arguments]}{type}{}
  The type is one of \textsl{Standard}, \textsl{Useful}, or
  \textsl{Advanced}. The argument list is a brief statement of the
  valid arguments to the card. Arguments in square brackets are 
  optional. The text description explains the arguments and 
  their uses more fully. Example uses of the card look like this:
\begin{verbatim}
  * brief description of the example
  CARD  arguments
\end{verbatim}
\end{Card}


%%%%%%%%%%%%%%%%%%%%%%%%%%%%%%%%%%%%%%%%%%%%%%% STRUCTURE
\section{Structural Information Cards}
\label{sec:Structural-Information-Cards}

The ATOMS card is used to specify the absorbing atom and its environment.
Alternatively, if atomic coordinates are not known, the OVERLAP card can be used 
to construct approximate potentials. Without the structural
information of either ATOMS or OVERLAP card, no calculations can be done. 



\begin{Card}{ATOMS}{}{Standard}{ato}
  ATOMS specifies the Cartesian coordinates (in \AA ngstroms) and unique potential indices
  of each atom in the cluster, one atom per line. If the LATTICE card is used to define the unit cell for reciprocal space calculations, the ATOMS card lists the atoms in the unit cell.
  See the discussion of the \htmlref{POTENTIALS}{card:pot} card and the \htmlref{COORDINATES}{card:coo} card for more info on how to specify atom types and coordinates. 

  An auxiliary code, \htmlref{\atoms}{http://cars9.uchicago.edu/~ravel/}, can generate 
  a real-space ATOMS list for crystals from crystallographic data.

\begin{verbatim}
  * A real-space example : the SF6 molecule
  * Coordinates in Angstrom
   ATOMS
   * x      y      z     ipot     SF6 molecule
    0.0     0.0     0.0      0      S K-shell hole
    1.56   0.00   0.00    1      F 1st shell atoms
    0.00   1.56   0.00    1
    0.00   0.00   1.56    1
   -1.56   0.00   0.00    1
    0.00  -1.56   0.00    1
    0.00   0.00  -1.56    1
    
  * A reciprocal-space example : the hexagonal (wurtzite) GaN crystal
  * See the LATTICE card for the corresponding lattice vectors
  * Coordinates are expressed as fractions of the lattice vectors
  * Note that CIF files are now the preferred way to input crystals by their unit cell.
   ATOMS                          * this unit cell contains 4 atoms
   *   x                             y                              z                     ipot  tag           
    0.288675130000    0.500000000000    0.000000000000  1    Ga
    0.577350270000    0.000000000000    0.811850000000  1    Ga
    0.288675130000    0.500000000000    0.609700000000  2    N
    0.577350270000    0.000000000000    1.414242700000  2    N

\end{verbatim}
\end{Card}


\begin{Card}{CIF}{cif\_file}{Standard}{cif}
This card specifies the name of a file containing the structural information in CIF format.  CIF is incompatible with ATOMS, LATTICE, SGROUP, and COORDINATES.  POTENTIALS is not required with CIF, as the potential types will be generated automatically.  However, POTENTIALS can be used to set the details (angular momentum cutoffs and spin variables) of the potentials.  See (K-space FEFF) for more information.
While both relative and absolute paths are supported, we recommend copying the \file{.cif} file to the working directory to avoid mistakes and confusion.
CIF currently requires that RECIPROCAL, TARGET, and KMESH are also set.  Note that TARGET in conjunction with CIF refers to the list of atoms as given in the \file{.cif} file (i.e., a list of the crystallographically inequivalent atom positions in the unit cell).
\begin{verbatim}
  * the file GaN.cif defines the unit cell of hexagonal (wurtzite) GaN :
   CIF  GaN.cif
\end{verbatim}
\end{Card}


\begin{Card}{LATTICE}{type scale}{Standard}{lat}
This card specifies the lattice.  First, its type must be specified using a single letter : P for primitive, F for face centered cubic, I for body centered cubic, H for hexagonal.  The following three lines give the three basis vectors in Carthesian Angstrom coordinates.  They are multiplied by scale (e.g., 0.529177 to convert from bohr to Angstrom).
\begin{verbatim}
  * the unit cell of hexagonal (wurtzite) GaN :
   LATTICE P   3.18800
      0.86603     -0.50000      0.00000              * ax ay az
      0.00000      1.00000      0.00000              * bx by bz
      0.00000      0.00000      1.62359              * cx cy cz
\end{verbatim}
\end{Card}



\begin{Card}{POTENTIALS}{ipot   Z   [tag   lmax1   lmax2  xnatph  spinph]}{Standard}{pot}
  The POTENTIALS card is followed by a list which assigns a unique
  potential index to each distinguishable atom. The potential index
  \texttt{ipot} is the index of the potential to be used for the phase shift
  calculation.
  
  The required list entries are the unique potential index
  \texttt{ipot} and the atomic number \texttt{Z}.  The \texttt{tag} is
  at most 6 characters and is used to identify the unique potential.  
  
  The optional list entries 
  \texttt{lmax1} and \texttt{lmax2} are used to limit the angular
  momentum bases of the self-consistent potentials (SCF) and full
  multiple scattering calculations (FMS).  If a negative number (e.g.,
  $\mathtt{lmax1}=-1$) is specified for either \texttt{lmax1} or
  \texttt{lmax2}, {\feff} will automatically use a default based on atomic number.  

  The next optional entry, \texttt{xnatph}, can be used to specify the
  stoichiometric number of each unique potential in the unit cell of a
  crystalline material. This helps in the calculation of the Fermi
  level. In the case of an infinite solid, $\mathtt{xnatph}=0.01$
  (default value) is a suitable value for the absorbing atom. For
  materials finite in extent, see the example below.

  The last optional entry \texttt{spinph} is used to specify 
  the spin amplitude and relative spin alignment for spin-dependent 
  calculations. See the \htmlref{SPIN}{card:spi} card in this section
  for more information on spin-dependent calculations.  

  The absorbing atom must be given unique potential index 0.  
  Unique potential indices are simply labels, so the order is 
  not important, except that the absorbing atom is index 0, and you may not
  have missing indices (i.e., if you use index 3, you must also have
  defined unique potentials 1 and 2).  Exception: If the material is defined by its unit cell the absorbing atom is not defined
  through the POTENTIALS card but in the \htmlref{TARGET}{card:mar} instead - in this case,
  there should not be a unique potential index 0 in the POTENTIALS card.  There is another 
  exception to the rule when the \htmlref{CFAVERAGE}{card:cfa} card is 
  used.   

  To save time, the code calculates the overlapped atom potential for
  each unique potential only once, using the first atom with a given unique 
  potential index.  For example, in Cu it is a good
  approximation to determine potentials only for the central atom and
  the first shell and to use the first shell potential
  ($\mathtt{ipot}=1$) for all higher shells.  Such approximations should
  always be checked.  If the neighborhood of the sample atom is not
  representative, the code will generate inaccurate potentials 
  and phase shifts, as well as poor XAS results.  Atoms of the same Z may have different potentials
  if their local environment is substantially different.  {\feff} leaves such judgments to the user.
  One can always make two atoms inequivalent, and compare their potentials, electronic configutations,
  and lDOS in the {\feff}output.

\begin{verbatim}
  * molecular SF6  Sulfur K edge, lamx1=default, lmax2=3 (spdf basis)
  POTENTIALS
  *   ipot     Z  tag  lmax1 lmax2  xnatph
       0      16   S    -1     3     1
       1       9   F    -1     3     6
\end{verbatim}
An example of spin-dependent usage can be found in the SPIN card description
in the next section.
\end{Card}



\begin{Card}{REAL}{}{Standard}{rea}
  This card tells {\feff} to work in real space.  This is the default mode, so it is never strictly necessary to use this card.  
  This card may be used for clarity in input files that mix
real-space and k-space calculations. See also the RECIPROCAL card.
\begin{verbatim}
  * do a real-space calculation
  REAL
\end{verbatim}
\end{Card}



\begin{Card}{RECIPROCAL}{}{Standard}{rec}
  This card tells {\feff} to work in reciprocal space.  It affects modules pot, xsph, fms and ldos.  This card requires
  the use of CIF or ATOMS and LATTICE; TARGET; and KMESH.
\begin{verbatim}
  * do a k-space calculation of a crystal
  RECIPROCAL
\end{verbatim}
\end{Card}



\begin{Card}{TARGET}{ic}{Standard}{tar}
  Specifies the location of the absorber atom for reciprocal space calculations.  It is entry {\it ic} of the ATOMS card if an ATOMS card and LATTICE card are used.  In conjunction with the CIF card it is entry {\it ic} the list of atoms as given in the \file{.cif} file (i.e., a list of the crystallographically inequivalent atom positions in the unit cell).  The target needs to be specified also for NOHOLE
  calculations.  Note that this cannot be specified in the POTENTIALS list because periodic boundary conditions would then produce an infinite number of core holes.
\begin{verbatim}
  * calculate a spectrum for the second atom in the ATOMS list or CIF file.
  TARGET 2
\end{verbatim}
\end{Card}



\begin{Card}{TITLE}{any\_descriptive\_text}{Standard}{tit}
  User supplied title lines. You may have up to 10 of these. Titles
  may have up to 75 characters. Leading blanks in the titles will be
  removed.
\begin{verbatim}
  TITLE  Andradite  (Novak and Gibbs, Am.Mineral 56,791 1971)
  TITLE  K-shell 300K
\end{verbatim}
\end{Card}



\begin{Card}{COORDINATES}{i}{Useful}{coo}
  i must be an integer from 1 through 6.  It specifies the units of the atoms of the unit cell given in the ATOMS card for reciprocal space calculations.    
  If the card is omitted, the default value $icoord=3$ is assumed. FIX check this
  \begin{enumerate}
  \item Cartesian coordinates, Angstrom units.  Like {\feff} - you can copy from a real-space feff.inp file if your lattice vectors coincide with atoms in that feff.inp file.
  \item Cartesian coordinates, fractional units (i.e., fractions of the lattice vectors ; should be numbers between 0 and 1).  Similar to {\feff}.
  \item Cartesian coordinates, units are fractional with respect to FIRST lattice vector.  Like SPRKKR. \emph{(default)}
  \item Given in lattice coordinates, in fractional units.  Like WIEN2k (but beware of some `funny' lattice types, e.g. rhombohedral, in WIEN2k case.struct if you're copy-pasting )
  \item Given in lattice coordinates, units are fractional with respect to FIRST lattice vector.
  \item Given in lattice coordinates, Angstrom units.
  \end{enumerate}
\begin{verbatim}
  * Say that a diamond lattice has been defined as :
  LATTICE P 6.0
  0.0	0.5 0.5
  0.5 0.0 0.5
  0.5 0.5 0.0
  * Now the atoms can be entered as :
  ATOMS
  0.0 0.0 0.0
  1.5 1.5 1.5
  COORDINATES 1      * identical to 6 for this example
  * Or another way is :
  ATOMS
  0.0 0.0 0.0
  0.25 0.25 0.25
  COORDINATES 2      * identical to 3, 4, and 5 for this example
\end{verbatim}
\end{Card}



\begin{Card}{RMULTIPLIER}{rmult}{Useful}{rmu}
  With RMULTIPLIER all atomic coordinates are multiplied by the
  supplied value. This is useful to adjust lattice spacing, for
  example, when fractional unit cell coordinates are used. By
  default, \texttt{rmult}=1.
\begin{verbatim}
  *increase distances by 1%
  RMULTIPLIER 1.01
\end{verbatim}
\end{Card}



\begin{Card}{SGROUP}{igroup}{Useful}{sgr}
  This card specifies the space group of the crystal (number from 1 through 230).  Currently not used and informative only.
\begin{verbatim}
  * simple primitive cell
  SGROUP 1
\end{verbatim}
\end{Card}



\begin{Card}{CFAVERAGE}{iphabs nabs rclabs}{Advanced}{cfa}
  A ``configuration'' average over the spectra of multiple 
  absorbing atoms is done if the CFAVERAGE card is used. 
  CFAVERAGE currently assumes phase transferability, which 
  is usually good for EXAFS calculations, but may not be 
  accurate for XANES. Note that the CFAVERAGE card is currently
  unreliable in general, and in particular is  
  incompatible with the \htmlref{DEBYE}{card:deb2} card for 
  options other than the correlated Debye model ($\mathtt{idwopt} > 0$).
  \begin{description}
  \item[\texttt{iphabs}]\hfill\\ potential index for the type of absorbing atoms  
    over which to make the configuration average (any potential index is allowed).
  \item[\texttt{nabs}]\hfill\\ the configuration average is made over
    the first \texttt{nabs} absorbers in the \file{feff.inp} file of
    type \texttt{iphabs}. You do not need to have potential of index 0
    in your input file when using the CFAVERAGE card, but you must
    have the same type of potential for iph=0 and iph=iphabs. The
    configurational average is done over ALL atoms of type
    \texttt{iphabs}, if \texttt{nabs} is less than or equal to zero.
  \item[\texttt{rclabs}]\hfill\\ radius to make a small atom list from a
    bigger one allowed in \file{feff.inp}. Currently the parameter
    controlling the maximum size of the list, \texttt{natxx}, is set to
    100,000, but this can be increased. The pathfinder will choke on too
    big an atoms list. You must choose \texttt{rclabs} to have fewer
    than 1,000 atoms in the small atom list. If your cluster has fewer than 1,000
    atoms simply use \texttt{rclabs}=0 or negative always to include
    all atoms.
  \end{description}
  Default values are \texttt{iphabs}=0, \texttt{nabs}=1,
  \texttt{rclabs}=0 (where $\mathtt{rclabs}=0$ means to consider an
  infinite cluster size).
\begin{verbatim}
  *average over all atoms with iph=2 in feff.inp
  CFAVERAGE 2  0  0
\end{verbatim}
\end{Card}
FIX check that there's a more substantial example of this somewhere and put a link to it


\begin{Card}{OVERLAP}{iph}{Advanced}{ove}
  The OVERLAP card can be used to construct approximate overlapped
  atom potentials when atomic coordinates are not known or specified.
  If the atomic positions are listed following the \htmlref{ATOMS}{card:ato} 
  card, the OVERLAP card is not needed. {\feffcur} will stop if both 
  the ATOMS and OVERLAP cards are used. The OVERLAP card contains the 
  potential index of the atom being overlapped and is followed by a list
  specifying the potential index, number of atoms of a given type to
  be overlapped and their distance to the atom being overlapped. The
  examples below demonstrate the use of an OVERLAP list. This option
  can be useful for initial single scattering XAFS calculations in
  complex materials where very little is known about the structure.

  You should verify that the coordination chemistry built in using the
  OVERLAP cards is realistic. It is particularly important to specify
  all the nearest neighbors of a typical atom in the shell to be
  overlapped. The most important factor in determining the scattering
  amplitudes is the atomic number of the scatterer, but the coordination 
  chemistry should be approximately correct to ensure good scattering 
  potentials. Thus it is important to specify as accurately as possible 
  the coordination environment of the scatterer. Note: If you use the 
  OVERLAP card, you cannot use the \htmlref{FMS}{card:fms} or 
  \htmlref{SCF}{card:scf} cards. Also the pathfinder won't be called 
  and you must explicitly specify single scattering paths using the 
  \htmlref{SS}{card:ss} card
  \begin{latexonly}
   , which is described in Section~\ref{sec:Path-enum-modul}
  \end{latexonly}.

\begin{verbatim}
  * Example 1. Simple usage
  * Determine approximate overlap for central and 1st nearest neighbor in Cu
  OVERLAP 0         determine overlap for central atom of Cu
    *iphovr   novr   rovr       * ipot, number in shell, distance
     1        12     2.55266
  OVERLAP 1         determine approximate overlap for 1st shell atoms
    *iphovr   novr   rovr       * ipot, number in shell, distance
     0        12     2.55266

  * Example 2. More precise usage
  * Determine approximate overlap for 3rd shell atoms of Cu
  OVERLAP 3
    0  1 2.55266    ipot, number in shell, distance
    1  4 2.55266
    2  7 2.55266
    2  6 3.61000
    2 24 4.42133
\end{verbatim}
\end{Card}


\begin{Card}{EQUIVALENCE}{ieq}{Advanced}{equ}
This optional card is only active in combination with the \htmlref{CIF}{card:cif} card.  It tells {\feff} how to generate
potential types from the list of atom positions in the \file{cif} file. 

If $ieq=1$, the crystallographic equivalence as expressed
in the \file{cif} file is respected; that is, every separate line containing a generating atom position will lead to a separate potential type.
This means that, e.g., in HOPG graphite, the two generating positions will give rise to two independent C potentials.  This is also the default behavior
if the EQUIVALENCE card is not specified.

If $ieq=2$, unique potentials are assigned based on atomic number Z only.  That is, all C atoms will share a C potential and so on.  This is how most {\feff} calculations
are run.  Whether it is sensible or not to do this depends on the system and on the property one wishes to calculate.  Keep in mind that {\feff} is a muffin tin code, and may
therefore be indifferent to certain differences between crystallographically inequivalent sites.  On the other hand, if an element occurs in the crystal with different oxidation states,
it may be necessary to assign separate potentials to these different types in order to describe the crystal properly and get accurate spectra.

If $ieq=3$, unique potentials are assigned based on atomic number Z and the first shell.  This can be useful e.g. to treat larger systems with crystal defects, where only first neighbors of the
defect need to be treated differently from all more distant atoms of a certain Z.  (To be implemented.)

If $ieq=4$, a hybrid of methods $1$ and $2$ is used.  That is, if the number of unique crystallographic positions does not exceed a hard-coded limit (nphx=9 in the current version), they are treated
with the correct crystallographic equivalence.  If the number of unique crystallographically inequivalent sites is larger, they get combined by atomic number Z.  This ad hoc approach is a practical way of simply limiting 
the number of unique potentials.  This makes sense because, first of all, there are certain hardcoded limits that would require recompilation of the code, requiring more RAM memory and more work than a user may want to do.
Secondly, our SCF algorithm tends to have a harder time reaching convergence as the number of potentials increases, leading to substantially longer calculation times or even convergence failure if the number of potentials 
becomes very large.

If $ieq=5$, unique potentials are assigned based on a label in the \file{cif} file.  That is, the user can edit the \file{cif} file in a text editor and mark different sites with labels such as "Ti1" and "Ti2".  {\feff} will
assign the same unique potential to all sites with the same label.  This gives the user complete control over potential assignment.  (To be implemented.)

If you require one of the solutions marked as "To be implemented", please contact us for assistance.

\begin{verbatim}
* Example : Do a traditional FEFF calculation where all atoms with the same Z
*            have the same potential
CIF graphite.cif
EQUIVALENCE 2
* This would be equivalent to a file using LATTICE and ATOMS card, and
* POTENTIALS
**       ipot    z       label   lmax1   lmax2
*        0       6       C       -1       -1     * for the core hole atom  
*        1       6       C       -1       -1     * for all other C atoms

* Example 2 : Do a calculation with true crystallographic equivalence,
*              as most bandstructure codes do:
CIF graphite.cif
EQUIVALENCE 1  * This is the default and could be omitted for the same results
* This would be equivalent to a file using LATTICE and ATOMS card, and
* POTENTIALS
**       ipot    z       label   lmax1   lmax2
*        0       6       C       -1       -1     * for the core hole atom 
*        1       6       C       -1       -1     * for half of the C atoms
*        2       6       C       -1       -1     * for the other half of the C atoms
\end{verbatim}
\end{Card}



%%%%%%%%%%%%%%%%%%%%%%%%%%%%%%%%%%%%%%%%%%%%%%% SPECTRUM
\section{Spectrum Information Cards}
\label{sec:Spectrum-Information-Cards}

These cards tell {\feff} which material properties to calculate.  In general, one can choose only one spectroscopy card
(EXAFS, XANES, DANES, XMCD, ELNES, EXELFS, FPRIME, NRIXS, XES).  To calculate a second type of spectrum, a new {\feff} calculation
is generally required, although part of the previous calculation may be reused.  The LDOS can be combined with any spectroscopy.  The NRIXS card must
be combined with either EXAFS or XANES - the output will be a NRIXS spectrum but the other card tells {\feff} how to calculate it.
The cards ELLIPTICITY, POLARIZATION and MULTIPOLE may be combined with certain spectroscopy cards.



\begin{Card}{EXAFS}{[xkmax]}{Standard}{exa}
 EXAFS is the default type of spectroscopy.  As such, the card may in principle be omitted, though it is good practice
 to always explicitly set the spectroscopy type being calculated.  The EXAFS card sets the maximum value of $k$ for EXAFS 
calculations. $k$ is set by \texttt{xkmax}, and the default value is 
20 \AA$^{-1}$. The code can calculate to even higher values, however, 
the user may be prompted to increase compilation time dimension settings. For high $k$ calculations it might be necessary 
to make smaller steps using the \htmlref{RGRID}{card:rgr} card.
\begin{verbatim}
  *make EXAFS calculation up to k=25 Angstroms^-1
  EXAFS 25
\end{verbatim}
\end{Card}



\begin{Card}{ELNES}{[xkmax xkstep vixan]}{Standard}{eln}
\texttt{E [aver [cross [relat]]]}\\% : beam energy in keV; optional parameters
\texttt{kx} \texttt{ky} \texttt{kz}\\% : beam direction in the crystal frame
$\mathtt{\beta}$ $\mathtt{\alpha}$\\% : collection semiangle, convergence semiangle (in mrad)
\texttt{nr} \texttt{na}\\% : q-integration mesh : radial size, angular size FIX make this rubbish optional
\texttt{dx} \texttt{dy}% : position of the detector (x,y angle in mrad)

\begin{description}
  \item[\texttt{xkmax}]\hfill\\ 
    The maximum k-value for the calculation. \texttt{xkmax}, \texttt{xkstep} 
    and \texttt{vixan} are exactly the same parameters as those used for the 
    \htmlref{XANES}{card:xan} card.

  \item[\texttt{xkstep}]\hfill\\ 
    The step size of the upper part of the k-mesh
  \item[\texttt{vixan}]\hfill\\ 
    The step size of the lower part of the k-mesh
  \item[\texttt{E}]\hfill\\ 
    energy of the electron beam in keV (typical values are 100-400 keV);
  \item[\texttt{aver}]\hfill\\ 
    1: calculate orientation averaged spectrum (e.g., a polycrystalline 
    sample, working at the magic angle) ; 0 : use specific sample to beam 
    orientation (default);
  \item[\texttt{cross}]\hfill\\
    1: use cross terms for the cross section (e.g., xy or yx ; default); 0: use 
    only direct terms (eg., atom coordinates entered in symmetric coordinate 
    frame ; assumed as default if aver is set to 1);
  \item[\texttt{relat}]\hfill\\ 
    1 (default): to use relativistic formula for the cross-section (default, 
    always recommended) ; 0 to use nonrelativistic formula;
  \item[\texttt{kx,ky,kz}]\hfill\\ 
    wave vector of the incoming electron in the crystal frame (i.e., the 
    Cartesian coordinate system in which the atom positions of the
    \htmlref{ATOMS}{card:ato} card are given).  In arbitrary units (only the 
    direction, not the size of k is used). This line must be present for 
    orientation-sensitive calculations, and absent for averaged calculations.
  \item[$\mathtt{\beta}$]\hfill\\
    the collection semiangle of the EELS detector in mrad (typical values are of
    the order of 1 mrad);
  \item[$\mathtt{\alpha}$]\hfill\\ 
    The convergence semiangle of the incoming beam in mrad (typical values are of
    the order of 1 mrad); 
  \item[\texttt{nr}, \texttt{na}]\hfill\\
    The cross section is integrated over the values of impulse transfer q allowed by $\alpha$ 
    and $\beta$.  The integration grid consists of nr concentric circles sampling 
    a disc of radius $\alpha$+$\beta$. Circle i contains $\mathtt{na}*(2\mathtt{i}-1)$ 
    points, making for $\mathtt{nr}*\mathtt{nr}*\mathtt{na}$ points total.
    These are nonphysical parameters and should be converged. Typical would be 50, 1; the integration is quite fast.  Only for small values of nr is it 
    necessary to increase \texttt{na} above 1.
  \item[\texttt{dx}, \texttt{dy}]\hfill\\ 
    The position of the detector in the scattering plane, specified by angles 
    in mrad along the x and y axes (the same as used in the ATOMS card) (typical 
    values are 0.0, 0.0)
\end{description}

The line giving beam orientation must be present if and only if
 an oriented spectrum is calculated.

The following example simulates an experiment with a 300 keV beam 
hitting the sample along the y-axis. The detector is set in the 
forward direction and has a 2.4 mrad (semi-)opening; the width of 
the incoming beam is 0 mrad. To do the integration over the detector 
aperture, $5*5*3=75$ points are used. The calculation is relativistic 
and takes sample to beam orientation into account. Default settings 
are used for the energy/k-mesh.

\begin{verbatim}
ELNES # calculate elnes.
300 # beam energy in keV
0 1 0 # beam direction in the crystal frame
2.4 0.0 # collection semiangle, convergence semiangle (in mrad)
5 3 # q-integration mesh : radial size, angular size
0.0 0.0 # position of the detector (x,y angle in mrad)
\end{verbatim}
\end{Card}

\begin{Card}{EXELFS}{xkmax}{Standard}{exe}
\texttt{E [aver [cross [relat]]]}\\
\texttt{kx} \texttt{ky} \texttt{kz}\\
$\mathtt{\beta}$ $\mathtt{\alpha}$ \\
\texttt{nr} \texttt{na} \\
\texttt{dx} \texttt{dy}

See the \htmlref{ELNES}{card:eln} card for a description of these 
parameters.  Note that \texttt{xkmax}, the maximum k value, is the only 
parameter immediately following the EXELFS card.
\end{Card}



\begin{Card}{LDOS}{emin   emax   eimag}{Standard}{ldo}
  To obtain the $\ell$DOS you need only run the \module{rdinp, atomic, pot} modules first 
  to produce the file \file{pot.bin}.
  LDOS is calculated in a separate \module{ldos} module, which runs  
  if the LDOS card appears in \file{feff.inp}.  It uses the cluster cutoff radius \textit{rfms2} specified
  by the FMS card.

  The angular momentum projected density of states is placed by
  default on a standard grid of 101 points. \texttt{emin} and
  \texttt{emax} are the minimum and maximum energies of the $\ell$DOS
  calculation and \texttt{eimag} is the imaginary part of potential
  used in the calculations. This is equivalent to Lorentzian broadening 
  of the $\ell$DOS with half-width \texttt{eimag}. If \texttt{eimag} 
  is negative, the code automatically sets it to be 1/3 of the energy 
  step. The output will be written in \file{ldosNN.dat} files. 
  If 101 points are not
  enough, you can divide the energy range by 2 and run the \module{ldos} module twice.
  The LDOS card is very useful when examining densities of states for
  interpreting XANES or when the self-consistency loop fails or gives
  very strange results. If one calculates the $\ell$DOS of a crystal in real space, it will always
  be broadened due to the effect of finite cluster size.
\begin{verbatim}
  *     emin emax eimag
  LDOS  -20  20   0.2
\end{verbatim}
\end{Card}



\begin{Card}{XANES}{[xkmax xkstep vixan]}{Standard}{xan}
  The XANES card is used when a calculation of the near edge structure
  including the atomic background and absolute energies is desired. All 
  parameters are optional and are used to change the output energy mesh
  for the XANES calculation. The XANES card is normally accompanied by the \htmlref{FMS}{card:fms} card for accurate results.

  The XANES calculation is limited to the (extended) continuum
  spectrum beyond the Fermi level. Thus bound states are not generally
  included; however, in molecules weakly bound states that are below the
  vacuum but above the muffin-tin zero will show up as resonances. The
  absolute energies are based on atomic total energy calculations using 
  the Dirac-Fock-Desclaux atom code. The accuracy of this approximation
  varies from a few eV at low Z to a few hundred eV for very large Z. 

\begin{description}
\item[\texttt{xkmax}]\hfill\\
   The maximum $k$ value of the XANES calculation. If FMS calculations are 
  being made, note that these are not accurate beyond about $k=6$; for 
  larger values of $k$, e.g. $k=20$ with the path expansion, FMS must be 
  turned off. 
\item[\texttt{xkstep}]\hfill\\
  This argument specifies the size of the output $k$ grid far from the edge. 
\item[\texttt{vixan}]\hfill\\
  This argument specifies the energy step of the grid at the edge. 
\end{description}  
  The default values are $\mathtt{xkmax}=8$, $\mathtt{xkstep}=0.07$, and
  $\mathtt{vixan}=0.0$.
%  $\mathtt{vixan}= \gamma_{\mathrm{ch}}/2+\mathtt{vi0}$, where
%  \texttt{vi0} is given by the \htmlref{EXCHANGE}{card:exch} card, described 
%  in Section~\ref{sec:Scatt-potent-modul}.

\begin{verbatim}
  * finer grid for XANES calculation
  XANES  6 0.05 0.3
\end{verbatim}
\end{Card}



\begin{Card}{ELLIPTICITY}{elpty x y z}{Useful}{ell}
  This card is used with the POLARIZATION card 
  \begin{latexonly}
    (see below)
  \end{latexonly}.
  The ellipticity \texttt{elpty} is the ratio of amplitudes of electric
  field in the two orthogonal directions of elliptically polarized
  light. Only the absolute value of the ratio is important for
  nonmagnetic materials. The present code can distinguish left- and
  right-circular polarization only with the XMCD or XNCD cards.
  A zero value of the ellipticity corresponds to linear polarization,
  and unity to circular polarization. The default value is zero.

  \texttt{x}, \texttt{y}, \texttt{z} are coordinates of any nonzero
  vector in the direction of the incident beam. This vector should be
  approximately normal to the polarization vector.
  
  Cannot be used with ELNES, EXELFS, or NRIXS.
\begin{verbatim}
  * Average over linear polarization in the xy-plane
  ELLIPTICITY  1.0  0.0  0.0  -2.0
\end{verbatim}
\end{Card}



\begin{Card}{MULTIPOLE}{le2 [l2lp]}{Useful}{mul}
  Specifies which multipole transitions to include in the calculations.
The options are: only dipole ($\mathtt{le2}=0$, default), dipole and magnetic dipole ($\mathtt{le2}=1$), dipole and quadrupole ($\mathtt{le2}=2$).  This card cannot be used with NRIXS and is not supported with EXELFS and ELNES.

The additional field \texttt{l2lp} can be used to calculate individual dipolar contributions
coming from $L \rightarrow L+1$ ($\mathtt{l2lp}=1$) and from $L \rightarrow L-1$ 
($\mathtt{l2lp}=-1$). Notice that in polarization dependent data there is also a
cross term, which is calculated only when $\mathtt{l2lp}=0$.
\begin{verbatim}
  MULTIPOLE  2   0  *combine dipole and quadrupole transitions.
  MULTIPOLE  0  -1  *calculate dipolar L -> L-1 transitions
\end{verbatim}
\end{Card}



\begin{Card}{POLARIZATION}{x y z}{Useful}{pol}
  This card specifies the direction of the electric field of the
  incident beam, or the main axis of the ellipse, in the case of
  elliptical polarization. \texttt{x}, \texttt{y}, \texttt{z} are the
  coordinates of the nonzero polarization vector. The \htmlref{ELLIPTICITY}{card:ell}
  card is not needed for linear polarization. If the POLARIZATION
  card is omitted, spherically averaged XAFS will be calculated.

  Note that polarization reduces the degeneracy of the paths,
  increasing the calculation time. Choosing polarization in the
  directions of symmetry axes will result in a faster calculation.
  
  Cannot be used with ELNES, EXELFS, or NRIXS.
\begin{verbatim}
  POLARIZATION  1.0  2.5  0.0
\end{verbatim}
\end{Card}


\begin{Card}{COMPTON}{[pqmax npq force-jzzp]}{Advanced}{com}
  To calculate the Compton scattering $J(p_q)$. pqmax is the upper limit on $p_q$.  npq is the number of $p_q$ points.  force-jzzp forces recalculation of intermediate $J(z,z')$.  Since $J(p_q)$ is symmetric, the lower limit on $p_q$ is hardcoded to be 0. 
  First, $J(z,z')$ is calculated.  This is the Fourier transform of $J(p_q)$.   Since this is computationally expensive, it is saved to \file{jzzp.dat}.  The final Compton profile is saved in \file{compton.dat}.
\begin{verbatim}
COMPTON
\end{verbatim}
\end{Card}



\begin{Card}{DANES}{[xkmax xkstep vixan]}{Advanced}{dan}
  To calculate the x-ray scattering amplitude $f'$ instead of absorption $f''$, 
  including solid state effects. Calculates the contribution from the specified
  edge and grid, which is specified as in the \htmlref{XANES}{card:xan} card.
%  This card is still experimental.
\end{Card}
 
 
 
\begin{Card}{FPRIME}{emin emax estep}{Advanced}{fpr}
  To calculate the x-ray scattering factor $f'$ far from the edge
  (only the atomic part). The energy grid is regular in energy with \texttt{estep} 
  between \texttt{emin} and \texttt{emax}. This is typically needed to find 
  the contributions from edges other than those calculated with the DANES card. 
  The total scattering amplitude is $$f'(Q,E) = f_0(Q) + f'(E) +if''(E)$$
  In the dipole approximation $f'$ and $f''$ do not depend on $Q$, but
  this is not true with quadrupole transitions added. This is currently 
  neglected
  and $$f'(E)={\rm DANES(edge)+FPRIME(all\ other\ edges) + (total\ energy\ term\ in\ fpf0.dat)}$$
  $f_0(Q)$ is also tabulated in \file{fpf0.dat};
  $f''$ is printed out by FPRIME and can be used to obtain the total $f'$.
  The total energy correction to $f'$ is given in the first line of
  \file{fpf0.dat} in Cromer-Liberman form, and in the more accurate Kissel-Pratt 
  form. See the references for more details.
\end{Card}


\begin{Card}{MDFF}{imdff [qqmdff cosmdff]}{Advanced}{mdf}
 Experimental feature.  Calculate the Mixed Dynamic Form Factor.  Currently available to Developers only. 
\begin{itemize}   \tightlist
\item			   imdff = 3 : EELS type MDFF calculation selected - summed over all q,q' pairs
\item			   imdff = 2 : NRIXS type MDFF calculation selected - for a single q,q' pair only
\item			   imdff = 1 : NRIXS type MDFF calculation selected - summed over all q,q' pairs
\item  			   imdff < 1 : MDFF calculation disabled
\end{itemize}
 If imdff=1 or =2 and qqmdff and cosmdff are not specified, calculate MDFF(q,q') using q and q' vectors from the NRIXS list of q-vectors (needs to contain at least two vectors).
 If imdff=1 or =2 and qqmdff and cosmdff are specified, calculate MDFF(q,q') using q vectors from the NRIXS list of q-vectors and generating vector q' as having length qqmdff and making an angle cosmdff with the vector q.
 If imdff=3, calculate MDFF(q,q') using the parameters of the ELNES or EXELFS card.  (Output on a grid; or sum if appropriate Bloch wave coefficients are provided.  Not implemented.)
 Note that imdff=3 calculates a dipole-selected MDFF, but does it really rapidly and using relativistic corrections appropriate for EELS.  \file{feff.inp} must also contain the ELNES or EXELFS card.
 Using imdff=1 or imdff=2 calculates the MDFF without a selection rule (or determined using the LJMAX or LDECMX cards), but is slower and does not have relativistic corrections.  Note that the NRIXS routines provide for weights, meaning that Bloch wave type (complex) coefficients can be added to simulate dynamical diffraction, if an external code is used to provide these coefficients.  \file{feff.inp} must also contain the NRIXS card.
 This card is probably too complex and may be butchered in the future.  If you want to use any of this functionality, you'd better be in touch for collaboration, or really like Fortran :-P.
 \end{Card}

\begin{Card}{NRIXS}{nq qx qy qz}{Advanced}{nri}
  Calculate the NRIXS spectrum for given momentum transfer $\vec{q}$.  Currently, $nq$ must be set either to any negative value to calculate
  a spherical average over q-vectors of fixed length $qx$ ($qy$ and $qz$ ignored) ; or set to 1 to calculate
  for the q-vector $qx$ $qy$ $qz$.   $nq > 1$ is implemented experimentally; contact the authors for assistance if needed.  Further options are available through the
  \htmlref{LDEC}{card:lde} and \htmlref{LJMAX}{card:ljm} cards.
\begin{verbatim}
  NRIXS -1 0.5 0.0 0.0     * spherical average
  
  NRIXS  1 0.2 0.2 0.1     * orientation-sensitive 
\end{verbatim}
\end{Card}



\begin{Card}{XES}{emin  emax  estep}{Advanced}{xes}
  To calculate nonresonant x-ray emission spectra (XES) for a specified
  grid. XES may be compared to the occupied DOS.
\end{Card}



\begin{Card}{XMCD or XNCD}{[xkmax xkstep estep]}{Advanced}{xnc}
  Use either of the cards to calculate x-ray circular dichroism
 (the output will contain both magnetic and natural).
 The code calculates XMCD and XNCD from specified edge and grid, specified 
 by auxiliary fields exactly as in the \htmlref{XANES}{card:xan} card.
 
  For nonmagnetic systems only XNCD will be present,
 while for magnetic materials with high symmetry only XMCD is present.
 Both will be present for magnetic materials with low symmetry, and
 x-ray direction (\htmlref{ELLIPTICITY}{card:ell} card) must be used to 
 disentangle the two contributions. The EXAFS region can also be used to 
 determine the position of spins relative to the magnetic field. The XMCD card 
 has to be present in \file{feff.inp} for these calculations. 

  The XNCD originates from cross dipole-quadrupole contributions
 for certain nonmagnetic materials, such as special types of crystals.
 It will change sign for opposite direction of propagation (use the 
 ELLIPTICITY card to do this). We performed calculations  for LiIO$_3$ 
 and found results very similar to previous multiple scattering XNCD 
 calculations. XNCD requires that the XANES card also be used.

  The XMCD (dipolar and quadrupolar) does not change sign under the
 change of direction of x-ray propagation, and is zero for nonmagnetic
 systems. The origin of the effect is that due to spin-orbit coupling,
 the right circular polarized light will produce more electrons
 with spin along or opposite to the direction of x-ray propagation.
 Thus it is important to use spin-dependent calculations for XMCD
 calculations. See the \htmlref{SPIN}{card:spi} card
 \begin{latexonly}
 and Section~\ref{sec:Spin-depend-calc}
 \end{latexonly}
 for more details on spin-dependent calculations and an example for XMCD.

  Note that the XMCD signal will only be contained in the output if the 
 {\feff} code has been compiled with $\mathtt{nspx=2}$. For the (default) 
 value $\mathtt{nspx=1}$, you have to combine data from two \file{xmu.dat}
 files. A simple program to do this, \file{spin.f} is available on the {\feff} 
 web site
 \begin{latexonly}, and printed in Section~\ref{sec:Spin-depend-calc}, 
 where you can also find more details on the signal extraction process
 \end{latexonly}.

\end{Card}



%%%%%%%%%%%%%%%%%%%%%%%%%%%%%%%%%%%%%%%%%%%%%%%%%%%%%%%%%%%%%%%%%%%%%%%%%%%%%%%%%%%%%%%%% PROGRAM CONTROL
\section{FEFF Program Control Cards}
\label{sec:FEFF-Program-Control-Cards}


\subsection{General Cards}
\label{sec:General-Cards}

The cards in this section generally affect the entire {\feff} calculation.
The CONTROL card is used to selectively run parts of {\feff}. The PRINT card controls which
output files are written.



\begin{Card}{CONTROL}{ipot ixsph ifms ipaths igenfmt iff2x}{Standard}{con}
  The CONTROL card lets you run one or more of the {\feff} program modules separately.
  There is a switch for each of six parts of {\feff} : 0 means not to run that module, 1
  means to run it. You can do the whole run in sequence, one module
  at a time, but you \emph{must} run all modules sequentially.
  \emph{Do not skip modules:} \hbox{\texttt{CONTROL 1 1 1 0 0 1}}
  is incorrect. The default is \hbox{\texttt{CONTROL 1 1 1 1 1 1}},
  i.e. run all 6 modules.  {\it ipot} controls modules atomic, pot and screen ; {\it ifms} controls
  modules fms and mkgtr ; and {\it iff2x} controls modules ff2x, sfconv, and eels.  The ldos module
  is not affected by the CONTROL card ; it is activated using the corresponding LDOS card.
\begin{verbatim}

  * example 1
  * calculate self consistent potentials, phase shifts and fms only
  CONTROL  1 1 1 0 0 0   ipot  ixsph  ifms   ipaths  igenfmt  iff2x

  * example 2
  * run paths, genfmt and ff2x; do not run pot, xsph, fms
  * this run assumes previous modules have already been run and
  * adds MS paths between rfms  and rpath to the MS expansion
  CONTROL  0 0 0 1 1 1    ipot  ixsph  ifms   ipaths  igenfmt  iff2x
\end{verbatim}

\end{Card}



\begin{Card}{END}{}{Standard}{end}
  The END card marks the end of the portion of the \file{feff.inp} file
  that {\feff} will read. All data following the END card is ignored.
  Without an END card, the entire input file is read.
\begin{verbatim}
  * ignore any lines in feff.inp that follow this card
  END
\end{verbatim}
\end{Card}



\begin{Card}{KMESH}{nkp(x) [nkpy nkpz [ktype [usesym] ] ]}{Standard}{kme}
This card specifies the mesh of k-vectors used to sample the full Brillouin Zone for the evaluation of Brillouin Zone integrals.  Nkp is the number of points used in the full zone.  It can be specified either as "nkpx nkpy nkpz", "nkp", or "nkp 0 0".  If usesym = 1, the zone is reduced to its irreducible wedge using the symmetry options specified in file symfile, which must be present in the working directory.   The k-mesh is constructed using the tetrahedron method of Bloechl et al., Phys. Rev. B, 1990.  The parameter ktype is meant for time-saving only and means:
\begin{itemize} \tightlist
\item ktype=1  :  regular mesh of nkp points for all modules
\item ktype=2  :  use nkp points for ldos/fms and nkp/5 points for pot  (significant time savings)
\item ktype=3  :  use nkp points for ldos/fms and nkp/5 points for pot (near edge) ; reduce nkp for all modules as we get away from near-edge (somewhat experimental)
\end{itemize}

\begin{verbatim}
  * use a k-mesh of 1000 points in the full BZ for a k-space calculation of a crystal
  KMESH 1000
  * use a k-mesh of 10x5x3 points for a large, irregular cell
  KMESH 10 5 3
  * use a k-mesh of 1000 points and try to save time:
  KMESH 1000 0 0 3
\end{verbatim}
\end{Card}



%% the use of \vspace{-4ex} in PRINT is a crufty hack to avoid having to
%% use the array package
\begin{Card}{PRINT}{ppot pxsph pfms ppaths pgenfmt pff2x}{Standard}{pri}
  The PRINT card determines which output files are printed by each
  module. 
\begin{latexonly}
  See Section~\ref{sec:Input-and-Output-Files} for details about 
  the contents of these files.
\end{latexonly}
  The default is print level 0 for each module.
\begin{verbatim}
  * add crit.dat and feffNNNN.dat files to minimum output
  PRINT  0  0  0  1  0  3
\end{verbatim}
\begin{latexonly}
  The print levels for each module are summarized in
  Table~\ref{tab:printlevels} on page \pageref{tab:printlevels}.
\end{latexonly}
\end{Card}



\begin{Card}{DIMS}{nmax lmax}{Advanced}{dim}
  This card limits the size of arrays so as not to exceed available memory.  nmax is the maximum number of atoms in the cluster
  for FMS matrix inversion.  Lmax is the maximal l-value for the potentials and Green's function.
  {\feffcur} first determines the number of atoms and maximal l-value from user input, i.e. number of atoms given in the ATOMS card,
  the FMS-radius given in the FMS and SCF card, and l-values given in the POTENTIALS card.  In a second step, it will truncate these values using the values of the DIMS card if present.  If the DIMS card is not present, default cutoff values will be loaded from 
  \file{feff90/src/COMMON/m\_dimsmod.f90} - these are fixed at compilation time.
\begin{verbatim}
  * Limit l-values to 2 or lower.  Leave number of atoms alone 
  * (i.e. use compilation time limit for number of atoms).
  * (negative numbers are ignored).
  DIMS -1 2
\end{verbatim}
\end{Card}  



\begin{Card}{EGRID}{}{Advanced}{egr}
This card can be used to customize the energy grid. The EGRID card is followed
by lines specifying the type of grid, minimum and maximum values for
the grid, and the grid step, i.e.
\begin{verbatim}
grid_type grid_min grid_max grid_step
\end{verbatim}
The grid\_type parameter is a string that can take the values
\textit{e\_grid}, \textit{k\_grid}, or \textit{exp\_grid}.
When using the \textit{e\_grid} or \textit{k\_grid} grid types,
grid\_min, grid\_max, and grid\_step are given in $eV$ or $\AA^{-1}$
respectively. For the exp\_grid type, grid\_min and grid\_max are the
minimum and maximum grid values in $eV$, and grid\_step is the
exponential, i.e.
$E_{i} = E_Min + exp(grid\_step*i) - 1.0$.
A fourth grid\_type \textit{user\_grid} is also
available for {\feff} but not in the JFEFF GUI. \textit{user\_grid} is followed by an arbitrary number of
lines, each specifying an energy point in $eV$, i.e.,
\begin{verbatim}
  user_grid
  0.1
  1.5
  3.45
  6.0
  .
  .
  .
\end{verbatim}
Note that the energies are all defined relative to the Fermi energy,
i.e. the edge. 

Up to ten grids can be specified for {\feff} ; up to 5 for JFEFF. The grids can also overlap. If one
is using multiple grid types, a useful parameter \textit{last} can be
used in place of grid\_min, i.e.
\begin{verbatim}
  e_grid -10 10 0.1
  k_grid last 10 0.1
\end{verbatim}
The \textit{last} parameter will use the last point of the previous
grid as grid\_min.
%  For type=1, {\feff} determines its energy grid internally.  If type=2 and options contains a filename, a single-column list of complex
%  energy values will be read from the file, and the spectrum will be calculated on this user-specified energy grid.  If type=3 and options
%  contains 3 real numbers, a logarithmic energy grid will be generated and used to express the spectrum.
%  This card was originally intended to use energy grids from other codes to make comparison easier.  We caution that this card may not be
%  sufficiently tested and problems may occur.
%\begin{verbatim}
%  * read in energy grid from file
%  EGRID 2 my_energy_grid.dat
%\end{verbatim}
\end{Card}



\subsection{ATOMIC, POT, SCREEN : Scattering Potentials}
\label{sec:Scatt-potent-modul}


\begin{Card}{AFOLP}{folpx}{Standard}{afo}
  This automatically overlaps all muffin-tins to a specified
  maximum value (default \texttt{folpx}=1.15) to reduce the effects of
  potential discontinuities at the muffin-tins. Automatic overlapping
  is done by default and is useful in highly inhomogeneous materials.
  Typical values of the overlapping fraction should be between 1.0 and
  1.3. See \htmlref{FOLP}{card:fol} for a non-automated version. 
  Automatic overlapping is done by default; to switch overlapping off, 
  use 1.0 as the argument for AFOLP.
\begin{verbatim}
  * touching muffin-tins; do not use automatic overlapping
  AFOLP  1.0
\end{verbatim}
\end{Card}



\begin{Card}{COREHOLE}{type}{Standard}{chl}
  While the HOLE or EDGE card specifies which edge to calculate, the COREHOLE card determines
  how the core state is treated.  There are three options : $none$, equivalent to the old NOHOLE card, meaning there is no core hole ; $RPA$, meaning the screen module calculates an RPA-screened core hole ; or a simple Final State Rule core hole (default).  It is recommended to use the RPA core hole for k-space calculations.  See \ref{sec:calcpots} for more comments on choosing a core hole.
  
\begin{verbatim}
  * To use the RPA screened core hole :
  COREHOLE RPA
  * To calculate without a core hole :
  COREHOLE none 
  * To use a final state rule (non-screened) core hole :
  COREHOLE FSR   * or omit the card
\end{verbatim}
\end{Card}



\begin{Card}{EDGE}{label s02}{Standard}{edg}
  The EDGE card sets the edge.  Simply use the hole label:
  \texttt{K} means $K$-shell, \texttt{L1} means $L_{I}$, and so on.  
  Calculations with very shallow 
  edges, e.g. $M$-shells and higher, are not well tested; please complain 
  to the authors if you encounter problems. As with the HOLE card, you may 
  also use the integer index instead of the label. If the
  entry for the amplitude reduction factor $S_0^2$ is less than 0.1, $S_0^2$ 
  will be estimated from atomic overlap integrals.

\begin{verbatim}
  * L1-shell core hole, S02 = 1
  EDGE  L1   1.0
\end{verbatim}
\end{Card}



\begin{Card}{SCF}{rfms1 [lfms1 nscmt ca nmix]}{Standard}{scf}
  This card controls {\feff}'s automated self-consistent potential
  calculations. All fields except rfms1 are optional.
  If this card is not specified, then all calculations are done with 
  non-self-consistent (overlapped atomic) potentials.
  By default \texttt{lfms1}=0, \texttt{nscmt}=30, \texttt{ca}=0.2, and 
  \texttt{nmix}=1.
  \begin{description}
  \item[\texttt{rfms1}]\hfill\\ This specifies the radius of the cluster
    for full multiple scattering during the self-consistency loop.
    Typically one needs about 30 atoms within the sphere specified by
    \texttt{rfms1}. Usually this value is smaller than the value \texttt{rfms}
    used in the \htmlref{FMS}{card:fms} card, but it should be larger than the 
    radius of the second coordination shell.  Will be ignored completely in k-space calculations.
    Must be converged in real-space calculations.
  \item[\texttt{lfms1}]\hfill\\ The default value 0 is appropriate for
    solids; in this case the sphere defined by \texttt{rfms1} is
    located on the atom for which the density of states is calculated.
    The value 1 is appropriate for molecular calculations and will
    probably save computation time, but may lead to inaccurate
    potentials for solids. When $\mathtt{lfms1} = 1$ the center of the
    sphere is located on the absorbing atom.
  \item[\texttt{nscmt}]\hfill\\ This is the maximum number of iterations
    the potential will be recalculated. A value of 0 leads to
    non-self-consistent potentials and Fermi energy estimates. A value of
    1 also yields non-self-consistent potentials but the Fermi energy is
    estimated more reliably from calculations of the $\ell$DOS.
    Otherwise, the value of \texttt{nscmt} sets an
    upper bound on the number of iterations in the self-consistency
    loop. Usually self-consistency is reached in about 10 iterations.
  \item[\texttt{ca}]\hfill\\ The convergence accelerator factor. This
    is needed only for the first iteration, since {\feff} uses
    the Broyden algorithm to reach self-consistency. A typical value
    is 0.2; however, you may want to try smaller values if there are
    problems with convergence. After a new density is calculated from
    the new Fermi level, the density after the first iteration is
    $$\rho_\mathrm{next} = ca*\rho_\mathrm{new} + (1-ca)*\rho_\mathrm{old}$$ 
    $\mathtt{ca}=1.0$ is extremely unstable and should not be used.
  \item[\texttt{nmix}]\hfill\\ This specifies how many iterations to do 
    with the mixing algorithm before starting the Broyden algorithm.
    The SCF calculations in materials containing f-elements may not converge. 
    We encountered such a
    problem for Pu. However, the SCF procedure converged if we started the
    Broyden algorithm after 10 iterations with the mixing algorithm with
    $\mathtt{ca}=0.05$.  \texttt{nmix} must be between 1 and 30; a value 
    outside of this range will be ignored, and replaced with an acceptable 
    value.
  \end{description}

\begin{verbatim}
  * Automated FMS SCF potentials for a molecule of radius 3.1 Angstroms
  SCF  3.1 1

  * To reach SCF for f-elements and UNFREEZEF we sometimes had to use
  SCF  3.7 0  30  0.05  10
\end{verbatim}
\end{Card}



\begin{Card}{S02}{s02}{Standard}{s02}
  The S02 card specifies the amplitude reduction factor $S_0^2$. If
  the entry for $S_0^2$ is less than 0.1, then the value of $S_0^2$ is
  estimated from atomic overlap integrals. Experimental values of
  $S_0^2$ are typically between 0.8 and 1.0.

  Alternatively, you can specify the value of $S_0^2$ in the 
  \htmlref{HOLE}{card:hol} or \htmlref{EDGE}{card:edg} card; however, 
  the meaning of the parameters in the \file{feff.inp} file is more 
  clear if you use the S02 card.
\begin{verbatim}
  * let FEFF calculate S02
  S02    0.0
\end{verbatim}
\end{Card}


\begin{Card}{CONFIG}{input [configuration]}{Useful}{cfg}
  This card modifies the electron configuration.  It allows the user to specify the ground-state occupation numbers of the orbitals of 
  a species of atoms, either by atomic number $Z$ or by potential type $iph$.  It is possible to assign atoms with the same $Z$ to different potential types with different configurations. 
  The parameter $input$ can take the following values: 
 \begin{itemize}
    \tightlist  
  \item $feff7$ signifying the same occupation numbers as were used in this old FEFF version;
  \item $file$ indicating there's a file called \file{config.inp} containing the relevant information;
  \item $card$ stating that the information is passed inside the card itself in the optional parameter $configuration$.
  \end{itemize}
  Note that several instances of CONFIG card can appear in \file{feff.inp} and all will be taken into account.
  There is a syntax to be used for specifying configurations; it follows the usual noble gas notations closely.  The resulting configuration
  is written to file \file{config.dat} for user inspection.  The general format is
  \texttt{ iph name [NobleGas] istate iocc- [iocc+] istate ... }
  where $iph$ is a potential index of \file{feff.inp} and $name$ a 2-letter element name (C, Au, ...).  If $iph$ is negative, the card applies to all
  atoms of type $name$.  It can even be used when there are no such atoms specified in \file{feff.inp}.  (Due to the presence of a core hole or when the ION card is used, {\feff} may use this information.)  The configuration can optionally be specified from
  a noble gas; acceptable values are He, Ne, Ar, Kr, Xe, Hg, or Rn.  All remaining fields must specify states.  $istate$ = 1s, 2p, 4f, ...  For s-states, only one 
  occupation number $iocc$ follows; for higher states, a l-1/2 and a l+1/2 occupation number is required.  Occupation numbers can be fractional (e.g. 1.5).  A positive
  number indicates a 'valence state'; a negative number indicates a 'core state'.  This distinction is only used when calculating the exchange-correlation potential and has no implications for the rest of the calculation.
\begin{verbatim}
* For all (iph=-1) C atoms (Z=6) use 1s^2 2s^2 2p^2 (note there's 2p_1/2 and 2p_3/2) 
* and consider the 2s and 2p valence electrons and 1s core electrons :
  CONFIG card 1   * <-- "1" counts the following lines
  -1 C 1s -2 2s 2 2p 1 1
\end{verbatim}
  A second example:
\begin{verbatim} 
* Start from FEFF7 type configurations, but for the I atoms of potential type iph=2, 
* start from Kr configuration and add 4d^10 5s^1 5p^6 (all valence)
  CONFIG feff7
  CONFIG card 1
  2 I Kr 4d 4 6 5s 1 5p 2 4
\end{verbatim}
  Finally, it is also possible to set the ispn variable:
\begin{verbatim}
* For all Cr atoms of potential type iph=4: 
  CONFIG card 1
  4 Cr Ar 3d 4 0 4s 1 4p 1 s 1 0 s 0
\end{verbatim}
  IMPORTANT: This card is new and experimental in {\feffcur}.  Please double-check \file{config.dat} to make sure your input has been parsed correctly, and contact the authors in case of doubt (or bugs).  Also be aware that non-standard configurations may lead {\feff} to fail, or may produce nonsense.
\end{Card}


\begin{Card}{EXCHANGE}{ixc vr0 vi0 [ixc0]}{Useful}{exc}
  The EXCHANGE card specifies the energy dependent exchange
  correlation potential to be used for the fine structure and for the
  atomic background. 

  \texttt{ixc} is an index specifying the
  potential model to use for the fine structure and the optional
  \texttt{ixc0} is the index of the model to use for the background
  function. 

  The calculated potential can be corrected by adding a
  constant shift to the Fermi level given by \texttt{vr0} and to a
  pure imaginary ``optical'' potential (i.e., uniform decay)
  given by \texttt{vi0}. Typical errors in {\feff}'s self-consistent
  Fermi level estimate are about 1 eV. (The
  \htmlref{CORRECTIONS}{card:cor} card 
  \begin{latexonly}
    in Section~\ref{sec:Comb-contr-from}
  \end{latexonly}
  is similar but allows the user to make small changes in \texttt{vi0}
  and \texttt{vr0} {\it after}  the rest of the calculation is completed,
  for example in a fitting process.)  

  The Hedin--Lundqvist self-energy is used by default and appears to be 
  the best choice for most applications we have tested in detail. The
  partially nonlocal model (ixc=5) gives slightly better results in some
  cases, but has not been tested extensively.

  Another useful exchange model is the Dirac-Hara exchange correlation
  potential with a specified imaginary potential vi0. This may be
  useful to correct the typical error in non-self-consistent estimates of
  the Fermi level of about +3 eV and to add final state and
  instrumental broadening.

  Defaults if the EXCHANGE card is omitted are: \texttt{ixc}=0
  (Hedin--Lundquist), vr0=0.0, vi0=0.0. For XANES, the ground state
  potential (\texttt{ixc0}=2) is used for the background function and
  for EXAFS the Hedin--Lundqvist (\texttt{ixc0}=0) is used.

  Indices for the available exchange models:
  \begin{itemize}
    \tightlist
  \item[ \texttt{0}\quad] Hedin--Lundqvist + a constant imaginary part
  \item[ \texttt{1}\quad] Dirac--Hara + a constant imaginary part
  \item[ \texttt{2}\quad] ground state + a constant imaginary part
  \item[ \texttt{3}\quad] Dirac--Hara + HL imag part + a constant
    imaginary part
  \item[ \texttt{5}\quad] Partially nonlocal: Dirac--Fock for core +
    HL for valence electrons + a constant imaginary part
  \end{itemize}
\begin{verbatim}
  *Hedin-Lundqvist -2eV edge shift and 1eV expt broadening
  EXCHANGE 0 2. 1.

  *Dirac-Hara exchange -3 eV edge shift and 5 eV optical potential
  EXCHANGE 1 3. 5.
\end{verbatim}
\end{Card}



\begin{Card}{HOLE}{ihole s02}{Useful}{hol}
  Deprecated.  Supported for compatibility only.  The HOLE card is equivalent to the EDGE card, but the shell is specified
  by a hole-code index. It includes the amplitude reduction factor $S_0^2$
  just as the EDGE card does. If the entry for $S_0^2$ is less than 0.1,
  then $S_0^2$ will be estimated from atomic overlap
  integrals. Experimental values of $S_0^2$ are typically
  between 0.8 and 1.0. The defaults if the HOLE card is omitted are
  \texttt{ihole}=1 for the $K$ shell and $S_0^2$=1. The hole codes
  are shown in Table~\ref{tab:holecodes}, however, note that {\feff}
  will not accept ihole=0 and one must use the \htmlref{NOHOLE}{card:noh} 
  card instead to calculate without the core-hole.

  For $\texttt{ihole}>4$, the core-hole lifetime parameter
  ($\gamma_{\textrm{ch}}$) is not tabulated in {\feff} and is set
  equal to 0.1 eV, since the final state losses are then dominated by
  the self-energy. Use the \htmlref{EXCHANGE}{card:exc} card to make
  adjustments ($\gamma_{\textrm{ch}} = 0.1 + 2\cdot\mathtt{vi0}$).

\begin{verbatim}
  * K-shell core hole, S02 estimated by overlap integrals
  HOLE  1   0.0
\end{verbatim}
\end{Card}



\begin{table}[htbp]
  \begin{center}
    \begin{tabular}[h]{|c>{\ttfamily}c|c>{\ttfamily}c|c>{\ttfamily}c|c>{\ttfamily}c|}
      \hline
      index & \textrm{edge} & index & \textrm{edge} & index &
      \textrm{edge} & index & \textrm{edge} \\
      \hline
      -  & -- & 7  & M3 & 14 & N5 & 21 & O5 \\
      1  & K  & 8  & M4 & 15 & N6 & 22 & O6 \\
      2  & L1 & 9  & M5 & 16 & N7 & 23 & O7 \\
      3  & L2 & 10 & N1 & 17 & O1 & 24 & P1 \\
      4  & L3 & 11 & N2 & 18 & O2 & 25 & P2 \\
      5  & M1 & 12 & N3 & 19 & O3 & 26 & P3 \\
      6  & M2 & 13 & N4 & 20 & O4 &    &    \\
      \hline
    \end{tabular}
    \caption[Available hole codes]{Available hole codes. The entries
      in the column $edge$ are recognized by the EDGE card.}
    \label{tab:holecodes}
  \end{center}
\end{table}



\begin{Card}{NOHOLE}{}{Useful}{noh}
  Deprecated.  Supported for compatibility only.  This card roughly simulates the effect of complete core-hole
  screening. It is useful to test the final state rule for
  calculated XAS, and to compare with other calculations (such as band
  structure or other codes) that do not have a core hole.
  The code will use the final states specified by the \htmlref{HOLE}{card:hol}
  or \htmlref{EDGE}{card:edg} card
  for the matrix element calculation --- but will calculate potentials and phase shifts as if
  there is no core-hole. For $d$DOS and $L_{II}$ or $L_{III}$
  absorption calculations, for example,  NOHOLE often gives better
  agreement for white line intensities. Conversely NOHOLE tends to give
  poor XANES intensities for K-shell absorption in insulators.
  
  It is now recommended to use the COREHOLE card instead.  However, old \file{feff.inp} files with
  NOHOLE still work.
\end{Card}



\begin{Card}{RGRID}{delta}{Useful}{rgr}
  The radial grid used for the potential and phase shift calculation
  is $$r(i) = \exp(-8.8 + (i-1)\cdot\delta)$$ where $\delta\equiv\mathtt{delta}$. 
  $\delta=0.05$ by default, which is sufficient for most cases. 
  However, occasionally there are convergence problems in the atomic background 
  at very high energies (the background curves upward) and in the phase
  shifts for very large atoms. If such convergence problems are encountered
  we suggest reducing $\delta$ to 0.03 or even 0.01. This will solve
  these problems at the cost of longer computation times (the time is
  proportional to $1/\delta $). This option is also useful for testing
  and improving convergence of atomic background calculations.
\end{Card}
\begin{verbatim}
  * reduce grid for more accurate background at high energy
  RGRID 0.03
\end{verbatim}



\begin{Card}{UNFREEZEF}{}{Useful}{unf}
In somes applications of $f$-electron
systems, we found that it is preferable to freeze the $f$-electron
density at the atomic value in order to achieve well converged
SCF potentials. However, this may sometimes lead to less accurate results,
and it doesn't improve convergence in all cases.  Freezing $f$-electrons is the default in {\feff}8.4. If one wants
to attempt to calculate the $f$-DOS self-consistently,
as in {\feff}8.0 and 8.1, the UNFREEZEF card should be used.
\begin{verbatim}
  * To include f-electrons in SCF calculations use
  UNFREEZEF 
\end{verbatim}
\end{Card}



\begin{Card}{CHSHIFT}{Advanced}{csh}
  Calculate initial core-state energy level using the final self-consistent potential. This card will give more accurate relative shifts of the edge, however, the absolute edge energy is not accurate with this method.
\end{Card}



\begin{Card}{CHBROADENING}{igammach}{Advanced}{chb}
  If $iGammaCH = 1$, the core state lifetime effects are taken into
  account by convolving the final spectrum with a Lorentzian of width
  $\Gamma_{ch}$, as is done with broadening given by the CORRECTIONS
  card.  If $iGammaCH = 0$ (default), 
  the Green's function is calculated for complex energy $E +
  i\Gamma_{ch}/2.0$, as is done for imaginary energy given in the
  EXCHANGE card. These should be equivalent for small
  $\Gamma_{ch}$. This card is useful for providing quick (rough) results for
  multiple edges. For
  example, one might like to calculate both the L2 and L3 edges of Fe
  without re-running the SCF and FMS modules. In this case the results
  should be quite good since the core $P_{1/2}$ and $P_{3/2}$
  core-hole potentials are essentially equivalent. In order to do this
  the following steps should be performed:

  \begin{enumerate}
    \item{Put the CHBROADENING card in the feff.inp file with igammach
    set to 1, i.e.
    \begin{verbatim}
        CHBROADENING 1
    \end{verbatim}}
    \item{Run feff through completely for the L2 edge, i.e., set
    the EDGE L2 card in the feff.inp file.}
    \item{Copy the ouput files of interest (xmu.dat, eels.dat, chi.dat,
    ...) to another name or another directory.}
    \item{Change the edge to L3, i.e., put EDGE L3 in the feff.inp
    file}
    \item{Run the following executables in the following order
      \begin{itemize}
	\item{rdinp}
	\item{atomic}
	\item{mkgtr}
	\item{path}
	\item{genfmt}
	\item{ff2x}
	\item{eels}
      \end{itemize}}
  \end{enumerate}

  You should now have output files (xmu.dat, eels.dat, chi.dat, ...)
  for the L3 edge. Note: A similar procedure can be used for quickly obtaining
  correct results of different polarizations of a single edge.
\end{Card}



\begin{Card}{CHWIDTH}{gamma}{Advanced}{chw}
 Set the core hole lifetime manually, instead of using {\feff}'s internal table.  Gamma is in eV.
\end{Card}

\begin{Card}{CORVAL}{emin}{Advanced}{cva}
  The core-valence separation energy is found by scanning the DOS within an energy window.  The \texttt{emin} parameter sets the lower bound (in eV) of this energy window.  It is 70 eV by default.  For some materials it is necessary to lower this bound, e.g. to 100 eV.  For example, when SCF convergence is elusive because occupation numbers for one or more $l$-values are changing drastically between SCF iterations due to states moving above and below a poor estimate of the core-valence separation energy.  We plan to replace the current mechanism by a more robust and automated algorithm, but in the meantime users can use the CORVAL card to handle some of these difficult cases.
\end{Card}

\begin{Card}{CORVAL}{Emin}{Advanced}{crv}
 {\feff} distinguishes between core and valence states.  It searches for a so-called "core-valence separation energy" within an energy window that works well for most cases.  Sometimes, the value found within the standard window is problematic, when states that are just above or below it shift across it during the SCF procedure.  This can lead to convergence difficulties or invalid results.  Using the CORVAL card, one can change the default lower bound of the energy window (usually in order to lower it).  Input in eV.  Default is $E_min = -70eV$.
\end{Card} 

\begin{Card}{EGAP}{}{Advanced}{ega}
  This card will only change results if run with the \htmlref{MPSE}{card:mps} card.
  For a material with a band gap, the Fermi energy is usually set
  approximately at the top of the gap. The self-energy should however
  be referenced to the middle of the gap. If the band gap is known,
  the EGAP card can be used to tell FEFF it's value.
\end{Card}



\begin{Card}{EPS0}{eps0}{Advanced}{eps}
  This card will only change results if run with the
  \htmlref{MPSE}{card:mps} card. This card will renormalize the poles
  which represent the loss function
  so that it is consistent with the dielectric function
  specified by \texttt{eps0}. The expected effect is that the
  broadening in the spectrum caused by the self-energy will increase
  as a function of \texttt{eps0}, while the stretch (blue shifting of
  peaks relative to the edge) will decrease. Users should be careful
  with this card since the renormalization of the loss function
  shouldn't change it's shape too much. A drastic change of shape in
  the loss function indicates that the original loss function is
  probably not a very good representatoin of the true loss function,
  or the dielectric constant is not correct. To check the shape of the
  new loss function, compare a plot of columns 1 and 4 in exc.dat with
  a plot of columns 1 and 2 in loss.dat.
\end{Card}



\begin{Card}{EXTPOT}{}{Advanced}{ext}
  Use external potentials calculated by another code.  Documentation currently unavailable.  Experimental feature.
\end{Card}



\begin{Card}{FOLP}{ipot folp}{Advanced}{fol}
  The FOLP card sets a parameter which determines by what factor the
  muffin-tin radii are overlapped for the specified potential index. 
  We recommend that the \htmlref{AFOLP}{card:afo} 
  card be used (default overlap = 1.15) in cases with severe anisotropy.
  FOLP should be used with caution, for example, for hydrogen or for 
  fitting AXAFS. Typically only values larger than 1 and less than 1.3
  should be used, except for hydrogen atoms, where we recommend
  the value 0.8. The AFOLP card is ignored when FOLP
  is used for a particular potential type.
\begin{verbatim}
  *  +20% overlap of muffin tin with unique potential 1
  *  -20% overlap of muffin tin with unique potential 2
  FOLP 1  1.2    * adjust overlap to fit AXAFS
  FOLP 2  0.8    * use 0.8 for hydrogen
\end{verbatim}
\end{Card}



\begin{Card}{INTERSTITIAL}{inters vtot}{Advanced}{int}
  The construction of the interstitial potential and density may be
  changed by using this card.
  \texttt{inters}=1 might be useful when only the surroundings of the absorbing
  atom are specified in \file{feff.inp}.
  \begin{description}
  \item[\texttt{inters}]\hfill\\ \texttt{inters} defines how to 
    find the interstitial potential. \texttt{inters}=0 (default): the interstitial 
    potential is found by averaging over the entire extended cluster in 
    \file{feff.inp}. \texttt{inters}=1 : the interstitial potential is found locally 
    around the absorbing atom.
  \item[\texttt{vtot}]\hfill\\ the volume per atom normalized
   by \texttt{ratmin}$^3$ (\texttt{vtot}=(volume per atom)/\texttt{ratmin}$^3$),
    where \texttt{ratmin} is the shortest bond for the absorbing atom. This 
    quantity defines the total volume (needed to calculate the interstitial 
    density) of the extended cluster specified in \file{feff.inp}. If \texttt{vtot} 
    $\leq0$ then the total volume is calculated as a sum of Norman sphere volumes. 
    Otherwise, $total\ volume = \mathtt{nat} * (\mathtt{vtot}*\mathtt{ratmin}^3)$,
    where \texttt{nat} is the number of atoms in an extended cluster. Thus 
    \texttt{vtot}=1.0 is appropriate for cubic structures, such as NaCl. The 
    INTERSTITIAL card may be useful for open systems (e.g. those which have ZnS 
    structure).
  \end{description}

\begin{verbatim}
  * improve interstitial density for ZnS structures.
  * vtot = (unit_cell_volume/number_of_atoms_in_unit_cell)/ratmin**3)=1.54
  INTERSTITIAL  0 1.54
\end{verbatim}
\end{Card}



\begin{Card}{ION}{ipot ionization}{Advanced}{ion}
  This card can be unstable and should be used with caution.
  The ION card ionizes all atoms with unique potential index
  \texttt{ipot}. Negative values and non-integers are permitted,
  however ionicities larger than 2 and less than $-1$ often yield
  unphysical results. Our experience with charge transfers using the
  SCF card suggests values for \texttt{ionization} about 5--10 times
  smaller than the formal oxidation state. The ION card is probably
  not needed if the potential is self-consistent. However, it can be
  used to put some total charge on a cluster. In this case we suggest
  using the same ionicity for all atoms in the cluster (i.e. total
  ionization divided by number of atoms). For example, for diatomics
  like Br2, the fully relaxed configuration has a formal ionization of
  1 on the scattering atom. Because of charge transfer, the actual
  degree of ionization is much smaller. In non-self-consistent
  calculations the default (non-ionized) scattering potentials are
  often superior to those empirically ionized, and the results should
  be checked both ways. If ION is omitted, the atoms are not ionized 
  by default.
\begin{verbatim}
  * Simulates effective ionization for formal valence state +1
  * ipot, ionization
  ION  1  0.2
\end{verbatim}
\end{Card}



\begin{Card}{JUMPRM}{}{Advanced}{jum}
  Remove potential jumps at muffin tin radii.  Documentation currently unavailable. FIX
\end{Card}



\begin{Card}{NUMDENS}{ipot numdens}{Advanced}{num}
  This card will only change results if used with the
  \htmlref{OPCONS}{card:opc} card.
  The NUMDENS card sets the number density $n_{\rm i}$ of
  the atoms of type 
  $i$ specified by potential ipot. The number density is used to
  calculate the loss function for use with the many-pole model
  self-energy. By default the number density is
  estimated using the cluster of atoms provided in the feff.inp file
  as well as the 
  stoichiometry. If the stoichiometry is not provided, it will be
  estimated as well based on the cluster provided.
  In some cases these estimates are inaccurate and a correct number
  density should be provided. The number is given by $n_{\rm i} = N_{i}/V$,
  where $N_{i}$ is the total number of atoms in a unit cell, and V is
  the unit cell volume.
\end{Card}



\begin{Card}{OPCONS}{}{Standard}{opc}
  This card provides a fast and easy way to compute a rough estimate of the
  loss function using a weighted average of atomic loss functions. The
  output is contained in the file \file{loss.dat} 
  and will be used in calculations of the self-energy if the
  \htmlref{MPSE}{card:mps} card is also specified. The
  \htmlref{NUMDENS}{card:num} is useful if the OPCONS card is
  specified. 
\end{Card}



\begin{Card}{PREPS}{}{Advanced}{pre}
  If PREPS is specified along with the  \htmlref{OPCONS}{card:mps}
  card, the atomic dielectric function used for calculating the loss
  function will be printed in epsilon.dat. 
\end{Card}


\begin{Card}{RESTART}{}{Advanced}{res}
  If RESTART is specified, {\feff} will start the SCF calculation of the potentials from an existing \file{pot.bin} file, rather than starting from
  overlapped atomic potentials.  This way one can continue an earlier SCF calculation (e.g. after it ran out of iterations), or one can use an existing
  set of potentials as a good starting point for a similar structure (e.g., after making small changes to the atom coordinates).  Note that this strategy is not
  guaranteed to lead to good results, or to even converge.  This is a new feature and has not been tested in all possible situations.  If you need assistance,
  please contact us. 
\end{Card}


\begin{Card}{SCREEN}{parameter value}{scr}
 The \module{screen} module, which calculates the RPA core hole potential, is a 'silent' module: it has no obvious input but instead runs entirely on 
 default values.  Using the SCREEN card you can change these default values.  They will be written to an optional \file{screen.inp} file (which you can also
 edit manually).  The SCREEN card can occur more than once in \file{feff.inp}; all entries will be applied to the calculation.
 \texttt{parameter} must be one of : ner (40), nei (20), maxl (4), irrh (1), iend (0), lfxc (0), emin (-40 eV), emax (0 eV), eimax (2 eV), ermin (0.001 eV), rfms (4.0), nrptx0 (251).  For most calculations the default values given (between brackets) are fine.  Occasionally we've changed rfms, maxl, or emin.  Note that the \module{screen} is only active with COREHOLE RPA.
 \begin{verbatim}
 * Set the cluster radius for the RPA potential calculation higher than the default of 4.0 .
 COREHOLE RPA
 SCREEN rfms 5.0
 \end{verbatim}
\end{Card}


\begin{Card}{SETEDGE}{}{Advanced}{set}
  This will set the edge energies to those found in the Elam/Mcmasters
  tables for the element and edge in question. By default the
  estimated edge energy is based on Hartree-Fock atomic
  calculations. This card is meant to be used with an
  external program for obtaining the full optical constants including
  response from all occupied electrons. 
\end{Card}



\begin{Card}{SPIN}{ispin [x  y  z] }{Advanced}{spi}
 This card is used to specify the type of spin-dependent calculation 
 for spin along the (x, y, z) direction, along the z-axis by default. 
 The SPIN card is required for the calculation of all spin-dependent effects, 
 including \htmlref{XMCD}{card:xnc} and SPXAS
 \begin{latexonly}
   (see Section~\ref{sec:SPXAS}) 
 \end{latexonly}. 

  Whenever spin-dependent calculations are made, the spin amplitude and 
 relative spin orientation should be specified in the \texttt{spinph} argument
 of the \htmlref{POTENTIALS}{card:pot} card.
 
 If {\feff} has been compiled with the parameter \texttt{nspx}=1 (default), 
 the values of the index \texttt{ispin} correspond to the following calculations:
\begin{table}[htbp]
  \begin{center}
    \begin{tabular}[h]{rl}
      \hline\hline
      \texttt{ispin} & \quad meaning \\
      \hline
      2    & spin-up SPXAS and $\ell$DOS\\
      $-2$ & spin-down SPXAS and $\ell$DOS\\
      1    & spin-up portion of XMCD calculations \\
      $-1$ & spin-down portion of XMCD calculations \\
      \hline\hline
    \end{tabular}
    \caption{Allowed values of the \texttt{ispin} argument of the SPIN card.}
    \label{tab:spin}
  \end{center}
\end{table}

  The default $\mathtt{ispin}=0$ is used for spin-independent calculations.

 If {\feff} is compiled with $\mathtt{nspx}=2$, one simply uses $\mathtt{ispin}=1$ 
 or $\mathtt{ispin}=2$ as needed, and the up and down spins are taken care of 
 automatically.
 \begin{latexonly}
   See Section~\ref{sec:Spin-depend-calc} for more information on 
   the strategies and options available for spin-dependent calculations.
 \end{latexonly}
  
  The spin-dependent potentials are calculated from the spin-dependent 
 densities, using von Barth-Hedin results for the uniform electron gas.
 We use this approximation to construct the spin-dependent muffin-tin potential. 
 This should be fine for EXAFS, where small details of the potential are irrelevant, 
 but may not be good enough in the XANES region, where the self-consistent 
 spin-dependent muffin-tin potential can lead to better results.

\begin{verbatim} 
  * Do spin-up XMCD calculation (XMCD card must also be included)
  SPIN 1
  XMCD
\end{verbatim}

POTENTIALS card for spin-dependent calculations:
\begin{verbatim} 
  * GdFeO, spin-dependent
  * Note that SPIN card must also be present in feff.inp, as well as 
  * additional cards specific to desired calculations
  POTENTIALS
  *  ipot  z  tag  lmax1  lmax2  xnatph  spinph
      0   64   Gd    3      3     0.1     7.0     (c sublattice)     up
      1   26   Fe    2      3     2       4.0     (a sublattice)     up
      2   26   Fe    2      3     3      -4.0     (d sublattice)     down
      3    8   O     2      3     12.0    0.0     (h sublattice)
      4   64   Gd    3      3     2.9     7.0     (c sublattice)     up
\end{verbatim}
\end{Card}



\subsection{XSPH: Cross-section and Phase Shifts}
\label{sec:Cross-section-phase}


\begin{Card}{LDEC} {ld}{Advanced}{lde}
    Only active for NRIXS.  This option gives the number of angular momentum channels $l \leq ld$ for which the decomposition
  of the inelastic scattering spectrum is given. The output for this card is written into file
  \file{xmul.dat} in \module{ff2x}.  Changing $ld$ does not change the total spectrum, but only the
  amount of partial data written to file.
\begin{verbatim}
  * Output momentum channels up to l=2
  LDEC 2
\end{verbatim}
\end{Card}



\begin{Card}{LJMAX}{lj}{Advanced}{ljm}
  This is an NRIXS option and will be ignored for any other type of spectroscopy.  LJMAX influences all modules starting 
  from xsph.  This card specifies the number of terms included in the calculation of the excitation matrix elements
  \begin{equation} \label{ljmax}
  M_L(\vec{q},E) = \bra{R_L(E)}e^{i \vec{q}\cdot \vec{r}}\ket{i}  \approx \bra{R_L(E)} 4\pi \sum^{l_j \geq l}_{lm} {i^{l} j_l(qr)Y^{*}_{lm}(\vec{q}) Y_{lm}(\vec{r})} \ket{i} 
  \end{equation}
  Typically the value of angular momentum channels needed is dependent on the momentum
  transfer and energy transfer range that one is calculating. For typical XANES calculation one
  does not need $lj > 10$ and most often $lj = 5$ is sufficient. For EXAFS one needs more than for
  XANES. However, most often this parameter has to be determined on case-by-case basis. If
  this parameter is too large it can occasionally cause small errors. For $lj > 25$ the code starts to become unstable.
  Default value is $lj = 1$, approximately corresponding to the dipole selection rule.  This is almost certainly too low for any
  situation.  There is no physical limitation on $lj$; it simply needs to be converged.
\begin{verbatim}
  * Use contributions to the GOS up to l=5
  LJMAX 5
\end{verbatim}
\end{Card}



\begin{Card}{MPSE}{ipl}{Standard}{mps}
 
  This card runs {\feff} with a many-pole model for the self energy.

  The values of  \texttt{ipl} correspond to the following:  
  1: use an ``average'' self-energy which is applied to the whole
     system (default).
  2: use a density dependendent self-energy which is different at
     every point inside the muffin-tin radius.

  It takes as input a file \file{loss.dat} containing the loss function.  This can
  either be measured or taken from a calculation. For a rough but very
  fast estimate of the loss function, see the
  \htmlref{OPCONS}{card:opc} card. See the references for more  
  information.  
  The output file \file{mpse.dat} contains the self energy and the renormalization
  constant (at the interstitial radius) as a function of photoelectron
  energy relative to the Fermi Energy in eV.

  PLASMON can be used as an alias to the MPSE card. 

\begin{verbatim}
* make regular FEFF calculations, print out the self-energy
MPSE  0
\end{verbatim}

\end{Card}

 
 
 \begin{Card}{PMBSE}{ipmbse nonlocal ifxc ibasis}{Advanced}{pmb}
  This card is  currently in development.
\end{Card}



\begin{Card}{RPHASES}{}{Advanced}{rph}
  If this card is present, real phase shifts rather than complex phase
  shifts will be used. The results of the calculation will not be
  accurate. This option is intended to allow users to obtain real
  scattering phase shifts for use with other programs, or for
  diagnostic purposes. The phase shifts can be written to output
  files \file{phaseNN.dat} using the \htmlref{PRINT}{card:pri} card. 
  If the RPHASES card is present, these will contain the real phase shifts.
\end{Card}



\begin{Card}{RSIGMA}{}{Advanced}{rsi}
  If this card is present, the imaginary part of the self-energy will be 
  neglected. It might be useful for calculations in the XANES region, since 
  the imaginary part of the Hedin-Lundqvist self-energy tends to overestimate 
  losses in this region.
\end{Card}



\begin{Card}{TDLDA}{ixfc}{Advanced}{tdl}
 Uses time-dependent local density approximation (TDLDA) theory 
 to account for screening of the x-ray field and of the 
 photoelectron--core-hole interaction. The parameter \hbox{\texttt{ixfc}} 
 determines whether static or dynamic screening is used. 
 $\mathtt{ixfc}=0$ (static screening) accounts for screening 
 of the x-ray field (but not the field of the core-hole), 
 blue-shifting the spectrum. Thus \hbox{\texttt{TDLDA 0}} 
 works well at high energies. $\mathtt{ixfc}=1$ (dynamic 
 screening) accounts for screening of the x-ray and core-hole 
 fields, blue-shifting the spectrum less than $\mathtt{ixfc}=0$. 
 The TDLDA card affects only module xsph.

 TDLDA theory takes into account polarization-type many body 
 effects (i.e., polarization of the electronic charge) which screen 
 the local x-ray field. These effects are most important for xrays 
 with energies less than 1 keV, hence \hbox{\texttt{TDLDA 0}} works 
 well at high energies. The screened interaction is calculated 
 partially based on the Bethe-Salpeter equation, in the basis of 
 local atomic states. This approximation yields efficient calculations 
 of the spectra in terms of screened transition matrix elements. Note 
 that TDLDA does not account for core-hole relaxation effects.
 
 L-shell x-ray absorption in 3d transition metals is sensitive to dynamic 
 screening effects. For rare-gas solids, dynamic screening accounts for 
 deviations of the $L_{3}$/$L_{2}$ intensity branching ratio from the 2:1 
 value of independent-electron theory.
 
 See the paper on dynamic screening by Ankudinov, Nesvizhskii, and Rehr 
 \begin{latexonly}
 (see the references in \htmlref{Appendix C}{sec:Append-C-Refer}) 
 \end{latexonly}
 for further details 
 on the implementation of TDLDA in the {\feff} calculations. See also Ch. 8 (pp. 82-105) of 
 \htmladdnormallink{A.I. Nesvizhskii's thesis}{http://leonardo.phys.washington.edu/feff/papers/dissertations/thesis_nesvizhskii.ps}) for a brief description of the TDLDA theory.
 
 On certain occasions, instabilities have been observed in the TDLDA routines, causing
 sometimes NaN output or unphysical glitches in the tails of edges.  If such symptoms occur,
 the {\feff} authors should be contacted for help.

\begin{verbatim}
  * use static screening. this will only impact module 2, XSPH
  TDLDA 0
\end{verbatim}
\end{Card}



  \subsection{FMS and MKGTR : Full Multiple Scattering}

\label{sec:Full-mult-scatt}


\begin{Card}{FMS}{rfms  [lfms2 minv toler1 toler2 rdirec]}{Standard}{fms}
  Compute full multiple scattering within a sphere of radius
  \texttt{rfms} centered on the absorbing atom (real space) or for the unit cell of the crystal (k-space).
\begin{description}
  \item[\texttt{rfms}]\hfill\\ Completely ignored in k-space calculations.  This is the cluster radius used in modules 
  \module{ldos} and \module{fms} but not in \module{pot}. It is also the lower 
  limit of pathfinder calculations. The \module{fms} module sums all MS paths within 
  the specified cluster. Typically, a converged XANES calculation requires
  about 50-150 atoms in a cluster, but sometimes more are needed.

  For EXAFS analysis, one typically calculates to $k=20$, but FMS results
  are not accurate at high energies. Thus if you are running
  {\feffcur} for EXAFS, you should not use the FMS card.

  If the value of \htmlref{RPATH}{card:rpa} 
  \begin{latexonly}
    as described in Section~\ref{sec:Path-enum-modul}
  \end{latexonly} 
  is greater than \texttt{rfms}, the
  pathfinder will look for paths which extend beyond the cluster used
  for the FMS and add them to the FMS calculation of the $\ell$DOS and
  XANES:
  $$G_{\mathrm{tot}}=G_{\mathtt{fms}} + G_0t_iG_0 +
  G_0t_iG_0t_jG_0+\cdots$$
  where at least one atoms $i$ in the path is outside the FMS cluster
  and the value of RPATH is the maximum half path length for the \module{ldos},
  \module{fms} and \module{path} modules. However, this is generally not recommended,
  as the MS expansion sometimes does not converge well in the XANES energy
  region. Thus it is generally best not to add paths for LDOS and XANES, and
  RPATH should be less than \texttt{rfms}. Adding single scattering
  paths only (NLEG 2) usually works well to check the convergence of
  FMS. But adding double scattering (NLEG 3) often leads to very bad
  results in XANES. Thus RPATH is useful for EXAFS, but for XANES only
  when the path expansion is stable.

  \item[\texttt{lfms2}]\hfill\\ Optional argument. This is a logical 
  flag that defines how the FMS is done, similar to the flag \texttt{lfms1} 
  in the \htmlref{SCF}{card:scf} card. With the default value of 0 (appropriate 
  for solids), the FMS is calculated for a cluster of size \texttt{rfms}
  around each representative unique potential. With \texttt{lfms}=1
  (appropriate for molecules), FMS is done only once for a cluster of 
  size \texttt{rfms} around the absorbing atom only. The proper use of this 
  flag can lead to a considerable time savings.

  For example, if you calculate FMS for a molecule smaller than 40
  atoms, there is no need to invert $\mathtt{nph}+1$ matrices, and
  $\mathtt{lfms1}=1$ will reduce time for calculations by a factor
  ($\mathtt{nph}+1$), where \texttt{nph} is the number of unique
  potentials listed in the \htmlref{POTENTIALS}{card:pot} card).

  Typically the FMS card will be used with $\mathtt{lfms2}=0$, for example:

\begin{verbatim}
  * for XANES and LDOS need about 100 atom cluster
  FMS  6.0
\end{verbatim}


  For molecules of less than 30 atoms of radius 4.0 {\AA} we suggest
  using $\mathtt{lfms2}=\mathtt{lfms1}=1$, as in:
\begin{verbatim}
  SCF 5.0 1 
  FMS  5.0 1
  RPATH  -1
\end{verbatim}

  \item[\texttt{minv}]\hfill\\ Optional. This is an index that defines 
the FMS algorithm used in the calculations. By default, ($\mathtt{minv}=0$)
 the FMS matrix inversion is performed using LU decomposition. However,
several alternatives have been designed for the FMS algorithm that start to
work faster than LU decomposition for clusters of more than 100 atoms.
(See the {\feff}8.2 reference).
We strongly recommend the Lanczos recursion method ($\mathtt{minv}=2$) which is very robust
and speeds the calculations by a factor of 3 or more.
The Broyden algorithm ($\mathtt{minv}=3$) is faster, but less reliable, and 
may fail to converge if the FMS matrix has large eigenvalues.

  \item[\texttt{toler1}]\hfill\\ Optional. This defines the tolerance to stop
recursion and Broyden algorithm. The default value of 0.001 gives results
in agreement with LU decomposition to within a linewidth. 

  \item[\texttt{toler2}]\hfill\\ Optional. Sets the matrix element of the $Gt$ matrix
to zero if its value is less than \texttt{toler2} (default 0.001).

  \item[\texttt{rdirec}]\hfill\\ Optional. Sets the matrix element of the $Gt$ matrix
to zero if the distance between atoms is larger than \texttt{rdirec}.
\end{description}
The last two variables can make the matrix $Gt$ very sparse so both recursion
and Broyden algorithms work faster. For example for large Si calculations
with the Lanczos algorithm, we used:
\begin{verbatim}
  FMS  29.4 0  2  0.001 0.001 40.0
\end{verbatim}
\end{Card}



\begin{Card}{DEBYE}{temp thetad [idwopt] [...]}{Useful}{deb1}
\begin{latexonly}
  See the description \htmlref{here}{car:deb2}.
\end{latexonly}
 \begin{htmlonly}
  {\large DEBYE in \module{fms}:}\newline
 \end{htmlonly}  

 \begin{htmlonly}
  {\large DEBYE in \module{ff2x}:}\newline
 FIX  NOT UPDATED!!!
  The DEBYE card is used to calculate Debye--Waller factors for each
  path using the correlated Debye Model. The model is best suited for
  homogeneous systems, where it is quite accurate. CAUTION: in
  heterogeneous systems the model only gives approximate values which
  can easily be off by factors of two or more. If this card is present, 
  the correlated Debye model Debye--Waller factors will be summed with the 
  DW factors from the \htmlref{SIG2}{card:sig} card and from the \file{list.dat} 
  file, if present. Note that the DEBYE card is currently incompatible with the 
  \htmlref{CFAVERAGE}{card:cfa} card for options other than the correlated Debye 
  model (\texttt{idwopt} $>$ 0). Temperatures are specified in kelvin.

\begin{verbatim}
  *Debye-Waller factors for Cu at 190K with correlated Debye Model
  DEBYE  190 315
\end{verbatim}

  By default, \texttt{idwopt}=0 specifies that the correlated Debye model 
  is used to calculate EXAFS Debye--Waller factors. Two additional models for
  calculating DW factors are available in {\feffcur} based on the information
  about the harmonic force constants in the material. \texttt{idwopt}=1
  means the equation of motion (EM) method is used to get Debye--Waller
  factors and \texttt{idwopt}=2 means the recursion method (RM) which
  is an improved correlated Einstein model. Both methods are faster than 
  molecular dynamics simulations, and the recursion method is much faster 
  than the equation of motion method. However, the equation of motion method 
  leads to somewhat more accurate results than the recursion method. These 
  additional methods seem to be superior to the correlated Debye model in cases 
  with tetrahedral coordination, such as solid Ge and many biological materials. 
  Both EM and RM methods need additional input (the force constants) and a 
  complete description of both is given in Anna Poiarkova's thesis (see the 
  {\feff} project web site, \htmladdnormallink{http://feff.phys.washington.edu/feff/}{http://feff.phys.washington.edu/feff/Docs/Docs.html}) 
  and in the associated documentation.

\begin{verbatim}
  * Calculate Debye-Waller factors for Cu at 190K with equation of motion
  DEBYE  190 0 1
\end{verbatim}
 \end{htmlonly}

\end{Card}



\begin{Card}{BANDSTRUCTURE}{}{Advanced}{ban}
  This card activates the module that calculates and prints out bandstructure.  This card does not work in the current version of {\feff}, and you may get an error if you try to use it.  The authors really want you to have your spaghetti, but they're still in the kitchen
  working on it!  Keep fingers crossed ...
\begin{verbatim}
  * give me a bandstructure.  Will likely acquire more options in the future
  BANDSTRUCTURE
\end{verbatim}
\end{Card}



\begin{Card}{STRFAC}{eta gmax rmax}{Advanced}{str}
  This card gives three non-physical internal parameters for the calculation of the KKR structure factors : the Ewald parameter and a multiplicative cutoff factor for
  sums over reciprocal (gmax) and real space (rmax) sums.  Multiplicative means the code makes a 'smart' guess of a cutoff radius, but if one suspects something fishy
  is going on, one can here e.g. use gmax=2 to multiply this guess by 2.  Eta is an absolute number.
  Given the stability of the Ewald algorithm, it shouldn't be necessary to use this card.  Its use is not recommended.
  Only active in combination with the RECIPROCAL card.
\begin{verbatim}
  STRFAC 0.4 2.0 2.0
\end{verbatim}
\end{Card}



\subsection{PATH: Path Enumeration}
\label{sec:Path-enum-modul}


\begin{Card}{RPATH}{rpath}{Standard}{rpa}
  The RPATH card determines the maximum effective (half-path)
  distance, \texttt{rpath}, of a given path. RPATH is equivalent to
  the RMAX card in the {\feff}7 code.
  Typically \texttt{rpath} is needed for EXAFS calculations only to
  set limits on the number of calculated paths. Note that
  \texttt{rpath} is one-half of the total path length in
  multiple-scattering paths. Setting this too large can cause the
  heap in the pathfinder to fill up. Default is \texttt{rpath} = 2.2
  times the nearest neighbor distance. Since the multiple scattering
  expansion is unstable close to the absorption edge, the path (MS)
  expansion should be used only for EXAFS calculations or for
  diagnosing the XANES or $\ell$DOS calculations. If you use FMS for XANES
  calculations, better results are obtained without the MS
  contribution. For EXAFS analysis this card is extremely useful,
  since it cuts off long paths which contribute only at high
  R values in R-space.
\begin{verbatim}
  * include MS paths with effective length up to 5.10 Ang
  RPATH     5.10
\end{verbatim}
\end{Card}



\begin{Card}{NLEG}{nleg}{Useful}{nle}
  The NLEG card limits the number of legs of each scattering path to
  \texttt{nleg}. If \texttt{nleg} is set to 2, only single scattering
  paths are found. The default is nleg = 8.

\begin{verbatim}
  * only single scattering paths (i.e. 2 legged paths)
  NLEG 2
\end{verbatim}
\end{Card}



\begin{Card}{PCRITERIA}{pcritk pcrith}{Advanced}{pcr}
  These criteria, like those described in the CRITERIA card, also
  limit the number of paths. However, they are applied in the
  pathfinder and eliminate unimportant paths while the pathfinder is
  doing its search. The pathfinder criteria do not know the
  degeneracy of a path and are therefore much less reliable than the
  curved wave and plane wave criteria in the \htmlref{CRITERIA}{card:cri} 
  card below. These path finder criteria (keep and heap) are turned off 
  by default, and we recommend that they be used only with very large
  runs, and then with caution.

  The keep-criterion \texttt{pcritk} looks at the amplitude of $\chi$ (in the plane wave
  approximation) for the current path and compares it to a single scattering
  path of the same effective length. To set this value, consider the
  maximum degeneracy you expect and divide your plane wave criterion
  by this number. For example, in fcc Cu, typical degeneracies are
  196 for paths with large r, and the minimum degeneracy is 6. So a
  keep criterion of 0.08\% is appropriate for a pw criteria of 2.5\%.

  The heap-criterion \texttt{pcrith} filters paths as the pathfinder 
  puts all paths into a heap (a partially ordered data structure), then 
  removes them in order of increasing total path length. Each path that 
  is removed from the heap is modified and then considered again as part 
  of the search algorithm. The heap filter is used to decide if a path has
  enough amplitude in it to be worth further consideration. If a path can
  be eliminated at this point, entire trees of derivative paths can
  be neglected, leading to enormous time savings. This test does not
  come into play until paths with at least 4 legs are being
  considered, so single scattering and triangular (2 and 3 legged)
  paths will always pass this test. Because only a small part of a
  path is used for this criterion, it is difficult to predict what
  appropriate values will be. To use this (it is only necessary if
  your heap is filling up, and if limiting rpath doesn't help), study
  the results in \file{crit.dat} from runs with shorter rpath and experiment
  with the heap criterion accordingly. In the future, we hope to
  improve this filter.

  Before using these criteria, study the output in the file
  \file{crit.dat} (use print option 1 for \module{path}
  \begin{latexonly}
    , see Table~\ref{tab:printlevels}
  \end{latexonly}
  ), 
  which has the values of \texttt{critpw}, keep criterion, and heap criterion 
  for all paths which pass the \texttt{critpw} filter.

  Default: If this card is omitted, the keep and heap criteria are set
  to zero, that is, no filtering will be done at this step in the
  calculation.

\begin{verbatim}
  * fcc Cu had degeneracies from 6 to 196, so correct for this by
  * dividing pw-crit of 2.5% by 30 to get 0.08 for keep crit. Check this
  * empirically by running with pcrits turned off and studying crit.dat.
  * After studying crit.dat, choose 0.5 for heap crit.
  PCRITERIA   0.08  0.5
\end{verbatim}
\end{Card}



\begin{Card}{SS}{index ipot deg rss}{Advanced}{ss}
  The SS card can {\it only} be used with the \htmlref{OVERLAP}{card:ove} card
  when the atomic structure is unknown, but the distance and coordination
  numbers are known, and one wants to generate an approximate EXAFS 
  contribution. Thus the pathfinder cannot be used in this
  case. Instead, the user has to specify explicitly the single
  scattering paths and their degeneracy. The OVERLAP card must be used
  to construct the potentials for use with the SS card. The
  parameters are:
\begin{description}
\item[\texttt{index}]\hfill\\
  The shell index and label used for the \file{feffNNNN.dat} file name.
\item[\texttt{ipot}]\hfill\\
  The unique potential index identifying the unique potential 
  of the scattering atom.
\item[\texttt{deg}]\hfill\\
  The degeneracy (or multiplicity) of the single scattering path.
\item[\texttt{rss}]\hfill\\ 
  The distance to the central atom.
\end{description}
  This information is used to write the file \file{paths.dat} and is not
  needed when the \htmlref{ATOMS}{card:ato} card is used. To generate single
  scattering paths with ATOMS, use \htmlref{NLEG}{card:nle} 2.
\begin{verbatim}
  *  index  ipot   deg  rss    generate single scattering results
  SS   29     1     48  5.98       parameters for 19th shell of Cu
\end{verbatim}
\end{Card}



\begin{Card}{SYMMETRY}{isym}{Advanced}{sym}
  {\feff} uses symmetry to reduce the list of paths.  This process is automated and takes into account
  both the properties of the system and of the spectroscopy (eg., time reversal symmetry, ...).  However
  sometimes one may wish to control the type of symmetry directly using the SYMMETRY card with isym ranging from 1 to 7.
  \begin{enumerate}
     \item  any path rotation, reflection and reversal are allowed
     \item  any rotation around evec, reflections in planes normal and parallel to evec, path reversal
     \item  reflections in 2 planes (xivec, evec) and (xivec, B field)
     \item  rotations around xivec, path reversal
     \item  rotations around spin axis
     \item  only 180 degrees rotation around spin axis
     \item  no symmetry operations
  \end{enumerate}
\begin{verbatim}
  * force using no symmetry at all - warning : long and slow path list!
  SYMMETRY 7
\end{verbatim}
\end{Card}



\subsection{GENFMT: XAFS Parameters}
\label{sec:Calc-contr-from}



\begin{Card}{CRITERIA}{critcw critpw}{Useful}{cri}
  Since the number of multiple scattering paths gets large very
  quickly, it is necessary to eliminate as many paths as possible.
  Fortunately, we have found that most multiple scattering paths have
  small amplitudes and can be neglected. Various cutoff criteria
  are used in {\feffcur} to limit the number of paths to consider. These
  criteria are based on the importance of the path, defined as the
  integral over the full energy range of $\chi(k)\cdot\mathtt{dk}$.
  Very close to the edge these cutoff criteria should be examined
  with care and in some cases reduced from the values used for EXAFS.
 \begin{description}
  \item[\texttt{critcw}]\hfill\\ 
  This is the cutoff for a full curved wave calculation. A
  typical curved wave calculation requires a complete spherical wave
  calculation, which typically takes seconds of CPU time per path.
  The default value of \texttt{critcw} is 4\%, meaning that any path
  with mean amplitude exceeding 4\% of the largest path will be used in
  the calculation of $\chi$. The criterion \texttt{critcw} is used by 
  \module{genfmt}.
  Since the XAFS parameter calculation is already done, the savings is
  not in computation time, but in disk space and ease of analysis. The
  values of critcw for each path are written in the file \file{list.dat}
  written by module \module{genfmt}.

  \item[\texttt{critpw}]\hfill\\ 
  This is a plane-wave approximation to $\chi$. This is
  extremely fast to calculate, and is used in the pathfinder. The
  default value of critpw is 2.5, meaning that any path with mean
  amplitude exceeding 2.5\% of the largest path, including degeneracy 
  factors, (in plane wave approximation) will be kept. Any path that does 
  not meet this criterion will not be written to \file{paths.dat}, and 
  there is no need to calculate the XAFS parameters for this path. The
  default for \texttt{critpw} is less than that for \texttt{critcw}
  since some paths are more important when the full curved wave
  calculation is done than they appear in the plane wave
  approximation. Since the plane wave estimate is extremely fast, use
  this to filter out as many paths as you can. The file
  \file{crit.dat} (written by the module \module{path}) tells you
  \texttt{critpw} for each path that passes the criterion.
 \end{description}
  The method of calculation of these importance factors has been
  improved for {\feffcur}, so don't worry if the values for
  some paths have changed slightly from previous versions. (Default
  values critcw=4\% critpw=2.5\%)

\begin{verbatim}
  CRITERIA  6.0  3.0   * critcw 6%, critpw 3%
  CRITERIA  0    0     * use all paths (cw and pw criteria turned off)
\end{verbatim}
\end{Card}



\begin{Card}{IORDER}{iord}{Advanced}{ior}
  Order of the approximation used in module \module{genfmt}. {\feff} uses
  order 2 by default, which is correct to terms of order $1/(pR)^2$,
  and corresponds to 6x6 scattering matrices in the Rehr--Albers
  formalism. Single scattering is calculated exactly to this order.
  The 6x6 approximation is accurate to within a few percent in every
  case we have tried (that is, higher order doesn't change the result
  more than a few percent). However $M_{\mathit{IV}}$ shells and
  higher shells may require increased iorder for coupling the matrix
  elements. Changing the default values requires some familiarity
  with the Rehr--Albers paper and the structure of the module \module{genfmt}.
  To do so, follow the instructions in the {\feff} source code in
  subroutine \texttt{setlam}. The key \texttt{iord} is passed to
  \texttt{setlam} for processing. You may need to change the code
  parameter \texttt{lamtot} if you want to do higher order
  calculations. For details of the algorithm used by \module{genfmt}, see the
  paper by J.J.\ Rehr and R.C.\ Albers
  \begin{latexonly}
    (see the references in Appendix~\ref{sec:Append-C-Refer})
  \end{latexonly}
  . For the
  $M_{\mathit{IV}}$ and higher edges, you may receive an error
  message like: \texttt{Lambda array overfilled}. In that case the
  calculations should be repeated with IORDER -70202 (10x10 matrices).
\begin{verbatim}
  * change iorder for M4 calculations
  IORDER -70202
\end{verbatim}
\end{Card}



\begin{Card}{NSTAR}{}{Advanced}{nst}
  When this card is present, \module{genfmt} writes the file
  \file{nstar.dat} with the effective coordination number $N^\star$
  which is the coordination number weighted by $\cos^2(\theta)$ to
  correct for polarization dependence in SEXAFS calculations.
\end{Card}



\subsection{FF2X: XAFS Spectrum}
\label{sec:Comb-contr-from}

\begin{latexonly}
\begin{Card}{DEBYE}{temperature  Debye-temperature [idwopt]}{Standard}{deb2}
  The DEBYE card is used to calculate Debye-Waller (DW) factors to account for thermal disorder.
  The effect of the card depends somewhat on the situation.  
  
  When used with the MS path expansion (EXAFS and RPATH cards), a DW factor is calculated for each
  path using the correlated Debye Model. They will be summed with the 
  DW factors from the \htmlref{SIG2}{card:sig} card and from the \file{list.dat} 
  file, if present - so the total DW factor for any path is the sum of these three factors!
  
  When used with the FMS technique (XANES and FMS cards), a single-scattering DW factor is calculated for each pair of atoms in the cluster and a factor $\exp(-\sigma^2 k^2)$ is multiplied to the free propagator.  This approach is inexact beyond single scattering paths.  However, it is likely adequate in the near-edge region where FMS is valid, since the shorter mean free path makes this region less sensitive to thermal effects.
  
  By default, $\mathtt{idwopt}=0$ specifies the correlated Debye model 
 for Debye--Waller factors. It is best suited for
  homogeneous systems, where it is quite accurate. Caution: in
  heterogeneous systems the model only gives approximate values which
  can easily be off by factors of two or more.
  Note that the DEBYE card is incompatible with the 
  \htmlref{CFAVERAGE}{card:cfa} card for options other than the correlated Debye 
  model. Temperatures are specified in Kelvin.

\begin{verbatim}
  *Debye-Waller factors for Cu at 190K with correlated Debye Model
  DEBYE  190 315
\end{verbatim}


   Several additional models for
  calculating DW factors are available in {\feffcur} based on the information
  about the harmonic force constants in the material.  The \htmlref{DEBYE}{car:deb2} card offers a choice between 5 different models for the DW factors:
\begin{itemize}
				\item  0 	Correlated-Debye method  (default) (CD)
				\item  1	 Equations of Motion method (EM)
				\item  2	 Recursion method (RM)
				\item  3	 Classical Correlated-Debye method (CCD)
				\item  4	 Read from "sig2.dat" file
				\item  5	 Dynamical-Matrix method (DM)
\end{itemize}  
  $\mathtt{idwopt}=1$
  means the equation of motion (EM) method is used to get Debye--Waller
  factors and $\mathtt{idwopt}=2$ means the recursion method (RM) which
  is an improved correlated Einstein model. Both methods are faster than 
  molecular dynamics simulations, and the recursion method is much faster 
  than the equation of motion method. However, the equation of motion method 
  leads to somewhat more accurate results than the recursion method. These 
  additional methods seem to be superior to the correlated Debye model in cases 
  with tetrahedral coordination, such as solid Ge and many biological materials. 
  $\mathtt{idwopt}=5$ uses ab initio DW factors from a dynamical matrix using the \module{dmdw} module.  This provides very accurate DW factors.
  All these methods are described in more detail in \htmlref{Chapter 3}{ref:DWfactors}.

\begin{verbatim}
  * Calculate Debye-Waller factors for Cu at 190K with equation of motion
  DEBYE  190 0 1
\end{verbatim}
\end{Card}
\end{latexonly}



\begin{Card}{ABSOLUTE}{}{Useful}{abs}
  By default {\feff} normalizes spectra at 40 eV above threshold.  This card disables normalization.  ABSOLUTE is automatically
  set whenever ELNES,  EXELFS, or NRIXS cards are used.
\begin{verbatim}
  ABSOLUTE    * don't normalize spectra in ff2x
\end{verbatim}
\end{Card}



\begin{Card}{CORRECTIONS}{vrcorr vicorr}{Useful}{cor}
  The real energy shift \texttt{vrcorr} moves $E_0$ in the final $\chi(k)$ and the
  imaginary energy shift \texttt{vicorr} adds broadening to the result. The real
  energy shift is useful to correct the error in {\feff}'s Fermi
  level estimate and the imaginary part can be used to correct
  for experimental resolution or errors in the core-hole lifetime.
  This error in the Fermi level is typically about 1 eV with self-consistent
  calculations and about 3 eV with overlapped atom potentials. The imaginary
  energy is typically used to correct for instrument broadening or as
  a correction to the mean free path calculated by {\feff}. This
  affects only the module \module{ff2x}, which combines the results in all of
  the \file{feffNNNN.dat} files. This card is useful in fitting loops
  because you can simply make such energy corrections and see the results
  without redoing the entire XAFS parameter calculation. Caution: the
  results are not as accurate as those obtained with the EXCHANGE card.
  Both energies are in eV.
  (See also the \htmlref{EXCHANGE}{card:exc} card
  \begin{latexonly}
    in Section~\ref{sec:Scatt-potent-modul}
  \end{latexonly}
  ).
\begin{verbatim}
  * Reduce E0 by 3.0 eV and add 1 eV of broadening (full width)
  * This will only affect module ff2x
  CORRECTIONS   3.0   1.0       real shift, imag shift
\end{verbatim}
\end{Card}



\begin{Card}{SIG2}{sig2}{Useful}{sig}
  Specify a global Debye--Waller factor to be used or added to
  Debye--Waller calculations (see the \htmlref{DEBYE}{card:deb2} card) for all paths. This
  value will be summed with the correlated Debye model value (if the
  DEBYE card is present) and any value added to \file{list.dat}. Units are
  \AA$^2$. This card can be used, for example to add Debye--Waller
  factors from structural disorder.

\begin{verbatim}
  SIG2 0.001    add 0.001 globally to all DW factors
\end{verbatim}
\end{Card}



\begin{Card}{SIG3}{AlphaT ThetaE}{Useful}{si3}
  Documentation currently unavailable. FIX
\end{Card}



\begin{Card}{MBCONV}{}{Advanced}{mbc}
  Many-body convolution.  Documentation currently unavailable. FIX
\end{Card}



\begin{latexonly}
\subsection{SFCONV: Spectral Function Convolution} FIX
\label{sec:Sfconv-modul}
\end{latexonly}



\begin{Card}{SFCONV}{}{Useful}{sfc}
  SFCONV convolutes the single particle XAS files (\file{xmu.dat}, 
  \file{chi.dat}, \file{feffNNNN.dat}, etc.) from {\feff} with a many 
  body spectral function to include many body effects in the spectra. 
  This includes an ab-initio calculation of the amplitude reduction 
  factor S$_0^2$. SFCONV uses Hedin-Lundqvist self energies, so it 
  will probably work best with XAS files calculated using {\feff}'s 
  HL self energies.
\begin{verbatim}
SFCONV
\end{verbatim}
\end{Card}



\begin{Card}{RCONV}{cen cname}{Advanced}{rco}
  Documentation currently unavailable. FIX
\end{Card}



\begin{Card}{SELF}{}{Advanced}{sel}
  Print out on-shell self energy as calculated by this module.
\begin{verbatim}
SELF
\end{verbatim}
\end{Card}



\begin{Card}{SFSE}{k0}{Useful}{sfs}
Print out off-shell self energy $\Sigma(k0,E)$.
\begin{verbatim}
SFSE
\end{verbatim}
\end{Card}




\subsection{COMPTON: Compton scattering} 
\label{sec:Compton-modul}


\begin{Card}{CGRID}{[zpmax ns nphi nz nzp]}{Useful}{cgr}
  Sets the grid on which rho(z,z') is evaluated for Compton scattering.  zpmax is the upper limit on z'.  The other four parameters set the density of the spatial grid.
\begin{verbatim}
CGRID 10 32 32 32 120 
\end{verbatim}
\end{Card}

\begin{Card}{RHOZZP}{}{Useful}{rho}
  Calculate a slice rho(z,z') with z fixed near the origin and z' moving outward along the z-axis to zpmax (set in CGRID).  The result is saved in \file{rhozzp.dat}.  Useful as a test, or to get a quick estimate of some parameters in the other COMPTON cards.  Only to be used in conjunction with the COMPTON card.
\begin{verbatim}
RHOZZP
\end{verbatim}
\end{Card}


\subsection{EELS: Electron Energy Loss Spectrum}
\label{sec:Eels-modul}
  This module assembles an EELS spectrum as specified by the card ELNES 
  or EXELFS. See Section~\ref{sec:EELS} for a more complete description of EELS calculations.



\begin{Card}{MAGIC}{emagic}{Useful}{mag}
  The MAGIC card makes {\feff} calculate cross sections as a function of 
  collection angle at a particular energy loss. So, in addition to the 
  energy-resolved output produced by the EELS cards, the MAGIC card produces 
  angle-resolved output.

  This card enables one to 
  find the magic angle for a material very quickly : run {\feff} for two 
  different orientations, and see where the sp$^2$-curves cross. This gives 
  you the magic collection angle for the convergence angle and EELS edge chosen 
  in \file{feff.inp}.  Plus the entire workings of the \module{eels} module are
  based on black magic, of course.

\begin{verbatim}
* create a plot that shows the sp^2 ratio at 20 eV above threshold.
MAGIC 20
\end{verbatim}
\end{Card}



%%%%%%%%%%%%%%%%%%%%%%%%%%%%%%%%%%%%%%%%%%%%%%%%%%%%%%%%%%%%%%%%%%%%%%%%%%%%%%%%%%%%%% FILES %%%%%%%%%%%%
%%%%%%%%%%%%%%%%%%%%%%%%%%%%%%%%%%%%%%%%%%%%%%%%%%%%%%%%%%%%%%%%%%%%%%%%%%%%%%%%%%%%%% FILES %%%%%%%%%%%%
\begin{latexonly}% ends at end of doc

\chapter{Input and Output Files}
\label{sec:Input-and-Output-Files}

Any {\feff} calculation produces a large number of files that may be confusing to the novice.  
The most important files are the master input file \file{feff.inp}, which we discussed at length in the previous chapters ;
the output file \file{xmu.dat}, which contains the XAS spectrum ; and the various \file{log*.dat} files, which are log files
written by {\feff} during execution.  These basic files are also readily accessible in the GUI, where the input options corresponding to
\file{feff.inp} are set in the interface; the log information scrolls by in a pop-up screen after {\feff} is launched ; and the XAS spectrum
is displayed by selecting 'plot'.

The GUI plotter can also display other output files ; however, for full flexibility the use of a plotting package like gnuplot is recommended.  The data files use a standard format and your favorite plotting package is likely able to read them.

{\feff} creates a variety of output files depending on the spectroscopy type selected by the user, and depending on verbosity settings.
See the \htmlref{PRINT}{card:pri} card in Section~\ref{sec:General-Cards} and \ref{table:printlevels} to obtain various diagnostic files.

{\feff} also writes several intermediate files.  Some of these group relevant settings from \file{feff.inp} and program defaults for each of 
{\feff}'s program modules.  These files, called \file{pot.inp}, \file{xsph.inp} and so on, can be manipulated by the expert user who knows what she is doing; see Section~\ref{sec:Addit-progr-contr}.
However we advise that novice users always work through \file{feff.inp} to avoid mistakes. 

Section~\ref{sec:File-structure-code} summarizes data flow.  The rest of this section describes selected files in more detail.

Regrettably this chapter is somewhat out of date and a handful of newer files are not described.  This includes NRIXS and COMPTON output.

\newpage

%% the use of \vspace{-4ex} in this table is a crufty hack to avoid
%% having to use the array package
\begin{table}[htbp] \label{table:printlevels}
    %% caption moved here for table visibility (links go to the caption)
    \caption[Print levels]{Print levels controlling output files from the modules.  Log-files are not included.
      For ELNES calculations, a proliferation of files may occur, eg. \file{xmu.dat} through \file{xmu09.dat} will be produced.
      NRIXS calculations not included in this table. \module{rdinp} not included.}
    \begin{center}
    \begin{tabular}[h]{p{0.1\linewidth}p{0.8\linewidth}}
          \hline\hline
      module & \hspace{5em} print levels\\
      \hline
      %%
      \module{atomic} &
      \vspace{-4ex}
      \begin{itemize}
        \tightlist
      \item[0] write \file{apot.bin} and \file{fpf0.dat}
      \end{itemize}\\
%%
      \module{pot} &
      \vspace{-4ex}
      \begin{itemize}
        \tightlist
      \item[0] write \file{pot.bin} only
      \item[1] add \file{misc.dat}
      \item[2] add \file{potNN.dat}
      \item[3] add \file{atomNN.dat}
      \end{itemize}\\
      %%
      \module{screen} &
      \vspace{-4ex}
      \begin{itemize}
        \tightlist
      \item[0] write \file{wscrn.dat} if COREHOLE RPA used
      \end{itemize}\\
      %%
      \module{ldos} &
      \vspace{-4ex}
      \begin{itemize}
        \tightlist
      \item[0] write \file{ldosNN.dat} if LDOS used
      \end{itemize}\\
      %%
      \module{xsph} &
      \vspace{-4ex}
      \begin{itemize}
        \tightlist
      \item[0] write \file{phase.bin} and \file{xsect.bin} only
      \item[1] add \file{axafs.dat} and \file{phase.dat}
      \item[2] add \file{phaseNN.dat} and \file{phminNN.dat}
      \item[3] add \file{ratio.dat} (for XMCD normalization) and \file{emesh.dat}.  Careful!  xsph now takes a lot of time.
      \end{itemize} \\
      %%
      \module{fms} &
      \vspace{-4ex}
      \begin{itemize}
        \tightlist
      \item[0] write \file{gg.bin}
      \item[1] write \file{gg.dat}
      \end{itemize}\\
      %%
      \module{mkgtr} &
      \vspace{-4ex}
      \begin{itemize}
        \tightlist
      \item[0] write \file{fms.bin} and \file{gtr.dat}
      \end{itemize}\\
      %%
      \module{path} &
      \vspace{-4ex}
      \begin{itemize}
        \tightlist
      \item[0] write \file{paths.dat} only
      \item[1] add \file{crit.dat}
      \item[3] add \file{fbeta} files (plane wave $|f(\beta)|$ approximations)
      \item[5] Write only \file{crit.dat} and do not write \file{paths.dat}.
        (This is useful when exploring the importance of paths for large runs.)
      \end{itemize}\\
      %%
      \module{genfmt} &
      \vspace{-4ex}
      \begin{itemize}
        \tightlist
      \item[0] write \file{list.dat}, and write \file{feff.bin} with all paths with importance greater than or equal to two thirds of the curved wave importance criterion
      \item[1] write all paths to \file{feff.bin}
      \end{itemize}\\
      %%
      \module{ff2x} &
      \vspace{-4ex}
      \begin{itemize}
        \tightlist
      \item[0] write \file{chi.dat} and \file{xmu.dat}
      \item[2] add \file{chipNNNN.dat} ($\chi(k)$ for each path individually)
      \item[3] add \file{feffNNNN.dat} and \file{files.dat}, and do not add \file{chipNNNN.dat} files
      \end{itemize}\\
      %%
      \module{sfconv} &
      \vspace{-4ex}
      \begin{itemize}
        \tightlist
      \item[0] overwrites \file{xmu.dat}
      \end{itemize}\\
      %%
      \module{eels} &
      \vspace{-4ex}
      \begin{itemize}
        \tightlist
      \item[0] write \file{eels.dat}; write \file{magic.dat} if MAGIC card used
      \end{itemize}\\
      %%
      \module{compton} &
      \vspace{-4ex}
      \begin{itemize}
        \tightlist
      \item[0] \file{compton.dat}, \file{rhozzp.dat},\file{jzzp.dat},\file{jpq.dat}
      \end{itemize}\\      
      \hline\hline
    \end{tabular}
    \label{tab:printlevels}
    \end{center}
\end{table}

\newpage

\section{Data flow}
\label{sec:File-structure-code}

\begin{description}
  %%
  %% Module 0
\item[\large\textbf{Module 0}]\dotfill\  {\large\module{rdinp}}
  \begin{description}
  \item[\textbf{Purpose of Module:}] Process input data
  \item[\textbf{Input files:}] \file{feff.inp}
  \item[\textbf{Output files:}] \file{geom.dat}, \file{global.inp},
       \file{pot.inp},\file{xsph.inp},\file{fms.inp},\file{path.inp},\file{genfmt.inp},
       \file{ff2x.inp},\file{ldos.inp}, and \file{eels.inp}
  \item[\textbf{Other output:}] \file{paths.dat} (only if the SS card is used)
  \item[\textbf{Description:}] Reads the \file{feff.inp} file, makes
    appropriate operations on the data, and writes the resulting
    information into several intermediate files, which
    contain formatted data needed for all modules.
  \end{description}
  %%
  %% Module 1
\item [\large\textbf{Module 1}]\dotfill\  {\large\textrm{atomic}}
  \begin{description}
  \item[\textbf{Purpose of Module:}] Calculate atomic
    potentials for the photoelectron
  \item[\textbf{Input files:}] \file{pot.inp} and \file{geom.dat}
  \item[\textbf{Output files:}] \file{apot.bin}
  \item[\textbf{Other output:}] \file{fpf0.dat}
  \item[\textbf{Description:}] Reads \file{pot.inp} and calculates atomic potentials
    for the photoelectron, which are written into \file{apot.bin}.
  \end{description}
  %%
  %% Module 2
\item [\large\textbf{Module 2}]\dotfill\  {\large\textrm{pot}}
  \begin{description}
  \item[\textbf{Purpose of Module:}] Calculate embedded atomic
    potentials for the photoelectron
  \item[\textbf{Input files:}] \file{pot.inp},\file{apot.bin} and \file{geom.dat}
  \item[\textbf{Output files:}] \file{pot.bin}
  \item[\textbf{Other output:}] diagnostic files (see
    Table~\ref{tab:printlevels} on page \pageref{tab:printlevels})
  \item[\textbf{Description:}] Reads \file{pot.inp} and calculates potentials
    for the photoelectron, which are written into \file{pot.bin}.
    Optionally, \module{pot} will write other diagnostic files with information
    about the potentials.
  \end{description}
  %%
  %% Module 3
\item [\large\textbf{Module 3}]\dotfill\  {\large\textrm{screen}}
  \begin{description}
  \item[\textbf{Purpose of Module:}] Calculate embedded atomic
    potentials for the photoelectron
  \item[\textbf{Input files:}] \file{pot.bin} and, optionally, \file{screen.inp}
  \item[\textbf{Output files:}] \file{wscrn.dat}
  \item[\textbf{Description:}] Calculates screening, which is written into \file{wscrn.dat}.
  \end{description}
  %%
  %% Module 4
\item [\large\textbf{Module 4}]\dotfill\  {\large\textrm{opconsat}}
  \begin{description}
  \item[\textbf{Purpose of Module:}] Unknown
  \item[\textbf{Input files:}] Unknown
  \item[\textbf{Output files:}] Unknown
  \item[\textbf{Description:}] Unknown
  \end{description}
    %%
  %% Module 5
\item[\large\textbf{Module 5}]\dotfill\  {\large\module{ldos}}
  \begin{description}
  \item[\textbf{Purpose of Module:}] Calculate LDOS
  \item[\textbf{Input files:}] \file{ldos.inp}, \file{geom.dat}, 
    and \file{pot.bin}.
  \item[\textbf{Output files:}] \file{ldosNN.dat} ($\ell$DOS) and 
    \file{logdos.dat}
  \item[\textbf{Other output:}] 
  \item[\textbf{Description:}] LDOS runs only if the LDOS card is 
    present in \file{feff.inp}. It outputs the angular momentum ($\ell$) projected DOS 
    into \file{ldosNN.dat} files, with each value of NN corresponding 
    to each unique potential.
  \end{description}
  %%
  %% Module 6
\item [\large\textbf{Module 6}]\dotfill\  {\large\module{xsph}}
  \begin{description}
  \item[\textbf{Purpose of Module:}] Calculate cross-section and phase shifts
  \item[\textbf{Input files:}] \file{xsph.inp}, \file{geom.dat},
    \file{global.dat} and \file{pot.bin}
  \item[\textbf{Output files:}] \file{phase.bin}, and \file{xsect.dat},
  \item[\textbf{Other output:}] diagnostic files (see
  Table~\ref{tab:printlevels} on page \pageref{tab:printlevels}),
    and \file{axafs.dat}.
  \item[\textbf{Description:}]  \module{xsph} writes the binary file
    \file{phase.bin}, which contains the scattering phase shifts and
    other information needed by \module{path} and \module{genfmt}. 
    The  atomic  cross-section data is written in \file{xsect.dat} and
    used in the module (\module{ff2x}) for overall normalization.
    Optionally, \module{xsph} will write other diagnostic files with 
    information about the phase shift calculations.
  \end{description}
  %%
  %% Module 7
\item [\large\textbf{Module 7}]\dotfill\  {\large\module{fms}}
  \begin{description}
  \item[\textbf{Purpose of Module:}] Calculate full multiple
  scattering for XANES, ELNES and $\ell$DOS
  \item[\textbf{Input files:}] \file{fms.inp}, \file{global.dat},
     \file{geom.dat}, and \file{phase.bin}
  \item[\textbf{Output files:}] \file{gg.bin}
  \item[\textbf{Other output:}] optionally \file{gg.dat}
  \item[\textbf{Description:}]  Performs the full multiple scattering
    algorithm.  Writes the Green's function matrix to \file{gg.bin}.
  \end{description}
  %%
  %% Module 8
\item [\large\textbf{Module 8}]\dotfill\  {\large\module{mkgtr}}
  \begin{description}
  \item[\textbf{Purpose of Module:}] Calculate full multiple
  scattering for XANES, ELNES and $\ell$DOS
  \item[\textbf{Input files:}] \file{fms.inp}, \file{global.dat},
     \file{geom.dat}, \file{gg.bin} and \file{phase.bin}
  \item[\textbf{Output files:}] \file{fms.bin}
  \item[\textbf{Other output:}] optionally \file{gtr.dat}
  \item[\textbf{Description:}]  Traces the Green's function matrix and adds in
  matrix elements.  Output written to \file{fms.bin}.
    Writes output into \file{fms.bin} for the \module{ff2x} module, which
    contains the $\chi(k)$ from \module{fms}.

    If an ELNES/EXELFS card is present, all of the requested components 
    of the sigma tensor are written to \file{fms.bin}, instead of just one. 
  \end{description}
  %%
  %% Module 9
\item [\large\textbf{Module 9}]\dotfill\  {\large\module{path}}
  \begin{description}
  \item[\textbf{Purpose of Module:}] Path enumeration
  \item[\textbf{Input files:}] \file{path.inp}, \file{geom.dat},
     \file{global.dat} and \file{phase.bin}
  \item[\textbf{Output files:}] \file{paths.dat}
  \item[\textbf{Other output:}] \file{crit.dat}
  \item[\textbf{Description:}] \module{path} writes \file{paths.dat} for use 
    by \module{genfmt} and as a complete description of each path for use of 
    the user. \module{path} will optionally write other diagnostic files. 
    The file \file{crit.dat} is particularly useful when studying 
    large numbers of paths. When studying large numbers of paths, 
    this module will optionally write only \file{crit.dat} and 
    not \file{paths.dat}.

    If an ELNES/EXELFS card is present, a separate \file{list.dat} file is 
    written for each polarization component (i.e., \file{list.dat}, 
    \file{list02.dat}, etc.).
  \end{description}
  %%
  %% Module 10
\item [\large\textbf{Module 10}]\dotfill\  {\large\module{genfmt}}
  \begin{description}
  \item[\textbf{Purpose of Module:}] Calculate scattering amplitudes and other
    XAFS parameters
  \item[\textbf{Input files:}] \file{genfmt.inp},
    \file{global.dat}, \file{phase.bin}, and \file{paths.dat}
  \item[\textbf{Output files:}] \file{feff.bin}, and \file{list.dat}
  \item[\textbf{Other output:}]
  \item[\textbf{Description:}] \module{genfmt} reads input files, and writes a file
    \file{feff.bin}, which contains all the EXAFS information for the
    paths, and \file{list.dat}, which contains some basic information
    about them. These files are the main output of {\feff} for EXAFS
    analysis. To read \file{feff.bin} into your own program, use the 
    subroutine feffdt as an example.

    If an ELNES/EXELFS card is present, a \file{listNN.dat} file is 
    written for each polarization component, and a separate \file{feff.bin} 
    file is written (i.e., \file{feff.bin}, \file{feff02.bin}, etc.). 
    The format of the files is unchanged.
  \end{description}
  %%
  %% Module 11
\item [\large\textbf{Module 11}]\dotfill\  {\large\module{ff2x}}
  \begin{description}
  \item[\textbf{Purpose of Module:}] Calculate specified x-ray spectrum
  \item[\textbf{Input files:}] \file{ff2x.inp}, \file{global.dat},
    \file{list.dat}, \file{feff.bin}, \file{fms.bin}, \file{xsect.bin}
  \item[\textbf{Output files:}] \file{chi.dat} and \file{xmu.dat}
  \item[\textbf{Other output:}] \file{chipNNNN.dat} and \file{feffNNNN.dat}
  \item[\textbf{Description:}] \module{ff2x} reads \file{list.dat},
    \file{fms.bin}, \file{feff.bin}, and writes \file{chi.dat}
    with the total XAFS from the paths specified in \file{list.dat}.
    Additional instructions are passed to \module{ff2x} from \file{feff.bin}, so you
    can change S02, the Debye temperature and some other parameters
    without re-doing the whole calculation. The file \file{list.dat}
    can be edited by hand to change the paths being considered, and
    individual \file{chipNNNN.dat} files with $\chi(k)$ from each path are
    optionally written. If any of the \htmlref{XANES}{card:xan}, 
    \htmlref{DANES}{card:dan}, \htmlref{FPRIME}{card:fpr} or 
    \htmlref{XNCD}{card:xnc} cards are specified, \module{ff2x} will write 
    the corresponding calculated data in \file{xmu.dat}. Various 
    corrections are possible at this point in the calculations---
    see the input cards above.

    If an ELNES/EXELFS card is present, this module reads the large  
    \file{fms.bin} and all the \file{feffNN.bin} files, and produces 
    a \file{xmuNN.dat} file containing the corresponding component of 
    the sigma tensor (\file{xmu.dat}, \file{xmu02.dat}, ..., \file{xmu09.dat}). 
    Those files have the traditional \file{xmu.dat} format. Similarly, 
    \file{chiNN.dat} files are produced.
  \end{description}
  %%
  %%
  %% Module 12
\item[\large\textbf{Module 12}]\dotfill\  {\large\module{sfconv}}
  \begin{description}
  \item[\textbf{Purpose of Module:}] Convolve output files with the 
    spectral function.
  \item[\textbf{Input files:}] \file{sfconv.inp}, \file{xmu.dat}, 
  \file{chi.dat}, \file{chipNNNN.dat}, or \file{feffNNNN.dat} files 
  \item[\textbf{Output files:}] \file{specfunct.dat}, 
  \item[\textbf{Other output:}] The following files are overwritten 
    with convolved spectral data: \file{xmu.dat}, \file{chi.dat}, 
    \file{chipNNNN.dat}, and \file{feffNNNN.dat}.
  \item[\textbf{Description:}] SFCONV convolutes the single particle 
    XAS files with a many body spectral function to include many body 
    effects on the spectra, including an ab-initio calculation of the 
    amplitude reduction factor S$_0^2$.  This module runs after
    \module{ff2x} if the SFCONV card is present in \file{feff.inp}.

    In the presence of an EELS card, the module is run for all 
    components of the sigma tensor: each \file{xmu.dat} file is 
    opened and altered individually.
  \end{description}
  %%
  %% Module 13
\item[\large\textbf{Module 13}]\dotfill\  {\large\module{eels}}
  \begin{description}
  \item[\textbf{Purpose of Module:}] Calculate EELS
  \item[\textbf{Input files:}] \file{eels.inp} 
  \item[\textbf{Output files:}] \file{eels.dat} and 
    \file{logeels.dat}
  \item[\textbf{Other output:}] optionally, \file{magic.dat}
  \item[\textbf{Description:}] EELS runs only in the presence of an 
    ELNES or EXELFS card. It reads \file{eels.inp}, and sums the partial spectra 
    from the various polarizations to assemble a physical EELS spectrum. 
  \end{description}

\end{description}



%\section{{\FEFF} File Reference}
\section{File Reference}
\label{sec:Descr-Outp-Files}

\subsection{Main Output Data}
\label{sec:Final-Results-Calc}

\begin{description}
\item[\file{chi.dat}]\hfill\\ Standard XAFS data containing $k$,
  $\chi(k)$, $|\chi(k)|$ relative to threshold ($k=0$). The header
  also contains enough information to specify which model was used to
  create this file.
\item[\file{xmu.dat}]\hfill\\ The file \file{xmu.dat} contains XANES, EXAFS or NRIXS
   data depending on the situation; $\mu$, $\mu_0$, and 
  $\tilde \chi = \chi \frac{\mu_0}{\mu_0({\rm edge} + 50 {\rm eV})}$ 
  as functions of absolute energy
  $E$, relative energy $E-E_f$ and wave number $k$.
\item[\file{feff.bin}] \hfill\\ A binary file that contains all the
  information about the XAFS from all of the paths. This replaces the
  old \file{feffNNNN.dat} files (which you can make using the PRINT
  card). If you want to use this file with your own analysis package,
  use the code in subroutine feffdt as an example of how to read it.
\item[\file{feffNNNN.dat}]\hfill\\ You have to use the PRINT option to
  obtain these files. Effective scattering amplitude and phase shift
  data, with $k$ referenced to threshold for shell nn: $k$, $\phi_c$,
  $|F_{\mathrm{eff}}|$, $\phi_{\mathrm{eff}}$, the reduction factor,
  $\lambda$, $\Re(p)$.

  If you need these, use the \htmlref{PRINT}{card:pri} option for \module{ff2x}
  greater than or equal to 3, which will read \file{feff.bin} and write the
  \file{feffNNNN.dat} files in exactly the form you're used to.
\item[\file{fpf0.dat}] \hfill\\ Thomson scattering amplitude $f_0(Q)$
  and constant contribution to f' from total energy term.
\item[\file{ratio.dat}] \hfill\\ Ratio $\mu_0(E)$, $ \rho_0(E)$ and
  their ratio versus energy, for XMCD sum rules normalization.
\item[\file{eels.dat}]\hfill\\ This contains the EELS spectrum: energy loss in 
  eV, total spectrum, contribution from each component of the cross section tensor.
\item[\file{magic.dat}]\hfill\\ This file is only written if the 
\htmlref{MAGIC}{card:mag} card is used. It contains the collection 
angle in rad, the pi to sigma ratio, the pi and sigma components 
of the spectrum, and the total spectrum. 
\end{description}


\subsection{EELS Files}
\label{sec:EELS-files}
\begin{description}
      \item[\file{eels.inp}]\hfill\\ is read by the \module{eels} module and 
        determines what {\feff} will actually do. Expert users can tweak this 
        file directly. It contains all the options of the 
        EELS and MAGIC cards. The very first 
        parameter determines whether \module{eels} is executed (=1) or not (=0). 
        The parameters on the next line select the components 
        (1-9) of the sigma tensor to be calculated.
        These parameters are very important because most other program modules check 
        for the presence of \file{eels.inp} and the values of these parameters 
        to determine their course of action. People who have done EELS and then 
        want to do something else in the same working directory may want to set 
        the execution switch to 0 (or comment the EELS card in \file{feff.inp} 
        and rerun \module{rdinp}, which amounts to the same) to make sure none 
        of the regular modules do anything special for EELS.
      \item[\file{eels.dat}]\hfill\\  contains the EELS spectrum. Its 
        first column has energy loss in eV, the second column the total 
        spectrum, and the next columns contain the contribution to the total 
        spectrum from each of the nine components of the cross section tensor (xx, xy, ..., zz)
      \item[\file{magic.dat}]\hfill\\  is only written if the MAGIC 
        card is present in feff.inp. It contains the collection angle in rad, the $\pi$ 
        to $\sigma$ ratio, the $\pi$ and $\sigma$ components of the spectrum, and the 
        total spectrum; all as a function of collection angle, evaluated at 
        the energy loss set by the MAGIC card.
      \item[\file{logeels.dat}]\hfill\\  contains reports on the 
        execution of the \module{eels} module. In particular, it contains a summary of the 
        input options used. Most of the information in the file is also 
        written to the screen during program execution.
\end{description}


\subsection{Intermediate Files}
\label{sec:Interm-Files}

\begin{description}
\item[\file{modN.inp} and \file{ldos.inp}]\hfill\\ These ASCII files contain
   basic information 
  from \file{feff.inp} for a particular module. They can still be edited,
  for example to take advantage of symmetries.
\item[\file{global.dat} ]\hfill\\ This ASCII file contains
   global information  about x-ray polarization and about
   configurational averaging.
\item[\file{geom.dat} ]\hfill\\ This ASCII file contains
   Cartesian coordinates of all atoms and first-bounce information 
   for the  degeneracy reduction in the pathfinder.
\item[\file{pot.bin}]\hfill\\ Charge density and potential (SCF or not) for
  all types of atoms. This file is used by the \module{xsph} module.
\item[\file{phase.bin}]\hfill\\ This is a binary file with the scattering
  phase shifts for each unique potential and with relativistic dipole matrix
  elements, normalized to total cross section in \file{xsect.bin}.
  It is used by the \module{fms}, \module{path} and \module{genfmt} modules.
\item[\file{xsect.bin}]\hfill\\ Total atomic cross section for x-ray
  absorption. This is an ASCII file, but it is highly sensitive to format. The 
  information it contains can be viewed, but editing this file is \emph{not} 
  recommended.
\item[\file{ldosNN.dat}]\hfill\\ $\ell$-projected density of states for the
  NN$^{\mathrm{th}}$ potential index (see the \htmlref{LDOS}{card:ldo} card)
\item[\file{fms.bin}]\hfill\\ contains the results of FMS calculations. Used
  by \module{ff2x} to get the total XAFS or XANES.
\item[\file{paths.dat}]\hfill\\ Written by the pathfinder, this is a
  description of all the paths that fit the criteria used by the
  pathfinder. It is used by \module{genfmt}. The path descriptions include
  Cartesian coordinates of atoms in the path, scattering angles, leg
  lengths and degeneracy. For details on editing this file by hand, see
  Section~\ref{sec:Addit-progr-contr}. (\file{pathNN.dat} files are also
  created during the $\ell$DOS calculations for each type of potential, but
  they are deleted after use.)
\item[\file{crit.dat}]\hfill\\ Values of the quantities tested against the
  various criteria in the pathfinder.
\item[\file{list.dat}]\hfill\\ List of paths to use for the final calculations.
  Written by \module{genfmt} when the XAFS parameters are calculated and used
  by \module{ff2x}. It contains the curved wave importance ratios, which
  you may wish to study. For details on editing this file by hand, see
  Section~\ref{sec:Addit-progr-contr}.

  The curved wave importance ratios are the importance of a particular
  path relative to the shortest single scattering path.
\item[\file{specfunct.dat}]\hfill\\ This is a binary file containing the 
  spectral function. If it is not present, \module{sfconv} will create it. Every 
  time the material is changed, \file{specfunct.dat} will be recomputed.
\end{description}



\subsection{Diagnostic Files}
\label{sec:Diagn-Files-from}

\begin{description}
\item[\file{misc.dat}]\hfill\\ Header file for quick reference.
\item[\file{phaseNN.dat}] \hfill\\ Complex phase shifts for each
  shell.
\item[\file{phminNN.dat}] \hfill\\ Real part of phase shifts for
  $\ell$=0,1,2 only. They are smaller versions of corresponding
  \file{phaseNN.dat}.
\item[\file{potNN.dat}]\hfill\\  Detailed atomic potentials and
  densities.
\item[\file{atomNN.dat}] \hfill\\ Diagnostic information on Desclaux
  free atom NN.
  \item[\file{.dimensions.dat}] \hfill\\ Passes array size between modules.  Mess with this and all
  hell will break loose.
  \item[\file{.scfconvergence-feff}] \hfill\\ Tracks charge density difference and convergence throughout SCF iterations in \module{pot}.
\end{description}


\subsection{Variables in the EXAFS and XANES Formulae}
\label{sec:Vari-EXAFS-form}
FIX move to chapter 3. add other spectroscopies.
\begin{Reflist}
\item[$k$] The wave number in units of \AA$^{-1}$.
  $k=\sqrt{E-E_f}$ where $E$ is energy and $E_f$ is the Fermi level
  computed from electron gas theory at the average interstitial charge
  density.
\item[$\chi(k)$]
  $$ \chi(k) = S_0^2  \mathcal{R}  \sum\limits_{\mathrm{shells}}
  \frac{NF_\mathrm{eff}}{kR^2} \exp(-2r/\lambda)
  \sin(2kR + \phi_{\mathrm{eff}} + \phi_c)
  \exp(-2k^2\sigma^2) \notag $$
\item[$\phi_c$]
  The total central atom phase shift, $\phi_c=2\delta_{\ell,c} - \ell\pi$
\item[$F_{\mathrm{eff}}$]
  The effective curved-wave backscattering amplitude in the EXAFS
  formula for each shell.
\item[$\phi_{\mathrm{eff}}$]
  The phase shift for each shell
\item[$\mathcal{R}$]
  The total central atom loss factor, $\mathcal{R}=\exp(-2\Im(\delta_c))$
\item[$R$]
  The distance to the central atom for each shell
\item[$N$]
  The mean number atoms in each shell
\item[$\sigma^2$]
  The mean square fluctuation in $R$ for each shell
\item[$\lambda$]
  The mean free path in \AA, $\lambda = {1/ |\Im p |}$
\item[$k_f$]
  The Fermi momentum at the average interstitial charge density
\item[$p(r)$]
  The local momentum, $p^2(r)=k^2+k_f^2(r)+\Sigma-\Sigma_f$
\item[$\Sigma(E)$]
  The energy dependent self energy at energy $E$, $\Sigma_f$ is the self
  energy at the Fermi energy.
\item[$\mu(E)$]
  The total absorption cross-section
\item[$\mu_0(E)$]
  The embedded atomic background absorption
\end{Reflist}



\section{Program Control Using Intermediate Output Files}
\label{sec:Addit-progr-contr}

In addition to the \htmlref{CONTROL}{card:con} card and other 
options in \file{feff.inp}, some parameters in the files read 
by the various modules can be changed. For example, you can create 
your own paths by editing \file{paths.dat} and explicitly change 
Debye--Waller factors in the final result by editing \file{list.dat}.

Users may edit the some files as a quick and sometimes convenient way
to prepare a given run. It is easiest to use an existing file as a
template, since the code that reads these files is fussy about their format.



\subsection{Using \file{paths.dat}}
\label{sec:paths.dat}

You can modify a path, or even invent new ones, such as paths
with more than the pathfinder maximum of 8 legs. For example, you
could make a path to determine the effect of a focusing atom on a
distant scatterer. Whatever index you enter for the path will be used
in the filename given to the \file{feffNNNN.dat} file. For example,
for the choice of index 845, the EXAFS parameters will appear in
\file{feff0845.dat}.
A handy way to add a single scattering path of length $R$ is to make a
2-leg path with the central atom at (0, 0, 0) and the scatterer at
($R$, 0, 0).

\module{genfmt} will need the positions, unique potentials, and character tags
for each atom in the path. The angles and leg lengths are printed out
for your information, and you can omit them when creating your own
paths by hand. The label lines in the file are required (there is code
that skips them, and if they're missing, you'll get incorrect results).






\subsection{Using \file{list.dat}}
\label{sec:list.dat}

This is the list of files that \module{ff2x} uses to calculate chi. It
includes the paths written by module \module{genfmt}, curved wave importance
factors, and user-defined Debye--Waller factors. If you want to set
Debye--Waller factors for individual paths, you may edit this file to
set them. \module{ff2x} will sum the Debye--Waller factors in this file with
the correlated Debye model $\sigma^2$ and the global $\sigma^2$, if
present. You may also delete paths from this file if you want to
combine some particular set of paths. (CAUTION: Save the original, or
you'll have to re-run \module{genfmt}!)



\subsection{Using \file{geom.dat}}
\label{sec:geom.dat}

This file can be manually edited to take advantage of the paths
symmetries. %See the NOGEOM card. %%currently disabled









%%%%%%%%%%%%%%%%%%%%%%%%%%%%%%%%%%%%%%%%%%%%%%%%%%%%%%%%%%%%%%%%%%%%%%%%%%% APPENDICES %%%%%%%%%%%%%
%%%%%%%%%%%%%%%%%%%%%%%%%%%%%%%%%%%%%%%%%%%%%%%%%%%%%%%%%%%%%%%%%%%%%%%%%%% APPENDICES %%%%%%%%%%%%%

\appendix
%%%%%%%%%%%%%%%%%%%%%%%%%%%%%%%%%%%%%%%%%%%%%%%%%%%%%%%%%%%%%%%%%%%%%%%%%%% COPYRIGHT %%%%%%%%%%%%%

\chapter{Copyright Information,  Restrictions and License}
\label{sec:Append-A-Copyr}

\section{Restrictions and License Information}
\label{sec:Restr-License-Inform}

The full {\feff} distribution is copyrighted software and a
license from the University of
Washington Office of Technology Transfer must be obtained for its use.
This is necessary to protect the interests both of users and the
University of Washington. Both academic/non-profit and commercial
licenses are available --- see Section~\ref{sec:Governm-Copyr}
of this document for details. New users should request the latest
version of this code. The license form may be
obtained from the {\feff} WWW pages,

\centerline{\htmladdnormallink{http://www.feffproject.org}
  {http://www.feffproject.org}}

\noindent or by writing or sending a fax to
\begin{quotation}
\noindent The {\feff} Project\\
c/o {\feff} Project Assistant\\
Department of Physics\\
BOX 351560\\
University of Washington\\
Seattle, WA 98195\\[2ex]
E-mail: \htmladdnormallink{feff@phys.washington.edu}
{mailto:feff@phys.washington.edu}\\
Telephone: (206) 543-4615
Fax: (206) 685-0635 
\end{quotation}


\section{Government Copyrights}
\label{sec:Governm-Copyr}

This work was supported in part by Grants from DOE. In accordance with
the DOE FAR rules part 600.33 ``Rights in Technical Data - Modified
Short Form'' the following clause applies to {\feff}:

(c)(1)The grantee agrees to and does hereby grant to the U.S.
Government and to others acting on its behalf:

(i) A royalty-free, nonexclusive, irrevocable, world-wide license for
Governmental purposes to reproduce, distribute, display, and perform
all copyrightable material first produced or composed in the
performance of this grant by the grantee, its employees or any
individual or concern specifically employed or assigned to originate
and prepare such material and to prepare derivative works based
thereon,

(ii) A license as aforesaid under any and all copyrighted or
copyrightable work not first produced or composed by the grantee in the
performance of this grant but which is incorporated in the material
furnished under the grant, provided that such license shall be only to
the extent the grantee now has, or prior to completion or close-out of
the grant, may acquire the right to grant such license without becoming
liable to pay compensation to others solely because of such grant.

(c)(2) The grantee agrees that it will not knowingly include any
material copyrighted by others in any written or copyrightable material
furnished or delivered under this grant without a license as provided
for in paragraph (c)(1)(ii) of this section, or without the consent of
the copyright owner, unless it obtains specific written approval of the
Contracting Officer for the inclusion of such copyright material.

\section{FEFF9 LICENSE}
\label{sec:F9license}

 FEFF PROGRAMS (referred to below as "the System")
 Copyright (c) 1986-2013, University of Washington.

 END-USER LICENSE

 A signed End-user License Agreement from the University of Washington
 Office of Technology Transfer is required to use these programs and
 subroutines.

 See the URL: http://leonardo.phys.washington.edu/feff/

 USE RESTRICTIONS:

 1. The End-user agrees that neither the System, nor any of its
 components shall be used as the basis of a commercial product, and
 that the System shall not be rewritten or otherwise adapted to
 circumvent the need for obtaining additional license rights.
 Components of the System subject to other license agreements are
 excluded from this restriction.

 2. Modification of the System is permitted, e.g., to facilitate
 its performance by the End-user. Use of the System or any of its
 components for any purpose other than that specified in this Agreement
 requires prior approval in writing from the University of Washington.

 3. The license granted hereunder and the licensed System may not be
 assigned, sublicensed, or otherwise transferred by the End-user.

 4. The End-user shall take reasonable precautions to ensure that
 neither the System nor its components are copied, or transferred out
 side of his/her current academic or government affiliated laboratory
 or disclosed to parties other than the End-user.

 5. In no event shall the End-user install or provide this System
 on any computer system on which the End-user purchases or sells
 computer-related services.

 6. Nothing in this agreement shall be construed as conferring rights
 to use in advertising, publicity, or otherwise any trademark or the
 names of the System or the UW. In published accounts of the use or
 application of FEFF the System should be referred to  by this name,
 with an appropriate literature reference:

 FEFF9: \emph{ Parameter-free calculations of X-ray spectra with FEFF9},
John J. Rehr, Joshua J. Kas, Fernando D. Vila, Micah P. Prange and K. Jorissen,
Phys. Chem. Chem. Phys. 12, 5503-5513 (2010).

 LIMITATION OF LIABILITY:

 1. THE UW MAKES NO WARRANTIES , EITHER EXPRESSED OR IMPLIED, AS TO
 THE CONDITION OF THE SYSTEM, ITS MERCHANTABILITY, OR ITS FITNESS FOR
 ANY PARTICULAR PURPOSE. THE END-USER AGREES TO ACCEPT THE SYSTEM
 'AS IS' AND IT IS UNDERSTOOD THAT THE UW IS NOT OBLIGATED TO PROVIDE
 MAINTENANCE, IMPROVEMENTS, DEBUGGING OR SUPPORT OF ANY KIND.

 2. THE UW SHALL NOT BE LIABLE FOR ANY DIRECT, INDIRECT, SPECIAL,
 INCIDENTAL OR CONSEQUENTIAL DAMAGES SUFFERED BY THE END-USER OR ANY
 OTHER PARTIES FROM THE USE OF THE SYSTEM.

 3. The End-user agrees to indemnify the UW for liability resulting
 from the use of the System by End-user. The End-user and the UW each
 agree to hold the other harmless for their own negligence.

 TITLE:

 1. Title patent, copyright and trademark rights to the System are
 retained by the UW. The End-user shall take all reasonable precautions
 to preserve these rights.

 2. The UW reserves the right to license or grant any other rights to
 the System to other persons or entities.


Note: According to the terms of the above End-user license, 
no part of the standard distributions of {\feff} can be included in other
codes without a license or permission from the authors. However, some subroutines
in {\feff} explicitly contain such a license, and all components subject to
other license agreements are excluded from the restrictions of the
End-user license. Moreover, we are willing to collaborate with other code
developers, and our development version with all the
comments (and subroutines in individual files!) can be made available,
although it makes use of some features that are not standard FORTRAN.
Also we cannot guarantee that any new version of {\feff} will be compatible 
with these subroutines or with any changes you make in the codes.



%%%%%%%%%%%%%%%%%%%%%%%%%%%%%%%%%%%%%%%%%%%%%%%%%%%%%%%%%%%%%%%%%%%%%%%%%%% INSTALLATION %%%%%%%%%%%%%

\chapter{Installation Instructions}
\label{sec:Append-B-Inst}
The structure of the {\feffcur} code differs from its ancestors.
Most users will now receive a package containing the JFEFF graphical user interface (GUI) as a Java program ; and the actual {\feff} code either as a set of bundled precompiled binaries.  This package installs with a few mouse clicks like modern software, and all complexity is hidden.  On first run, the JFEFF GUI copies example files, the users guide, and a few cloud computing files to the user's home folder.  The adventurous user may dig into the JFEFF installation files, where he will find the {\feff} source code in monolithic form: a dozen of very long fortran90 source files suitable for compilation and optimization on the user's computer, but not very convenient for programming work.

The older type of distribution, where the user is given a complex tree of many fortran90 source files, is still available for the expert user who wishes to edit or read the {\feff} source code.

Some versions of JFEFF installed a bash implementation of the "Scientific Cloud Computing Toolset (SC2IT)" to your computer.  This is no longer the case.  The same functionality is now implemented directly in JFEFF.  You can still download the standalone SC2IT-bash from the feffproject website.
  

\section{Installation}
\label{sec:installation}

Installing the GUI is easy enough.  Download the JFEFF installer, run it, and simply let it guide you through the installation steps.  At the end of the procedure, all required files are installed on your computer, and you will be directed to the web page for more information.  The GUI will open and you can try to run an example calculation.  

The folder \url{~/jfeff_examples} contains a large set of examples.  It also contains this Users Guide, and links to system folders containing the {\feff} executables and a \file{feff} script to run all the executables in the right order.  You can add these to your path if you plan to use {\feff} on the command line.

The program files are installed to appropriate system folders.  (On MS Windows, \file{C:/Program Files/JFEFF} .  On Mac OS X, \file{/Applications/jfeff.app} .  On Linux, \url{~/JFEFF} .)  For most operating systems we provide both a 32bit and a 64bit precompiled version of {\feff}.  It is up to you to decide which version is appropriate.  Almost all Mac OS X machines are 64bit, except for very old ones (e.g., a MacBook 1,1).  The Mac OS X executables do not work on PowerPPC machines.  We will generate appropriate executables for you on request - please contact us.  We are interested in making sure {\feff} runs pretty much anywhere, so let us know if you encounter problems. 

On MS Windows and Linux, the installer should place a shortcut on your Desktop that will start JFEFF.  On MS Windows it will also create a Start Menu entry.  On Mac OS X, JFEFF is added to your LaunchPad (Lion only) and the Applications folder in your Dock.  

JFEFF looks for the {\feff} executables in default locations.  If you prefer to use your own set of executables, you can go into the $Settings$ dialog and browse to the correct path there.  It is probably a good idea to look at these $Settings$ anyway.

There are just about a million Linux flavors out there, and we've noticed there can be some unforeseen quirks.  These range from harmless (e.g., the desktop shortcut doesn't work) to more severe (the provided binaries cannot be executed).  We are very interested in hearing about what didn't work out-of-the-box.  However, please do your best to troubleshoot before asking us for help as it may be very difficult for us to replicate your situation.  We may ask you if you can provide us with ssh access to your computer.  A few notable issues have occurred on Ubuntu: some versions seem to ship without java and also without a c-shell installed.  You will need both.  (You can try `which java` and `which csh` from a command line terminal to see if you have these essential packages.)

Now you should be ready to open your first \file{feff.inp} file and start calculating.  The \htmlref{Tutorial Chapter}{sec:tutorial} is an excellent place to start.


Note that the GUI will not work unless you have a recent version of Java.  We believe that Sun Java SE JRE 5.0 or later should be good ; now is a good time to update.  For security reasons it is generally a good idea to have the latest version of Sun Java, which you can download for free online.  If you cannot install Java, you can still use the {\feff} code from the command line.


\section{Compiling your own code}
\label{sec:compilation}

Compiling your own {\feff} binaries is a good idea, as the resulting executables will generally be faster than ours.  If you obtained the {\feff} source tree (probably as a package \file{feff90.tar.gz}), then the code is in \file{feff90/src} - so that's where you need to go.

If you're on Linux/Unix/OS X, you're lucky.  Open the \file{Compiler.mk} file and edit the statement specifying the fortran90 compiler and its compilation options.  Close the file, and type {\it make} on the command line.  If you're interested in using parallel MPI calculations for increased speed on parallel computer infrastructure, give the {\it make mpi} command.  When switching between
sequential and MPI compilation, you should ALWAYS use the {\it make clean} command to flush intermediate files, otherwise the executables may crash and burn.
The resulting executables now ought to be in \file{feff90/bin/Seq} or \file{feff90/bin/MPI}.  These folders will contain a set of 16 executables that together comprise the {\feff} code.  If you plan on using the JFEFF GUI, make sure you edit its Settings accordingly.

Alternatively, if we did not give you the source tree but you installed the JFEFF GUI, you can still compile your own code by going to \url{~/jfeff_examples/feff9code/mod}.  This folder contains the so-called "modular version" of the source code, which is intended just for compilation.  The source tree mentioned above, by contrast, contains the same code but with a separate file for each individual routine, which is much handier for developers but unnecessary for users just compiling for reasons of efficiency.  You will find a \file{Compile} script that you can edit for compiler options and use to compile the {\feff} code.  It is possible that you will have to copy the modular source files to another directory first for reasons of file permissions.

We can make "project files" available for Mac OS X + XCode 3 + Intel Fortran Composer ; and for MS Windows + MS Visual Studio + Intel Fortran Composer .  This is how we produce MS Windows executables.  Alternatively, you could try to make the "Makefile" setup work on MS Windows + Cygwin.

Since the two main dimensioning parameters, $nclusx$ and $lx$, are now dynamically set by the code during each run, there is less need for recompilation
than in older versions of {\feff}.  Other dimensioning parameters, including the maximum number of potentials $nphx$ and the maximum number of spin
states $nspx$, are still static and can only be changed through recompilation.  You may want to edit \file{feff90/src/COMMON/m\_dimsmod.f90} to set these to the desired value before you compile.

We have endeavoured to make {\feff} portable to all architectures without any modification. 
If your machine does not reproduce the test output files \file{xmu.dat}
and/or \file{chi.dat} to high accuracy,
please let us know. Also, please report any compiler problems or warning
messages to the authors, as this will help us achieve full portability.

Compilation requirements have changed from previous versions of {\feff}.  {\feff} is now a fortran90 code, and hence a fortran90 compiler is now required.  If you do not currently have a fortran90 compiler, Intel Fortran Compiler is free for personal use on Linux platforms, and reduced pricing is available for academic licenses.  

We provide precompiled binaries for different architectures, compiled using Intel Fortran Composer 12.5 on all platforms.  We are satisfied with that compiler, but you can also try gfortran, g95, pgf90, or any compiler you like.

We generally recommend starting out with a conservative compilation - all optimization disabled,
maximum safety compiler flags especially for maintaining floating point precision.  Then recompile a
faster version checking against the original results.  Use a testcase containing the SCF and FMS cards
to make sure the FMS routines are safely compiled.

The {\feff} source code is fully self-contained, i.e. it doesn't need to be linked against external libraries.

We've noticed that on some machines it is necessary to increase stack size or available virtual memory
in order to run the code succesfully.  Your sys admin or your system's documentation can help you do this.


%%%%%%%%%%%%%%%%%%%%%%%%%%%%%%%%%%%%%%%%%%%%%%%%%%%%%%%%%%%%%%%%%%%%%%%%%%% REFERENCES %%%%%%%%%%%%%

\chapter{References}
\label{sec:Append-C-Refer}


Please cite {\feff} and an appropriate {\feff} reference if the code or its results are used in published work. 

The main references for the theory of {\feff} are:

\emph{Ab initio theory and calculations of X-ray spectra},
 J.J. Rehr, J.J. Kas, M.P. Prange, A.P. Sorini, Y. Takimoto, F.D. Vila, Comptes Rendus Physique 10 (6) 548-559 (2009)
 
and for EXAFS theory:

\emph{Theoretical Approaches to X-ray Absorption Fine Structure},
J. J. Rehr and R. C. Albers, Rev. Mod. Phys. {\bf72}, 621, (2000).
 

The main reference for calculations using the current version {\feffcur}  is:

\emph{Parameter-free calculations of X-ray spectra with FEFF9}, J.J. Rehr, J.J. Kas, F.D. Vila, M.P. Prange, K. Jorissen, Phys. Chem. Chem. Phys. 12 (21) 5503-5513 (2010).


\medskip
\medskip
Other publications discussing specific {\feff} developments are given below.

\begin{Reflist}

\item [\textit{FEFF9}]  ``Parameter-free calculations of X-ray spectra with FEFF9", J.J. Rehr, J.J. Kas, F.D. Vila, M.P. Prange, K. Jorissen, Phys. Chem. Chem. Phys. 12 (21) 5503-5513 (2010).

\item [\textit{K-space}]  ``Calculations of electron energy loss and x-ray absorption spectra in periodic systems without
a supercell," K. Jorissen and J.J. Rehr, Phys. Rev. B 81, 245124 (2010).

\item [\textit{EELS}] ``Multiple scattering calculatinos of relativistic electron
energy loss spectra," K. Jorissen, J.~J. Rehr, and J. Verbeeck,
 Phys. Rev. B 81, 155108 (2010).

\item [\textit{NRIXS}] ``Inelastic Scattering from Core-electrons: a Multiple Scattering
Approach," J.~A. Soininen, A.~L. Ankudinov, and J.~J. Rehr,
Phys. Rev. B 72, 045136 (2005).

\item [\textit{Inelastic Mean Free Path}]  ``Ab initio calculations of mean free paths and stopping powers,"
A. P. Sorini, J. J. Kas, J. J. Rehr, M. P. Prange and Z. H. Levine,
Phys. Rev. B {\bf 74}, 165111 (2006).

\item [\textit{Debye-Waller Factors}] ``Theoretical X-Ray Absorption Debye-Waller Factors,"
Fernando D. Vila, J. J. Rehr, H. H. Rossner, H. J. Krappe,
Phys. Rev. B {\bf76}, 014301 (2007).

\item [\textit{Many-Pole Self Energy}]  ``Many-pole model of inelastic losses in x-ray absorption spectra,"
J.J. Kas, A. P. Sorini, M. P. Prange, L. W. Cambell, and J. A. Soininen,
and J. J. Rehr, Phys. Rev. B {\bf 76}, 195116 (2007).

\item [\textit{FEFF8}]  ``Real Space Multiple Scattering
    Calculation of XANES," A.L.\ Ankudinov, B.\ Ravel,
  J.J.\ Rehr, and S.D.\ Conradson, Phys.\ Rev.\ B \textbf{58}, 7565 (1998).

\item [\textit{Parallellization and Lanczos}] ``Parallel calculation of electron multiple scattering 
   using Lanczos algorithms," A.L.\ Ankudinov, C.\ Bouldin, J.J.\ Rehr, J.\ Sims, H.\ Hung,
    Phys.\ Rev.\ B \textbf{65}, 104107 (2002).
 
\item [\textit{TDLDA}] ``Dynamic screening effects in x-ray absorption spectra," A.L.\ Ankudinov, A.I.\ Nesvizhskii, and J.J.\ Rehr,
Phys.\ Rev.\ B 67, 115120 (2003). 

\item [\textit{EELS}] ``Practical aspects of
electron energy-loss spectroscopy (EELS) calculations using FEFF8,"M.S.\ Moreno, K.\ Jorissen, and J.J.\ Rehr, 
Micron, {\bf38}, 1 (2007).
 
\item [\textit{FEFF7}] A.L.\ Ankudinov and J.J.\ Rehr, \emph{Relativistic
    Spin-dependent X-ray Absorption Theory}, Phys.\ Rev.\ B \textbf{56},
  R1712 (1997).
 
  A.L.\ Ankudinov, PhD Thesis, \emph{Relativistic Spin-dependent
    X-ray Absorption Theory}, University of Washington, (1996); 
  contains a review of x-ray absorption theory, a whole chapter
  of information about {\feff} for expert users, example
  applications, and the full {\feff}7 program tree.
 
\item [\textit{FEFF6}] S.I.\ Zabinsky, J.J.\ Rehr, A.\ Ankudinov, R.C.\
  Albers and M.J.\ Eller, \emph{Multiple Scattering Calculations of
    X-ray Absorption Spectra}, Phys.\ Rev.\ B \textbf{52}, 2995 (1995).
 
\item [\textit{FEFF5}] J.J.\ Rehr, S.I.\ Zabinsky and R.C.\ Albers,
  \emph{High-order multiple scattering calculations of
    x-ray-absorption fine structure}, Phys.\ Rev.\ Lett.\ \textbf{69},
  3397 (1992).
  
\item [\textit{FEFF4}] J.\ Mustre de Leon, J.J.\ Rehr, S.I.\
  Zabinsky, and R.C.\ Albers, \emph{Ab initio curved-wave
    x-ray-absorption fine structure}, Phys.\ Rev.\ B \textbf{44}, 4146
  (1991).
 
 J.J.\ Rehr, J.\ Mustre de Leon, S.I.\ Zabinsky, and
  R.C.\ Albers, \emph{Theoretical X-ray Absorption Fine Structure
    Standards}, J.\ Am.\ Chem.\ Soc.\ \textbf{113}, 5135 (1991).
 
\item [\textit{Sum rules}]
A.I.\ Nesvizhskii, A.L.\ Ankudinov, and J.J.\ Rehr,
  \emph{Normalization and convergence of x-ray absorption sum
  rules}, Phys.\ Rev.\ B \textbf{63}, 094412 (2001).
 
\item [\textit{Multiple Scattering theory}] J.J.\ Rehr and R.C.\ Albers,
  \emph{Scattering-matrix formulation of curved-wave
    multiple-scattering theory: Application to x-ray-absorption fine
    structure}, Phys.\ Rev.\ B \textbf{41}, 8139 (1990).
 
\item [\textit{Dirac-Fock atomic code}] A.L.\ Ankudinov, S.I.\ Zabinsky and J.J.\
  Rehr, \emph{Single configuration Dirac-Fock atom code}, Comp.\ Phys.\
  Comm.\ \textbf{98}, 359 (1996).

\end{Reflist}


%%%%%%%%%%%%%%%%%%%%%%%%%%%%%%%%%%%%%%%%%%%%%%%%%%%%%%%%%%%%%%%%%%%%%%%%%%% CODE DIMENSIONS %%%%%%%%%%%%%

\chapter{Code Variables and Dimensions}
\label{sec:Append-D-Code}
For historical reasons {\feff} is a mix of older fortran77 code and newer
fortran90 code. Hence the most important array dimensions are treated dynamically,
while others are set at compilation time.  If you need
to run larger problems than the dimension statements in the code
allow, you must change the dimensions in the file \file{feff90/src/COMMON/m\_dimsmod.f90} and recompile.
If you need help, please contact the authors.


The following parameters are supported for dynamical treatment :
\begin{verbatim}
c      number of atoms for FMS.
       integer nclusx
c      highest orbital momentum for FMS module.
       integer lx
\end{verbatim}

For these parameters, the module \module{rdinp} will determine from user input what array size is needed, eg.
to fit all atoms of the cluster and all orbital momentum values.  These will be truncated to a precompiled upper
limit to protect exceeding available memory (which might result in nasty errors that would be problematic
especially on large computing systems where tasks have to go through a queue and crashes cannot be immediately
rectified) ; optionally, the user can set these truncation values using the \htmlref{DIMS}{card:dim} card.

For all static parameters below, all arrays will be allocated with a fixed dimension regardless of the
problem at hand.  These parameters are (values may be different in your release) :


\begin{verbatim}
c      max number of spins: 1 for spin average; 2 for spin-dep
       parameter (nspx=1)
c      max number of atoms in problem for the pathfinder
       parameter (natx =2000)
c      max number of atoms in problem for the rdinp and ffsort
       parameter (nattx =10000)
c      max number of unique potentials (potph) (nphx must be ODD to 
c      avoid compilation warnings about alignment in COMMON blocks)
       parameter (nphx = 9)
c      max number of ang mom (arrays 1:ltot+1)
       parameter (ltot = 24)
c      Loucks r grid used through overlap and in phase work arrays
       parameter (nrptx = 1251)
c      Number of energy points genfmt, etc.
       parameter (nex = 450)
c      Max number of distinct lambda's for genfmt
c      15 handles iord 2 and exact ss
       parameter (lamtot=15)
c      vary mmax and nmax independently
       parameter (mtot=4, ntot=2)
c      max number of path atoms, used in path finder, NOT in genfmt
       parameter (npatx = 8)
c      matches path finder, used in GENFMT
       parameter (legtot=npatx+1)
c      max number of overlap shells (OVERLAP card)
       parameter (novrx=8)
c      max number of header lines
       parameter (nheadx=30)
       
\end{verbatim}


Of these parameters, a user is most likely to want to change nphx and nspx.
Care should be taken in changing these parameters.  Whenever the expected length of an array changes,
errors, or, worse, unphysical results may occur when {\feff} then reads intermediate files produced
before the changes.  In practice, this means that
\begin{itemize}
\item whenever one changes user input (\file{feff.inp}) or the hardwired truncation limits for the dynamical
dimension parameters such that the length of any arrays in the calculation will change, all affected parts
of the calculation must be rerun before continuing or reusing parts of it.
\item whenever one changes any of the static parameters, all intermediate files containing arrays dimensioned by
the affected parameter will become useless and the calculation must be redone to regenerate these files before they can be used.
\end{itemize}

Of course, it is entirely possible to have more than one compiled version of {\feff} on your system, as long as you
don't mix them up and always use the same version for a given calculation.

It should also be noted here that there is an internal limit on the 
number of paths (set to 1200) that will be read from \file{feff.bin}. 
This limit was chosen to handle any reasonable problem without using 
an excessive amount of memory. If you must use more paths, change the 
parameter \texttt{npx} in the {\feff} source in subroutine \texttt{ff2x} 
to whatever you need. This will require more memory. We have not had 
a case where the filter criteria were not able to solve the problem 
with fewer than 1200 paths.

The files \file{.feff.dims} and \file{.feff.comp.stats} list the values of all important parameters used in a given
calculation.    It'd be a really bad idea to edit these files, but looking won't hurt.


%%%%%%%%%%%%%%%%%%%%%%%%%%%%%%%%%%%%%%%%%%%%%%%%%%%%%%%%%%%%%%%%%%%%%%%%%%% SPIN.F PROGRAM %%%%%%%%%%%%%

\chapter{Spin.f program for XMCD and SPXAFS}
\label{sec:Append-F-Spinf}
\file{spin.f}:
\begin{verbatim}
      program spin_sum
c    written in grandmother's fortran-77   
      implicit double precision (a-h,o-z)
c     This program read two xmu.dat files for spin -up and -down,
c     calculated with Feff8.20 for the SAME paths list.
c     spin-up file is fort.1, spin-down file is fort.2
c     Both have to be edited: All lines should be deleted except
c       1) line: xsedge+100, used to normalize mu           1.3953E-04
c          leave only on this line:  1.3953E-04
c       2) 6-column data lines
c     The output will be written in fort.3 in 6 columns
c     E+shift1  E(edge)+shift2  xk cmd_total cmd_background  cmd_fs
c     where total = atomic background + fine structure

c     There are 3 possibilities
c     case 1) you want XMCD signal and used SPIN  1 or -1
c     case 2) you want XMCD signal and used SPIN 2 or -2, in order
c       to use non-relativistic formula for XMCD
c       factor li/2j+1 which was not convenient to do in a program
c     case 3) you want SPXAFS and used  SPIN \pm 2
c     ENTER your case here (icase is positive integer only)
      icase = 2

c     if icase=2 ENTER factor=(-1)**(L+1/2-J) * L/(2*J+1)
c     where L,J are for your edge (ex. for L3 L=1 J=3/2, for L2 L=1 J=1/2)
c     for L3
c      factor = 0.25
c     for L2
c     factor = -0.5

c     ENTER the energy shift you want for columns 1 and 2 in xmu.dat
      shift1 = 0
      shift2 = 0

c     everything below is automated further
      read (1,*,end=10) ap
      read (2,*,end=10) am
      xnorm = 0.5 *(ap+am)
c     read the data
   3     read(1,*,end=10)   x1, x2, ek, y1, y2, y3
         read(2,*,end=10)  x1, x2, ek, z1, z2, z3
         if (icase.eq.1) then
c           no xafs in this case:xfs - atomic part of XMCD
            t1 = (y1*ap + z1*am)/xnorm
            t2 = (y2*ap + z2*am)/xnorm
            t3 = (y3*ap + z3*am) /xnorm

         elseif (icase.eq.2) then
            t1 = (y1*ap - z1*am)*factor /xnorm
            t2 = (y2*ap - z2*am)*factor /xnorm
            t3 = (y3*ap - z3*am)*factor /xnorm

         elseif (icase.eq.3) then
c           factor=0.5 always for SPXAFS
            t1 = (y1*ap - z1*am)/2.0/xnorm
            t2 = (y2*ap - z2*am)/2.0/xnorm
            t3 = (y3*ap - z3*am)/2.0/xnorm
c           you may want average total XAS as output in last column
c           t3 = (y1*ap + z1*am)/2.0/xnorm
         endif
         x1 =x1 + shift1
         x2 =x2 + shift2
         write(3,5)    x1, x2, ek, t1, t2, t3
   5     format (6e13.5)
      goto 3
  10  continue
      stop
      end
\end{verbatim}


%%%%%%%%%%%%%%%%%%%%%%%%%%%%%%%%%%%%%%%%%%%%%%%%%%%%%%%%%%%%%%%%%%%%%%%%%%% DMDW %%%%%%%%%%%%%

\chapter{Technical notes for dynamical matrix based Debye-Waller factors}
\label{sec:Append-G-DMDW}
DMDW is a set of tools developed to calculate Debye-Waller (DW) factors and
other related quantities from a dynamical matrix (matrix of force constants or
Hessian matrix) using the Lanczos recursive algorithm.[Refs.] This set includes
a module integrated into FEFF, a standalone version that can be used
independently of FEFF and a Fortran module that can be integrated into
third-party programs. DMDW also includes conversion tools to generate the
required input files from different ab initio programs.

\section{Installing DMDW}
\label{sec:Append-G-DMDW-Install}
\subsection{Installing as a FEFF module}
DMDW is included in the {\feff} distribution and is installed as a {\feff} module with all the other program modules.
Please refer to the {\feff} installation instructions. After the {\feff} installation is completed a
module called \module{dmdw} is located under the {\feff} \file{feff90/bin/Seq} directory structure.

\subsection{Creating and installing the standalone version}
Within the \file{src/DMDW} directory in the {\feff} distribution, execute the following
command:
\begin{verbatim}
make standalone
\end{verbatim}
This will create a directory \file{feff90/src/DMDW/dmdw\_standalone}. Edit \file{dmdw\_standalone/src/Makefile}
and change the "F90" variable to your fortran 90 compiler of choice. Then execute the following commands:
\begin{verbatim}
make
make install
make examples
\end{verbatim}
If everything worked correctly the reference results located in
\file{dmdw\_standalone/examples/Reference\_Results} should agree with those generated
in \file{dmdw\_standalone/examples} using the newly compiled executable.

\section{Using DMDW to calculate Debye-Waller factors}
\label{sec:Append-G-DMDW-Use}
\subsection{Using within a XANES or EXAFS calculation in FEFF}
See \htmlref{Chapter 5}{sec:DWfactors}.

\subsection{Calculating DW factors using the standalone version}
Capabilities of DMDW beyond the {\feff} DEBYE card can be accessed by means of the \module{dmdw} module or by
compiling the standalone version. All the detailes described below apply to
the input used by both the \module{dmdw} module and the standalone version. During the
execution of a normal FEFF run using ab initio DW factors, an input file \file{dmdw.inp} for the
\module{dmdw} module is automatically generated based on the options used in the DEBYE
card. This autogenerated input can be used "as is" with the standalone version,
or further edited to access other capabilities.

\file{dmdw.inp} is composed of at least 5 lines of input.  All the
parameters must be entered and no default values are available:
\begin{verbatim}
Lanczos_Order
nT T_Min T_Max
DW_Type
Filename
nPathDesc
PathDesc1
PathDesc2
.
.
\end{verbatim}
where:
\begin{itemize}
\item Line  1 - Lanczos\_Order: Number of Lanczos iterations (integer).

  This parameter is equivalent to the DMDW\_Order parameter described for
  the DEBYE card. It corresponds to the number of Lanczos iterations to be used
  in the calculation. Well converged results are usually obtained for
  $DMDW\_Order=6-10$. For small size systems, these values might be too large. As a
  rule of thumb, this value should be less than $3*(Number of atoms)-6$. Some
  paths in systems with high symmetry might require a lower recursion
  order. The user should always check for convergence with this parameter.

\item Line  2 - nT: Number of temperature values in grid (integer)
          T\_Min, T\_Max: Minimum and maximum temperature values (real, in K)

  Define a grid of temperatures in which to calculate the DW factors. This
  option is very efficient in the generation of whole temperature curves since
  it performs the Lanczos procedure only once and then calculates the DW for
  each temperature.

\item Line  3 - DMDW\_Type: Type of DW calculation (integer).

  This parameter is equivalent to the one described for the DEBYE card,
  but more options are available.
  The possible values of DMDW\_Type are:
	0	Parallel $s^{2}$
	1	Perpendicular$s^{2}$
	2	Crystallographic $u^{2}$
	3	Vibrational free energy calculation
	
  The parallel $s^{2}$ is the usual mean-square relative displacement (MSRD) along a
  path. The perpendicular $s^{2}$ is the MSRD orthogonal to a path. The
  crystallographic $u^{2}$ is the mean-square displacement of a given atom with
  respect to its stationary position. Finally, the vibrational free energy
  associated with that crystallographic $u^{2}$ can also be calculated.

  [Options 1-3 are not fully activated in this release. Also, 
  the meaning of this parameter might change in a future release]

\item Line  4 - Filename: Name of file containing the dynamical matrix (string)

  The file must be present in the same directory as the DMDW input and be in
  "dym" format (see below).

\item Line  5 - nPathDesc: Number of path descriptors (integer)

  Define the number of path descriptors to use for the generation of paths.
  
\item Lines 6... - PathDescN: Nth path descriptor used to generate a list of paths
                        (integer and real, see below)

  A path descriptor has the following form:
\begin{verbatim}  
    nAt At(1)...At(nAt) Path\_Length
\end{verbatim}    
  where:
  
    nAt: Number of atoms in the path (integer)
    At(i): Index of atom that must be included in the path (integer)

		These indices correspond to the ones used in the "dym" file. The
		number 0 is a wildcard representing any atom in the structure.
		For instance, the atom indices "1 0 2" represent a double
		scattering paths starting at atom 1, ending at atom 2 and
		passing through every other allowed atom in the system. The
		paths are generated in such a way that no consecutive repeated
		indices are allowed.

    Path\_Length: Effective path length cutoff (real, in Bohr)
    
		This parameter helps fine-tune the generated path list, removing
		paths that are longer than necessary.
\end{itemize}



 \section{The "dym" dynamical matrix file format}

  A "dym" file contains the information required by the Lanczos algorithm. This
  includes the atomic masses, structure and force constants. Two conversion
  scripts are included in the \file{feff90/bin/Seq} directory to convert Gaussian 03 formatted
  checkpoint ("fchk") files (fchk2dym) and Quantum Espresso dynamical matrix
  files (dynG2dym) into our "dym" dynamical matrix format. The "fchk2dym"
  command has been thoroughly tested, but the "dynG2dym" has not. 
  
  We have also used dynamical matrices calculated by the ABINIT program.  However, we don't currently offer
  automated tools for this scenario.  Users may contact the authors for assistance using ABINIT and \module{dmdw}.
  
  The current format of the "dym" files is as follows:
\begin{itemize}
\item Line 1 - dym\_Type: Dynamical matrix file type (integer)

  This value is for future use. Set to 1 for now.

\item Line 2 - nAt: Number of atoms (integer)

  Number of atoms in the system.
  
\item Lines 2..2+nAt - Atomic numbers (integer)

  Atomic numbers of atoms in the system.

\item Lines 2+nAt+1..2+2*nAt - Atomic masses (real, in AMU)

  Atomic masses of the atoms in the system.

\item Lines 2+2*nAt+1..2+3*nAt - Atomic coordinates (real, in Bohr)

  Cartesian coordinates ("x y z") of the atoms in the system.

\item Lines 2+3*nAt+1..End - Dynamical matrix in atom pair block format (integer and
                       real, see below,in atomic units):

  The force constants in the system are stored for each pair of atoms in the
  system using the following block format:

     i j
     d2E/dxidxj d2E/dxidyj d2E/dxidzj
     d2E/dyidxj d2E/dyidyj d2E/dyidzj
     d2E/dzidxj d2E/dzidyj d2E/dzidzj

  where:
  
     i, j: Indices defining the atomic pair
     d2E/daidbj: Second derivative of the energy (i.e. force constant) with
                 respect to the a coordinate of atom i and the b coordinate of
		 atom j, where a,b={x,y,z}.
\end{itemize}

Example "dym" files can be found in the \file{feff90/test/DMDW} directory.


\section{Examples}
\label{sec:Append-G-DMDW-Example}

See \htmlref{Chapter 5}{sec:DWfactors}.

\section{Troubleshooting \module{dmdw}}
\label{sec:Append-G-DMDW-Trouble}

\subsection{Lanczos problems}
\begin{itemize}
\item If the code warns that there are less poles than Lanczos iterations, it
  usually means that the iteration order is too high. Try with a smaller
  number.

\item When the structure associated with a dynamical matrix is not sufficiently
  optimized, the program is likely to report that certain paths result in
  poles associated with imaginary frequencies. The code currently ignores
  these poles by setting ther weigth to zero. Usually this doesn't affect the
  results significantly, but they should be considered very carefully anyway.

\item The code checks the symmetry of the dynamical matrix. If it isn't sufficiently
  symmetric, the results should be examined carefully.
\end{itemize}


%%%%%%%%%%%%%%%%%%%%%%%%%%%%%%%%%%%%%%%%%%%%%%%%%%%%%%%%%%%%%%%%%%%%%%%%%%% TROUBLESHOOTING %%%%%%%%%%%%%

\chapter{Trouble-Shooting {\feff} Problems and Bug Reports}
\label{sec:Append-H-Trouble}

{\feff} has been extensively tested on many different architectures,
but occasionally new bugs show up. In an effort to maintain portable
and trouble-free codes, we take all bug reports seriously. Please
let us know if you encounter any compilation error or warning messages.
Often we receive reports by users of older
versions of {\feff} of bugs that have been fixed in more recent
releases. Other code failures can be traced to input file
errors, sometimes quite subtle, and some are compiler bugs, for which
we try to find a workaround.

To  report a bug, please tell us the version of the code you are using
and which operating system and compiler you have. Please include a
\file{feff.inp} if the problem occurs after compilation
and enough detail concerning the warning or error messages or
other difficulties you have so that we can attempt to reproduce the problem.

Some known and commonly encountered difficulties are:
\begin{itemize}
\item Non-physical, widely spaced distributions of atoms. Symptoms of
  this common problem are very large muffin-tin radii (see the header
  of any \file{.dat} file) and possibly a failure of the phase-shift
  program to converge. This gives error message \texttt{hard test
    fails in fovrg}.
\item An error in assigning potential indices; the first atom with a
  given potential index must have the geometry representative of this
  potential type. This is sometimes fixed by using a somewhat larger
  cluster; in fact it is usually desirable to have a larger cluster
  for potential construction than that used in the XAFS calculation
  due to errors in the potentials at surfaces. Unless the atom
  distribution is physically possible, you can expect the code to have
  problems.
\item Hash collision in the pathfinder. This is now rare, but can
  usually be corrected simply by changing distances in the fourth
  decimal place.
\item For the $M_{\mathit{IV}}$ and higher edges you may receive the
  error message like: \texttt{Lambda array overfilled}. The
  calculations should be repeated with IORDER -70202 card.
\item For systems with a large number of potentials (eg. adamantane,
  a system of C and H containing 26 different potentials) small negative numbers
  occur during the SCF calculation (negative density warning) and the DOS
  calculation (negative "`noise"' in the ldos).  This is due to limited precision 
  in the FMS routines with so many different potentials, and probably also more likely
  for very light atoms.  For reasons of efficiency, the FMS routines work in single
  precision.  One can eliminate the negative numbers by recompiling {\feff} using
  double precision as default for real numbers (eg. using the -r8 option for the ifort
  compiler).  However, this will make the calculation significantly slower.  Usually it is enough to look at the DOS and make sure it is physical.
\end{itemize}

\end{latexonly} % begins before chapter 3

\end{document}



%%% Local Variables:
%%% mode: latex
%%% TeX-master: t
%%% End:
